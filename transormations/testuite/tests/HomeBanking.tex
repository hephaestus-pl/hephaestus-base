\documentclass{article}

\usepackage{a4wide}

%% ODER: format ==         = "\mathrel{==}"
%% ODER: format /=         = "\neq "
%
%
\makeatletter
\@ifundefined{lhs2tex.lhs2tex.sty.read}%
  {\@namedef{lhs2tex.lhs2tex.sty.read}{}%
   \newcommand\SkipToFmtEnd{}%
   \newcommand\EndFmtInput{}%
   \long\def\SkipToFmtEnd#1\EndFmtInput{}%
  }\SkipToFmtEnd

\newcommand\ReadOnlyOnce[1]{\@ifundefined{#1}{\@namedef{#1}{}}\SkipToFmtEnd}
\usepackage{amstext}
\usepackage{amssymb}
\usepackage{stmaryrd}
\DeclareFontFamily{OT1}{cmtex}{}
\DeclareFontShape{OT1}{cmtex}{m}{n}
  {<5><6><7><8>cmtex8
   <9>cmtex9
   <10><10.95><12><14.4><17.28><20.74><24.88>cmtex10}{}
\DeclareFontShape{OT1}{cmtex}{m}{it}
  {<-> ssub * cmtt/m/it}{}
\newcommand{\texfamily}{\fontfamily{cmtex}\selectfont}
\DeclareFontShape{OT1}{cmtt}{bx}{n}
  {<5><6><7><8>cmtt8
   <9>cmbtt9
   <10><10.95><12><14.4><17.28><20.74><24.88>cmbtt10}{}
\DeclareFontShape{OT1}{cmtex}{bx}{n}
  {<-> ssub * cmtt/bx/n}{}
\newcommand{\tex}[1]{\text{\texfamily#1}}	% NEU

\newcommand{\Sp}{\hskip.33334em\relax}


\newcommand{\Conid}[1]{\mathit{#1}}
\newcommand{\Varid}[1]{\mathit{#1}}
\newcommand{\anonymous}{\kern0.06em \vbox{\hrule\@width.5em}}
\newcommand{\plus}{\mathbin{+\!\!\!+}}
\newcommand{\bind}{\mathbin{>\!\!\!>\mkern-6.7mu=}}
\newcommand{\rbind}{\mathbin{=\mkern-6.7mu<\!\!\!<}}% suggested by Neil Mitchell
\newcommand{\sequ}{\mathbin{>\!\!\!>}}
\renewcommand{\leq}{\leqslant}
\renewcommand{\geq}{\geqslant}
\usepackage{polytable}

%mathindent has to be defined
\@ifundefined{mathindent}%
  {\newdimen\mathindent\mathindent\leftmargini}%
  {}%

\def\resethooks{%
  \global\let\SaveRestoreHook\empty
  \global\let\ColumnHook\empty}
\newcommand*{\savecolumns}[1][default]%
  {\g@addto@macro\SaveRestoreHook{\savecolumns[#1]}}
\newcommand*{\restorecolumns}[1][default]%
  {\g@addto@macro\SaveRestoreHook{\restorecolumns[#1]}}
\newcommand*{\aligncolumn}[2]%
  {\g@addto@macro\ColumnHook{\column{#1}{#2}}}

\resethooks

\newcommand{\onelinecommentchars}{\quad-{}- }
\newcommand{\commentbeginchars}{\enskip\{-}
\newcommand{\commentendchars}{-\}\enskip}

\newcommand{\visiblecomments}{%
  \let\onelinecomment=\onelinecommentchars
  \let\commentbegin=\commentbeginchars
  \let\commentend=\commentendchars}

\newcommand{\invisiblecomments}{%
  \let\onelinecomment=\empty
  \let\commentbegin=\empty
  \let\commentend=\empty}

\visiblecomments

\newlength{\blanklineskip}
\setlength{\blanklineskip}{0.66084ex}

\newcommand{\hsindent}[1]{\quad}% default is fixed indentation
\let\hspre\empty
\let\hspost\empty
\newcommand{\NB}{\textbf{NB}}
\newcommand{\Todo}[1]{$\langle$\textbf{To do:}~#1$\rangle$}

\EndFmtInput
\makeatother
%
%
%
%
%
%
% This package provides two environments suitable to take the place
% of hscode, called "plainhscode" and "arrayhscode". 
%
% The plain environment surrounds each code block by vertical space,
% and it uses \abovedisplayskip and \belowdisplayskip to get spacing
% similar to formulas. Note that if these dimensions are changed,
% the spacing around displayed math formulas changes as well.
% All code is indented using \leftskip.
%
% Changed 19.08.2004 to reflect changes in colorcode. Should work with
% CodeGroup.sty.
%
\ReadOnlyOnce{polycode.fmt}%
\makeatletter

\newcommand{\hsnewpar}[1]%
  {{\parskip=0pt\parindent=0pt\par\vskip #1\noindent}}

% can be used, for instance, to redefine the code size, by setting the
% command to \small or something alike
\newcommand{\hscodestyle}{}

% The command \sethscode can be used to switch the code formatting
% behaviour by mapping the hscode environment in the subst directive
% to a new LaTeX environment.

\newcommand{\sethscode}[1]%
  {\expandafter\let\expandafter\hscode\csname #1\endcsname
   \expandafter\let\expandafter\endhscode\csname end#1\endcsname}

% "compatibility" mode restores the non-polycode.fmt layout.

\newenvironment{compathscode}%
  {\par\noindent
   \advance\leftskip\mathindent
   \hscodestyle
   \let\\=\@normalcr
   \(\pboxed}%
  {\endpboxed\)%
   \par\noindent
   \ignorespacesafterend}

\newcommand{\compaths}{\sethscode{compathscode}}

% "plain" mode is the proposed default.
% It should now work with \centering.
% This required some changes. The old version
% is still available for reference as oldplainhscode.

\newenvironment{plainhscode}%
  {\hsnewpar\abovedisplayskip
   \advance\leftskip\mathindent
   \hscodestyle
   \let\hspre\(\let\hspost\)%
   \pboxed}%
  {\endpboxed%
   \hsnewpar\belowdisplayskip
   \ignorespacesafterend}

\newenvironment{oldplainhscode}%
  {\hsnewpar\abovedisplayskip
   \advance\leftskip\mathindent
   \hscodestyle
   \let\\=\@normalcr
   \(\pboxed}%
  {\endpboxed\)%
   \hsnewpar\belowdisplayskip
   \ignorespacesafterend}

% Here, we make plainhscode the default environment.

\newcommand{\plainhs}{\sethscode{plainhscode}}
\newcommand{\oldplainhs}{\sethscode{oldplainhscode}}
\plainhs

% The arrayhscode is like plain, but makes use of polytable's
% parray environment which disallows page breaks in code blocks.

\newenvironment{arrayhscode}%
  {\hsnewpar\abovedisplayskip
   \advance\leftskip\mathindent
   \hscodestyle
   \let\\=\@normalcr
   \(\parray}%
  {\endparray\)%
   \hsnewpar\belowdisplayskip
   \ignorespacesafterend}

\newcommand{\arrayhs}{\sethscode{arrayhscode}}

% The mathhscode environment also makes use of polytable's parray 
% environment. It is supposed to be used only inside math mode 
% (I used it to typeset the type rules in my thesis).

\newenvironment{mathhscode}%
  {\parray}{\endparray}

\newcommand{\mathhs}{\sethscode{mathhscode}}

% texths is similar to mathhs, but works in text mode.

\newenvironment{texthscode}%
  {\(\parray}{\endparray\)}

\newcommand{\texths}{\sethscode{texthscode}}

% The framed environment places code in a framed box.

\def\codeframewidth{\arrayrulewidth}
\RequirePackage{calc}

\newenvironment{framedhscode}%
  {\parskip=\abovedisplayskip\par\noindent
   \hscodestyle
   \arrayrulewidth=\codeframewidth
   \tabular{@{}|p{\linewidth-2\arraycolsep-2\arrayrulewidth-2pt}|@{}}%
   \hline\framedhslinecorrect\\{-1.5ex}%
   \let\endoflinesave=\\
   \let\\=\@normalcr
   \(\pboxed}%
  {\endpboxed\)%
   \framedhslinecorrect\endoflinesave{.5ex}\hline
   \endtabular
   \parskip=\belowdisplayskip\par\noindent
   \ignorespacesafterend}

\newcommand{\framedhslinecorrect}[2]%
  {#1[#2]}

\newcommand{\framedhs}{\sethscode{framedhscode}}

% The inlinehscode environment is an experimental environment
% that can be used to typeset displayed code inline.

\newenvironment{inlinehscode}%
  {\(\def\column##1##2{}%
   \let\>\undefined\let\<\undefined\let\\\undefined
   \newcommand\>[1][]{}\newcommand\<[1][]{}\newcommand\\[1][]{}%
   \def\fromto##1##2##3{##3}%
   \def\nextline{}}{\) }%

\newcommand{\inlinehs}{\sethscode{inlinehscode}}

% The joincode environment is a separate environment that
% can be used to surround and thereby connect multiple code
% blocks.

\newenvironment{joincode}%
  {\let\orighscode=\hscode
   \let\origendhscode=\endhscode
   \def\endhscode{\def\hscode{\endgroup\def\@currenvir{hscode}\\}\begingroup}
   %\let\SaveRestoreHook=\empty
   %\let\ColumnHook=\empty
   %\let\resethooks=\empty
   \orighscode\def\hscode{\endgroup\def\@currenvir{hscode}}}%
  {\origendhscode
   \global\let\hscode=\orighscode
   \global\let\endhscode=\origendhscode}%

\makeatother
\EndFmtInput
%



\title{A Simple Module for Experimenting with Advice Compositions}

\author{Rodrigo Bonif\'{a}cio}

\begin{document}

\maketitle

\section{Introduction}

This document is a \emph{Literate Haskell} (LHS) module, proposed for 
experimenting with advice compositions. Basically, it allow us to 
understand the effect of applying the different types of advices 
(Before, After, and Around) using the Home Banking product line. 



\section{List of Scenarios}

The scenarios shown in Figures~\ref{} are considered in this module:
Using our approach, a scenario could be extended by some advice, if a selection of features was selected or not. 
Nevertheless, in this document we are primarly interested in testing and evaluating the 
\texttt{evalFunction}, which takes as input an advice \texttt{adv} and a 
scenario \texttt{sc}, and then returns a new scenario that is the result of composition 
between them. One interesting point about using LHS is that one could show other scenarios in 
this document, just introducing a command sucha as: 

\begin{tabbing}\tt
~\char92{}perform\char123{}let~s~\char61{}~show~\char40{}scenarioToLatex~SCID\char41{}~in~putStrLn~\char40{}formatString~s\char41{}\char125{}
\end{tabbing}

This command guides the lhs2tex compile to call the GHC Interpreter and 
to evaluate the \texttt{let ...} expression. 

\begin{figure}

\subsubsection*{Scenario sc01} \begin{tabbing} {\bf Description:} \= This scenario allows a customer to withdraw money from a previously selected account \\ {\bf From steps:} \> start \\ {\bf To steps:} \> end \\ \end{tabbing} \begin{tabular}{|p{0.4in}|p{1.2in}|p{1.2in}|p{1.2in}|p{1.2in}|} \hline Id & User Action & Condition & System Response & Annotations \\ \hline sc0101 & The customer selects the withdraw option. & - & The system creates a new withdrawal and asks for the amount to withdraw. & \\ \hline sc0102 & The customer fills in the amount to withdraw. & - & The system retrieves the current balance of the selected account. & \\ \hline sc0103 & - & - & The system verifies that the requested amount is not greater than current balance plus U\$ 5000.00 & authentication \\ \hline sc0104 & - & - & The bank system withdraws the amount from the account. & transaction \\ \hline sc0105 & - & - & The cash money is provided to the customer. & \\ \hline \end{tabular}
\caption{Withdraw money transaction.}
\label{fig:sc01}
\end{figure}

\newpage

\section{List of Advice}

Next we enumerate each advice considered in this module. First, 
{\bf Advice ADV01} models the PIN authentication mechanism. It is applied 
after each step marked with the \texttt{authentication} tag.

\begin{figure}[htb]
 
\begin{tabbing} {\bf Pointcut:} \= {\bf AFTER} AnnotationRef "authentication"\\ \end{tabbing} \begin{tabular}{|p{0.4in}|p{1.2in}|p{1.2in}|p{1.2in}|p{1.2in}|} \hline Id & User Action & Condition & System Response & Annotations \\ \hline adv0101 & - & - & The system asks the customer to enter her personal identification number. & \\ \hline adv0102 & The customer fills in the personal identification number. & - & The system authenticates the customer's personal identification number. & \\ \hline \end{tabular}
 \caption{PIN authentication advice.}
 \label{fig:adv01}
\end{figure}

Therefore, calling \texttt{evalFunction adv01 sc01} leads to the 
scenario shown in Figure~\ref{fig:eval-adv01}. 

\begin{figure}[htb]
 
\begin{tabular}{|p{0.4in}|p{1.2in}|p{1.2in}|p{1.2in}|p{1.2in}|} \hline Id & User Action & Condition & System Response & Annotations \\ \hline sc0101 & The customer selects the withdraw option. & - & The system creates a new withdrawal and asks for the amount to withdraw. & \\ \hline sc0102 & The customer fills in the amount to withdraw. & - & The system retrieves the current balance of the selected account. & \\ \hline sc0103 & - & - & The system verifies that the requested amount is not greater than current balance plus U\$ 5000.00 & authentication \\ \hline adv0101 & - & - & The system asks the customer to enter her personal identification number. & \\ \hline adv0102 & The customer fills in the personal identification number. & - & The system authenticates the customer's personal identification number. & \\ \hline sc0104 & - & - & The bank system withdraws the amount from the account. & transaction \\ \hline sc0105 & - & - & The cash money is provided to the customer. & \\ \hline \end{tabular}
 \caption{The effect of applying \texttt{evaluateAdvice adv01 sc01}.}
 \label{fig:eval-adv01}
\end{figure}

\newpage

Note that the evaluation of the PIN advice, regarding the first scenario, just introduces 
the steps related to the PIN authentication, after any step marked with the ``authentication'' 
tag. Consider now a different advice, which starts a transaction before any step marked with 
the ``transaction'' tag, then it returns to the flow of events of the scenario, and finally commit a transaction. 
In order to model an extension such as that, we design the \emph{Around Advice} depicted in  
Figure~\ref{fig:adv02}.

\begin{figure}[htb]
 
\begin{tabbing} {\bf Pointcut:} \= {\bf AROUND} AnnotationRef "transaction"\\ \end{tabbing} \begin{tabular}{|p{0.4in}|p{1.2in}|p{1.2in}|p{1.2in}|p{1.2in}|} \hline Id & User Action & Condition & System Response & Annotations \\ \hline adv0201 & - & - & The transaction handler starts the processing of a transaction. & \\ \hline \multicolumn{5}{c}{{\bf PROCEED}} \\ \hline adv0202 & - & - & An entry with the transaction information is logged to the overview of the completed transactions of the customers account. & \\ \hline adv0203 & - & - & The transaction is removed from the transaction queue. & \\ \hline \end{tabular}
 \caption{Around Advice for the transaction concern. After the first step, it should \emph{Proceed} with the scenario flow and 
  then perform the remaining two steps (adv0202, adv0203).}
 \label{fig:adv02}
\end{figure}

\newpage

Figure~\ref{fig:eval-adv02} shows the effect of evaluating the Transaction concern advice with respect to the 
withdraw scenario. 

\begin{figure}[htb]
 
\begin{tabular}{|p{0.4in}|p{1.2in}|p{1.2in}|p{1.2in}|p{1.2in}|} \hline Id & User Action & Condition & System Response & Annotations \\ \hline sc0101 & The customer selects the withdraw option. & - & The system creates a new withdrawal and asks for the amount to withdraw. & \\ \hline sc0102 & The customer fills in the amount to withdraw. & - & The system retrieves the current balance of the selected account. & \\ \hline sc0103 & - & - & The system verifies that the requested amount is not greater than current balance plus U\$ 5000.00 & authentication \\ \hline adv0201 & - & - & The transaction handler starts the processing of a transaction. & \\ \hline sc0104 & - & - & The bank system withdraws the amount from the account. & transaction \\ \hline sc0105 & - & - & The cash money is provided to the customer. & \\ \hline adv0202 & - & - & An entry with the transaction information is logged to the overview of the completed transactions of the customers account. & \\ \hline adv0203 & - & - & The transaction is removed from the transaction queue. & \\ \hline \end{tabular}
 \caption{The effect of applying \texttt{evaluateAdvice adv02 sc01}.}
 \label{fig:eval-adv02}
\end{figure}

So, just to understand the effect of the PROCEED construct. Suppose that a specifier forgot to 
introduce the PROCEED clause in the advice of Figure~\ref{fig:adv02}. The result of applying 
this \emph{wrong} asset to the withdraw scenario is depicted in Figure~\ref{fig:eval-adv03}. Notice 
that steps \texttt{sc0104} and \texttt{sc0105} are not present in the resulting scenario. This occurs 
because, without the PROCEED construct, all steps of the scenario that follows a specific joinpoint 
are ignored.  

\begin{figure}[htb]
 
\begin{tabular}{|p{0.4in}|p{1.2in}|p{1.2in}|p{1.2in}|p{1.2in}|} \hline Id & User Action & Condition & System Response & Annotations \\ \hline sc0101 & The customer selects the withdraw option. & - & The system creates a new withdrawal and asks for the amount to withdraw. & \\ \hline sc0102 & The customer fills in the amount to withdraw. & - & The system retrieves the current balance of the selected account. & \\ \hline sc0103 & - & - & The system verifies that the requested amount is not greater than current balance plus U\$ 5000.00 & authentication \\ \hline adv0201 & - & - & The transaction handler starts the processing of a transaction. & \\ \hline adv0202 & - & - & An entry with the transaction information is logged to the overview of the completed transactions of the customers account. & \\ \hline adv0203 & - & - & The transaction is removed from the transaction queue. & \\ \hline \end{tabular}
 \caption{The effect of applying \texttt{evaluateAdvice adv03 sc01}.}
 \label{fig:eval-adv03}
\end{figure}

\newpage 

\section{The input code}
\begin{hscode}\SaveRestoreHook
\column{B}{@{}>{\hspre}l<{\hspost}@{}}%
\column{15}{@{}>{\hspre}l<{\hspost}@{}}%
\column{16}{@{}>{\hspre}l<{\hspost}@{}}%
\column{17}{@{}>{\hspre}l<{\hspost}@{}}%
\column{31}{@{}>{\hspre}l<{\hspost}@{}}%
\column{40}{@{}>{\hspre}l<{\hspost}@{}}%
\column{109}{@{}>{\hspre}l<{\hspost}@{}}%
\column{E}{@{}>{\hspre}l<{\hspost}@{}}%
\>[B]{}\Varid{sc01}\mathbin{::}\Conid{Scenario}{}\<[E]%
\\
\>[B]{}\Varid{sc01}\mathrel{=}\Conid{Scenario}\;\text{\tt \char34 sc01\char34}\;{}\<[E]%
\\
\>[B]{}\hsindent{17}{}\<[17]%
\>[17]{}\text{\tt \char34 This~scenario~allows~a~customer~to~withdraw~money~from~a~previously~selected~account\char34}\;{}\<[E]%
\\
\>[B]{}\hsindent{17}{}\<[17]%
\>[17]{}[\mskip1.5mu \Conid{IdRef}\;\text{\tt \char34 start\char34}\mskip1.5mu]\;{}\<[E]%
\\
\>[B]{}\hsindent{17}{}\<[17]%
\>[17]{}[\mskip1.5mu \Varid{sc0101},\Varid{sc0102},\Varid{sc0103},\Varid{sc0104},\Varid{sc0105}\mskip1.5mu]\;{}\<[E]%
\\
\>[B]{}\hsindent{17}{}\<[17]%
\>[17]{}[\mskip1.5mu \Conid{IdRef}\;\text{\tt \char34 end\char34}\mskip1.5mu]{}\<[E]%
\\[\blanklineskip]%
\>[B]{}\Varid{adv01}\mathbin{::}\Conid{Advice}{}\<[E]%
\\
\>[B]{}\Varid{adv01}\mathrel{=}\Conid{AfterAdvice}\;[\mskip1.5mu \Conid{AnnotationRef}\;\text{\tt \char34 authentication\char34}\mskip1.5mu]\;[\mskip1.5mu \Varid{adv0101},\Varid{adv0102}\mskip1.5mu]{}\<[E]%
\\[\blanklineskip]%
\>[B]{}\Varid{adv02}\mathbin{::}\Conid{Advice}{}\<[E]%
\\
\>[B]{}\Varid{adv02}\mathrel{=}\Conid{AroundAdvice}\;[\mskip1.5mu \Conid{AnnotationRef}\;\text{\tt \char34 transaction\char34}\mskip1.5mu]\;[\mskip1.5mu \Varid{adv0201},\Varid{adv02P},\Varid{adv0202},\Varid{adv0203}\mskip1.5mu]{}\<[E]%
\\[\blanklineskip]%
\>[B]{}\Varid{adv03}\mathbin{::}\Conid{Advice}{}\<[E]%
\\
\>[B]{}\Varid{adv03}\mathrel{=}\Conid{AroundAdvice}\;[\mskip1.5mu \Conid{AnnotationRef}\;\text{\tt \char34 transaction\char34}\mskip1.5mu]\;[\mskip1.5mu \Varid{adv0201},\Varid{adv0202},\Varid{adv0203}\mskip1.5mu]{}\<[E]%
\\[\blanklineskip]%
\>[B]{}\Varid{sc0101}\mathbin{::}\Conid{Step}{}\<[E]%
\\
\>[B]{}\Varid{sc0101}\mathrel{=}\Conid{Step}\;\text{\tt \char34 sc0101\char34}\;{}\<[E]%
\\
\>[B]{}\hsindent{15}{}\<[15]%
\>[15]{}\text{\tt \char34 The~customer~selects~the~withdraw~option.\char34}\;{}\<[E]%
\\
\>[B]{}\hsindent{15}{}\<[15]%
\>[15]{}\text{\tt \char34 -\char34}\;{}\<[E]%
\\
\>[B]{}\hsindent{15}{}\<[15]%
\>[15]{}\text{\tt \char34 The~system~creates~a~new~withdrawal~and~asks~for~the~amount~to~withdraw.\char34}\;{}\<[E]%
\\
\>[B]{}\hsindent{15}{}\<[15]%
\>[15]{}[\mskip1.5mu \mskip1.5mu]{}\<[E]%
\\[\blanklineskip]%
\>[B]{}\Varid{sc0102}\mathbin{::}\Conid{Step}{}\<[E]%
\\
\>[B]{}\Varid{sc0102}\mathrel{=}\Conid{Step}\;\text{\tt \char34 sc0102\char34}\;{}\<[E]%
\\
\>[B]{}\hsindent{15}{}\<[15]%
\>[15]{}\text{\tt \char34 The~customer~fills~in~the~amount~to~withdraw.\char34}\;{}\<[E]%
\\
\>[B]{}\hsindent{15}{}\<[15]%
\>[15]{}\text{\tt \char34 -\char34}\;{}\<[E]%
\\
\>[B]{}\hsindent{15}{}\<[15]%
\>[15]{}\text{\tt \char34 The~system~retrieves~the~current~balance~of~the~selected~account.\char34}\;{}\<[E]%
\\
\>[B]{}\hsindent{15}{}\<[15]%
\>[15]{}[\mskip1.5mu \mskip1.5mu]{}\<[E]%
\\[\blanklineskip]%
\>[B]{}\Varid{sc0103}\mathbin{::}\Conid{Step}{}\<[E]%
\\
\>[B]{}\Varid{sc0103}\mathrel{=}\Conid{Step}\;\text{\tt \char34 sc0103\char34}\;{}\<[E]%
\\
\>[B]{}\hsindent{15}{}\<[15]%
\>[15]{}\text{\tt \char34 -\char34}\;{}\<[E]%
\\
\>[B]{}\hsindent{15}{}\<[15]%
\>[15]{}\text{\tt \char34 -\char34}\;{}\<[E]%
\\
\>[B]{}\hsindent{15}{}\<[15]%
\>[15]{}\text{\tt \char34 The~system~verifies~that~the~requested~amount~is~not~greater~than~current~balance~plus~U\$~5000.00\char34}\;{}\<[E]%
\\
\>[B]{}\hsindent{15}{}\<[15]%
\>[15]{}[\mskip1.5mu \text{\tt \char34 authentication\char34}\mskip1.5mu]{}\<[E]%
\\[\blanklineskip]%
\>[B]{}\Varid{sc0104}\mathbin{::}\Conid{Step}{}\<[E]%
\\
\>[B]{}\Varid{sc0104}\mathrel{=}\Conid{Step}\;\text{\tt \char34 sc0104\char34}\;{}\<[E]%
\\
\>[B]{}\hsindent{15}{}\<[15]%
\>[15]{}\text{\tt \char34 -\char34}\;{}\<[E]%
\\
\>[B]{}\hsindent{15}{}\<[15]%
\>[15]{}\text{\tt \char34 -\char34}\;{}\<[E]%
\\
\>[B]{}\hsindent{15}{}\<[15]%
\>[15]{}\text{\tt \char34 The~bank~system~withdraws~the~amount~from~the~account.\char34}\;{}\<[E]%
\\
\>[B]{}\hsindent{15}{}\<[15]%
\>[15]{}[\mskip1.5mu \text{\tt \char34 transaction\char34}\mskip1.5mu]{}\<[E]%
\\[\blanklineskip]%
\>[B]{}\Varid{sc0105}\mathbin{::}\Conid{Step}{}\<[E]%
\\
\>[B]{}\Varid{sc0105}\mathrel{=}\Conid{Step}\;\text{\tt \char34 sc0105\char34}\;{}\<[E]%
\\
\>[B]{}\hsindent{15}{}\<[15]%
\>[15]{}\text{\tt \char34 -\char34}\;{}\<[E]%
\\
\>[B]{}\hsindent{15}{}\<[15]%
\>[15]{}\text{\tt \char34 -\char34}\;{}\<[E]%
\\
\>[B]{}\hsindent{15}{}\<[15]%
\>[15]{}\text{\tt \char34 The~cash~money~is~provided~to~the~customer.\char34}\;{}\<[E]%
\\
\>[B]{}\hsindent{15}{}\<[15]%
\>[15]{}[\mskip1.5mu \mskip1.5mu]{}\<[E]%
\\[\blanklineskip]%
\>[B]{}\Varid{adv0101}\mathbin{::}\Conid{Step}{}\<[E]%
\\
\>[B]{}\Varid{adv0101}\mathrel{=}\Conid{Step}\;\text{\tt \char34 adv0101\char34}\;{}\<[E]%
\\
\>[B]{}\hsindent{17}{}\<[17]%
\>[17]{}\text{\tt \char34 -\char34}\;{}\<[E]%
\\
\>[B]{}\hsindent{17}{}\<[17]%
\>[17]{}\text{\tt \char34 -~\char34}\;{}\<[E]%
\\
\>[B]{}\hsindent{17}{}\<[17]%
\>[17]{}\text{\tt \char34 The~system~asks~the~customer~to~enter~her~personal~identification~number.\char34}\;{}\<[E]%
\\
\>[B]{}\hsindent{17}{}\<[17]%
\>[17]{}[\mskip1.5mu \mskip1.5mu]{}\<[E]%
\\[\blanklineskip]%
\>[B]{}\Varid{adv0102}\mathbin{::}\Conid{Step}{}\<[E]%
\\
\>[B]{}\Varid{adv0102}\mathrel{=}\Conid{Step}\;\text{\tt \char34 adv0102\char34}\;{}\<[E]%
\\
\>[B]{}\hsindent{17}{}\<[17]%
\>[17]{}\text{\tt \char34 The~customer~fills~in~the~personal~identification~number.\char34}\;{}\<[E]%
\\
\>[B]{}\hsindent{17}{}\<[17]%
\>[17]{}\text{\tt \char34 -~\char34}\;{}\<[E]%
\\
\>[B]{}\hsindent{17}{}\<[17]%
\>[17]{}\text{\tt \char34 The~system~authenticates~the~customer's~personal~identification~number.\char34}\;{}\<[E]%
\\
\>[B]{}\hsindent{17}{}\<[17]%
\>[17]{}[\mskip1.5mu \mskip1.5mu]{}\<[E]%
\\[\blanklineskip]%
\>[B]{}\Varid{adv0201}\mathbin{::}\Conid{Step}{}\<[E]%
\\
\>[B]{}\Varid{adv0201}\mathrel{=}\Conid{Step}\;\text{\tt \char34 adv0201\char34}\;{}\<[E]%
\\
\>[B]{}\hsindent{16}{}\<[16]%
\>[16]{}\text{\tt \char34 -\char34}\;{}\<[E]%
\\
\>[B]{}\hsindent{16}{}\<[16]%
\>[16]{}\text{\tt \char34 -\char34}\;{}\<[E]%
\\
\>[B]{}\hsindent{16}{}\<[16]%
\>[16]{}\text{\tt \char34 The~transaction~handler~starts~the~processing~of~a~transaction.\char34}\;{}\<[E]%
\\
\>[B]{}\hsindent{16}{}\<[16]%
\>[16]{}[\mskip1.5mu \mskip1.5mu]{}\<[E]%
\\[\blanklineskip]%
\>[B]{}\Varid{adv02P}\mathbin{::}\Conid{Step}{}\<[E]%
\\
\>[B]{}\Varid{adv02P}\mathrel{=}\Conid{Proceed}{}\<[E]%
\\[\blanklineskip]%
\>[B]{}\Varid{adv0202}\mathbin{::}\Conid{Step}{}\<[E]%
\\
\>[B]{}\Varid{adv0202}\mathrel{=}\Conid{Step}\;\text{\tt \char34 adv0202\char34}\;{}\<[E]%
\\
\>[B]{}\hsindent{16}{}\<[16]%
\>[16]{}\text{\tt \char34 -\char34}\;{}\<[E]%
\\
\>[B]{}\hsindent{16}{}\<[16]%
\>[16]{}\text{\tt \char34 -\char34}\;{}\<[E]%
\\
\>[B]{}\hsindent{16}{}\<[16]%
\>[16]{}\text{\tt \char34 An~entry~with~the~transaction~information~is~logged~to~the~overview~of~the~completed~transactions~of~the~customers~account.\char34}\;{}\<[E]%
\\
\>[B]{}\hsindent{16}{}\<[16]%
\>[16]{}[\mskip1.5mu \mskip1.5mu]{}\<[E]%
\\
\>[B]{}\Varid{adv0203}\mathbin{::}\Conid{Step}{}\<[E]%
\\
\>[B]{}\Varid{adv0203}\mathrel{=}\Conid{Step}\;\text{\tt \char34 adv0203\char34}\;{}\<[E]%
\\
\>[B]{}\hsindent{17}{}\<[17]%
\>[17]{}\text{\tt \char34 -\char34}\;{}\<[E]%
\\
\>[B]{}\hsindent{17}{}\<[17]%
\>[17]{}\text{\tt \char34 -\char34}\;{}\<[E]%
\\
\>[B]{}\hsindent{17}{}\<[17]%
\>[17]{}\text{\tt \char34 The~transaction~is~removed~from~the~transaction~queue.\char34}\;{}\<[E]%
\\
\>[B]{}\hsindent{17}{}\<[17]%
\>[17]{}[\mskip1.5mu \mskip1.5mu]{}\<[E]%
\\[\blanklineskip]%
\>[B]{}\Varid{test1}\mathrel{=}\Conid{TestCase}\;(\Varid{assertEqual}\;\text{\tt \char34 expected~ids~before~weaving\char34}\;[\mskip1.5mu \text{\tt \char34 sc0101\char34},\text{\tt \char34 sc0102\char34},\text{\tt \char34 sc0103\char34},\text{\tt \char34 sc0104\char34},\text{\tt \char34 sc0105\char34}\mskip1.5mu]\;{}\<[109]%
\>[109]{}[\mskip1.5mu \Varid{sId}\;\Varid{s}\mid \Varid{s}\leftarrow \Varid{steps}\;\Varid{sc01}\mskip1.5mu]){}\<[E]%
\\[\blanklineskip]%
\>[B]{}\Varid{test2}\mathrel{=}\Conid{TestCase}\;(\Varid{assertEqual}\;\text{\tt \char34 expected~ids~after~weaving\char34}\;{}\<[E]%
\\
\>[B]{}\hsindent{31}{}\<[31]%
\>[31]{}[\mskip1.5mu \text{\tt \char34 sc0101\char34}{}\<[E]%
\\
\>[B]{}\hsindent{31}{}\<[31]%
\>[31]{},\text{\tt \char34 sc0102\char34}{}\<[E]%
\\
\>[B]{}\hsindent{31}{}\<[31]%
\>[31]{},\text{\tt \char34 sc0103\char34}{}\<[E]%
\\
\>[B]{}\hsindent{31}{}\<[31]%
\>[31]{},\text{\tt \char34 adv0101\char34}{}\<[E]%
\\
\>[B]{}\hsindent{31}{}\<[31]%
\>[31]{},\text{\tt \char34 adv0102\char34}{}\<[E]%
\\
\>[B]{}\hsindent{31}{}\<[31]%
\>[31]{},\text{\tt \char34 sc0104\char34}{}\<[E]%
\\
\>[B]{}\hsindent{31}{}\<[31]%
\>[31]{},\text{\tt \char34 sc0105\char34}\mskip1.5mu]\;[\mskip1.5mu \Varid{sId}\;\Varid{s}\mid \Varid{s}\leftarrow \Varid{steps}\;(\Varid{evaluateAdvice}\;\Varid{adv01}\;\Varid{sc01})\mskip1.5mu]){}\<[E]%
\\[\blanklineskip]%
\>[B]{}\Varid{tests}\mathrel{=}\Conid{TestList}\;[\mskip1.5mu \Conid{TestLabel}\;\text{\tt \char34 test1\char34}\;\Varid{test1},\Conid{TestLabel}\;\text{\tt \char34 test2\char34}\;\Varid{test2}\mskip1.5mu]{}\<[E]%
\\[\blanklineskip]%
\>[B]{}\Varid{formatString}\mathbin{::}\Conid{String}\to \Conid{String}{}\<[E]%
\\
\>[B]{}\Varid{formatString}\mathrel{=}\Varid{removeCifrao}\mathbin{\circ}\Varid{removeCR}\mathbin{\circ}\Varid{removeBLCommand}{}\<[E]%
\\[\blanklineskip]%
\>[B]{}\Varid{removeCifrao}\mathbin{::}\Conid{String}\to \Conid{String}{}\<[E]%
\\
\>[B]{}\Varid{removeCifrao}\;\Varid{s}\mathrel{=}{}\<[E]%
\\
\>[B]{}\mathbf{let}\;\Varid{st}\mathrel{=}\Varid{words}\;\Varid{s}{}\<[E]%
\\
\>[B]{}\mathbf{in}\;\Varid{unwords}\;[\mskip1.5mu \mathbf{if}\;\Varid{c}\equiv \text{\tt \char34 U\$\char34}\;\mathbf{then}\;\text{\tt \char34 U\char92 \char92 \$\char34}\;{}\<[40]%
\>[40]{}\mathbf{else}\;\Varid{c}\mid \Varid{c}\leftarrow \Varid{st}\mskip1.5mu]{}\<[E]%
\\[\blanklineskip]%
\>[B]{}\Varid{removeCR}\mathbin{::}\Conid{String}\to \Conid{String}{}\<[E]%
\\
\>[B]{}\Varid{removeCR}\;\Varid{s}\mathrel{=}{}\<[15]%
\>[15]{}[\mskip1.5mu \mathbf{if}\;\Varid{c}\equiv \text{\tt '\char92 n'}\;\mathbf{then}\;\text{\tt '~'}\;\mathbf{else}\;\Varid{c}\mid \Varid{c}\leftarrow \Varid{s}\mskip1.5mu]{}\<[E]%
\\[\blanklineskip]%
\>[B]{}\Varid{removeBLCommand}\mathbin{::}\Conid{String}\to \Conid{String}{}\<[E]%
\\
\>[B]{}\Varid{removeBLCommand}\;\Varid{s}\mathrel{=}\Varid{replaceString}\;\text{\tt \char34 \char92 \char92 bl\char34}\;\text{\tt \char34 \char92 \char92 \char92 \char92 ~\char92 \char92 hline\char34}\;\Varid{s}{}\<[E]%
\ColumnHook
\end{hscode}\resethooks




\end{document}
