%This Latex file is machine-generated by the Hephaestus

\documentclass[a4paper,11pt]{article}
\newcommand{\bl}{\\ \hline}
\title{Use Case Model Generated from TaRGeT}
\begin{document}
\maketitle
\section*{Use cases}
\subsection*{Use case UC01}
\begin{itemize}
\item {\bf Name: }Opening the TaRGeT
\item {\bf Description: }It describes how the TaRGeT should behave when it is
				started up.
\end{itemize}
\subsubsection*{Scenario SC01}
\begin{tabular}{p{1in}p{4in}}
{\bf Description:} & Startup the TaRGeT with .NET and Java runtime
					environment installed \\
{\bf From steps:} & START \\
{\bf To steps:} & END \\
\end{tabular}
 
\begin{tabular}{|p{0.8in}|p{1.6in}|p{1.6in}|p{1.6in}|}
\hline
Id & User Action & Condition & System Response  \bl 
1M & 
						Start the TaRGeT.
				 & The .NET and Java runtime environment are installed in
						the machine (see requirements for further information about Java
						and .NET versions). & 
						A splash screen is displayed. Verify the requirements
						document to
						check the splash screen.
				 \bl 
2M & 
						Wait some seconds.
				 & - & 
						The TaRGeT is started. No TaRGeT project is opened. A
						background image
						is displayed (see requirements document for more
						details).
					 \bl 
\end{tabular}
\subsubsection*{Scenario SC02}
\begin{tabular}{p{1in}p{4in}}
{\bf Description:} & Start up the TaRGeT when .NET runtime environment is
					not installed in the machine. \\
{\bf From steps:} & START \\
{\bf To steps:} & END \\
\end{tabular}
 
\begin{tabular}{|p{0.8in}|p{1.6in}|p{1.6in}|p{1.6in}|}
\hline
Id & User Action & Condition & System Response  \bl 
1A & Try to start the TaRGeT. & Java runtime environment is installed in the machine.
						The .NET runtime environment is not installed in the machine (see
						requirements for further information about Java and .NET
						versions). [ER_TARGET_0020, ER_TARGET_0025] & A message is displayed informing that an error has
						occurred and asking the user to see the log file. [FR_TARGET_0230]
					 \bl 
\end{tabular}
\subsubsection*{Scenario SC03}
\begin{tabular}{p{1in}p{4in}}
{\bf Description:} & Start up the TaRGeT when Java runtime environment is
					not installed in the machine. \\
{\bf From steps:} & START \\
{\bf To steps:} & END \\
\end{tabular}
 
\begin{tabular}{|p{0.8in}|p{1.6in}|p{1.6in}|p{1.6in}|}
\hline
Id & User Action & Condition & System Response  \bl 
1B & Try to start the TaRGeT. & The Java runtime environment is not installed in the
						machine (see requirements for further information about Java
						version). [ER_TARGET_0020] & A message is displayed informing that no JRE is
						installed. The TaRGeT is not started up. [FR_TARGET_0230]
					 \bl 
\end{tabular}
\subsection*{Use case UC05}
\begin{itemize}
\item {\bf Name: }Creating a Project from Blank Work Area
\item {\bf Description: }Create a TaRGeT project.
\end{itemize}
\subsubsection*{Scenario SC04}
\begin{tabular}{p{1in}p{4in}}
{\bf Description:} & Create a new project. \\
{\bf From steps:} & START \\
{\bf To steps:} & END \\
\end{tabular}
 
\begin{tabular}{|p{0.8in}|p{1.6in}|p{1.6in}|p{1.6in}|}
\hline
Id & User Action & Condition & System Response  \bl 
1M & Go to File option in the menu bar. & The TaRGeT is already started up. No project is opened.
					 & The New Project option is enabled in the drop down
						menu.
						The Close Project option is disabled. \bl 
2M & Select the New Project option. [FR_TARGET_0020] & - & The New Project screen is displayed. The default
						project
						name is displayed in the Project Name field. The default
						project
						location is displayed in the Destination Folder field.
					 \bl 
3M & Click on Finish button. & - & A progress bar is displayed. A new project work area is
						displayed. The project folder is created in the specified
						location. [FR_TARGET_0110, FR_TARGET_0010] \bl 
\end{tabular}
\subsubsection*{Scenario SC05}
\begin{tabular}{p{1in}p{4in}}
{\bf Description:} & Type a valid project name. \\
{\bf From steps:} & 2M \\
{\bf To steps:} & 3M \\
\end{tabular}
 
\begin{tabular}{|p{0.8in}|p{1.6in}|p{1.6in}|p{1.6in}|}
\hline
Id & User Action & Condition & System Response  \bl 
1A & Type a valid project name. & - & The "Project Name" field is filled. [FR_TARGET_0025]
					 \bl 
\end{tabular}
\subsubsection*{Scenario SC06}
\begin{tabular}{p{1in}p{4in}}
{\bf Description:} & Cancel the project creation. \\
{\bf From steps:} & 2M,1A,1C \\
{\bf To steps:} & END \\
\end{tabular}
 
\begin{tabular}{|p{0.8in}|p{1.6in}|p{1.6in}|p{1.6in}|}
\hline
Id & User Action & Condition & System Response  \bl 
1B & Click on the "Cancel" button. & - & The tool goes back to the window it was before the
						project creation. \bl 
\end{tabular}
\subsubsection*{Scenario SC07}
\begin{tabular}{p{1in}p{4in}}
{\bf Description:} & Browse destination folder. \\
{\bf From steps:} & 2M,1A \\
{\bf To steps:} & 3M \\
\end{tabular}
 
\begin{tabular}{|p{0.8in}|p{1.6in}|p{1.6in}|p{1.6in}|}
\hline
Id & User Action & Condition & System Response  \bl 
1C & Browse and select the location in which the project shall
						be created. & - & The "Destination Folder" field is filled. \bl 
\end{tabular}
\subsubsection*{Scenario SC08}
\begin{tabular}{p{1in}p{4in}}
{\bf Description:} & Create the project using shortcut. \\
{\bf From steps:} & START \\
{\bf To steps:} & END \\
\end{tabular}
 
\begin{tabular}{|p{0.8in}|p{1.6in}|p{1.6in}|p{1.6in}|}
\hline
Id & User Action & Condition & System Response  \bl 
1H & Press CTRL+N. & - & The "New Project" screen is displayed. The default
						project name is displayed in the "Project Name" field. The default
						project location is displayed in the "Destination Folder" field.
						[FR_TARGET_0020] \bl 
2H & Click on "Finish" button. & - & A progress bar is displayed. A new project work area is
						displayed. The project folder is created in the specified
						location. [FR_TARGET_0110, FR_TARGET_0010] \bl 
\end{tabular}
\subsubsection*{Scenario SC09}
\begin{tabular}{p{1in}p{4in}}
{\bf Description:} & Type an invalid project name. \\
{\bf From steps:} & 2M \\
{\bf To steps:} & END \\
\end{tabular}
 
\begin{tabular}{|p{0.8in}|p{1.6in}|p{1.6in}|p{1.6in}|}
\hline
Id & User Action & Condition & System Response  \bl 
1D & Type an invalid project name (e.g. with one of the
						following characters ":\/<>?*). & - & A message is displayed indicating that some characters
						are not allowed. The "Finish" button is disabled. [FR_TARGET_0025]
					 \bl 
\end{tabular}
\subsubsection*{Scenario SC10}
\begin{tabular}{p{1in}p{4in}}
{\bf Description:} & No project name is typed. \\
{\bf From steps:} & 2M \\
{\bf To steps:} & END \\
\end{tabular}
 
\begin{tabular}{|p{0.8in}|p{1.6in}|p{1.6in}|p{1.6in}|}
\hline
Id & User Action & Condition & System Response  \bl 
1E & Type a blank project name. [FR_TARGET_0025] & - & A message is displayed indicating that a name must be
						specified. The "Finish" button is disabled.  \bl 
\end{tabular}
\subsubsection*{Scenario SC11}
\begin{tabular}{p{1in}p{4in}}
{\bf Description:} &  Project directory already exists. \\
{\bf From steps:} & 2M \\
{\bf To steps:} & END \\
\end{tabular}
 
\begin{tabular}{|p{0.8in}|p{1.6in}|p{1.6in}|p{1.6in}|}
\hline
Id & User Action & Condition & System Response  \bl 
1F & Type a valid project name. [FR_TARGET_0025] & - & The "Project Name" field is filled.  \bl 
2F & Browse and select the location in which there is a
						directory with the same name of the project. & - & The "Destination Folder" field is filled.  \bl 
3F & Click on "Finish" button. & - & A dialog is displayed indicating that the project folder
						already exists. The dialog asks if the user you want to proceed
						and erase all content of the folder. [FR_TARGET_0020] \bl 
4F & Click on "Yes" button. & - & All the folder contents is deleted and the new project is
						created.  \bl 
\end{tabular}
\subsubsection*{Scenario SC12}
\begin{tabular}{p{1in}p{4in}}
{\bf Description:} & Do not delete all folder contents. \\
{\bf From steps:} & 3F \\
{\bf To steps:} & END \\
\end{tabular}
 
\begin{tabular}{|p{0.8in}|p{1.6in}|p{1.6in}|p{1.6in}|}
\hline
Id & User Action & Condition & System Response  \bl 
1G & Click on "No" button. & - & The focus goes back to the "New Project" window.
					 \bl 
\end{tabular}
\subsection*{Use case UC_10}
\begin{itemize}
\item {\bf Name: }Creating a Project from an Already Opened Project
\item {\bf Description: }A project is already opened.
\end{itemize}
\subsubsection*{Scenario SC13}
\begin{tabular}{p{1in}p{4in}}
{\bf Description:} & A project is already opened. \\
{\bf From steps:} & START \\
{\bf To steps:} & UC_05#3M,UC_05#1A,UC_05#1B,UC_05#1C,UC_05#1D,UC_05#1E,
					UC_05#1F \\
\end{tabular}
 
\begin{tabular}{|p{0.8in}|p{1.6in}|p{1.6in}|p{1.6in}|}
\hline
Id & User Action & Condition & System Response  \bl 
1M & Go to "File" option in the menu bar.  & The TaRGeT is already started up. A project is already
						opened. & The "New Project" option is enabled in the drop down
						menu.  \bl 
2M & Select the "New Project" option. [FR_TARGET_0020]  & - & A dialog box is displayed informing that the current
						project will be closed in order to create a new one.  \bl 
3M & Click on "Yes" button. & - & The current project is closed. The "New Project" screen
						is displayed. The default project name is displayed in the
						"Project Name" field. The default project location is displayed in
						the "Destination Folder" field.  \bl 
\end{tabular}
\subsubsection*{Scenario SC14}
\begin{tabular}{p{1in}p{4in}}
{\bf Description:} & Cancel the project creation. \\
{\bf From steps:} & 2M \\
{\bf To steps:} & END \\
\end{tabular}
 
\begin{tabular}{|p{0.8in}|p{1.6in}|p{1.6in}|p{1.6in}|}
\hline
Id & User Action & Condition & System Response  \bl 
1A & Click on "No" button. & - & The popup is closed and the tool goes back to the main
						window with the previous project opened. \bl 
\end{tabular}
\subsection*{Use case UC_15}
\begin{itemize}
\item {\bf Name: }Closing a Project
\item {\bf Description: }Close a project.
\end{itemize}
\subsubsection*{Scenario SC15}
\begin{tabular}{p{1in}p{4in}}
{\bf Description:} & Close a project. \\
{\bf From steps:} & START \\
{\bf To steps:} & END \\
\end{tabular}
 
\begin{tabular}{|p{0.8in}|p{1.6in}|p{1.6in}|p{1.6in}|}
\hline
Id & User Action & Condition & System Response  \bl 
1M & Go to "File" option in the menu bar.   & TaRGeT is started up and a project is already opened.
					 & The "Close Project" option is enabled.  \bl 
2M & Select the "Close Project" option. [FR_TARGET_0035]
					 & - & A message is displayed asking if the user really wants to
						close the project.  \bl 
3M & Click on "Yes" button. & - & The project is closed. The blank project area is
						displayed. The background image is shown. [FR_TARGET_0225]
					 \bl 
\end{tabular}
\subsubsection*{Scenario SC16}
\begin{tabular}{p{1in}p{4in}}
{\bf Description:} & Cancel the project closing. \\
{\bf From steps:} & 2M \\
{\bf To steps:} & END \\
\end{tabular}
 
\begin{tabular}{|p{0.8in}|p{1.6in}|p{1.6in}|p{1.6in}|}
\hline
Id & User Action & Condition & System Response  \bl 
1A & Click on "No" button. & - & The popup is closed and the tool goes back to the main
						window. \bl 
\end{tabular}
\subsubsection*{Scenario SC17}
\begin{tabular}{p{1in}p{4in}}
{\bf Description:} & Close a project using shortcut. \\
{\bf From steps:} & START \\
{\bf To steps:} & 3M \\
\end{tabular}
 
\begin{tabular}{|p{0.8in}|p{1.6in}|p{1.6in}|p{1.6in}|}
\hline
Id & User Action & Condition & System Response  \bl 
1B & Press CTRL+W. & - & A message is displayed asking if the user really wants to
						close the project. [FR_TARGET_0035] \bl 
\end{tabular}
\subsection*{Use case UC_20}
\begin{itemize}
\item {\bf Name: }Opening an Existing Project from Blank Work Area
\item {\bf Description: }Open an existing project.
\end{itemize}
\subsubsection*{Scenario SC18}
\begin{tabular}{p{1in}p{4in}}
{\bf Description:} & Open an existing project. \\
{\bf From steps:} & START \\
{\bf To steps:} & END \\
\end{tabular}
 
\begin{tabular}{|p{0.8in}|p{1.6in}|p{1.6in}|p{1.6in}|}
\hline
Id & User Action & Condition & System Response  \bl 
1M & Go to "File" option in the menu bar.   & TaRGeT is already started up. No project is opened.
					 & The "Open Project" option is enabled.  \bl 
2M & Select the "Open Project" option. [FR_TARGET_0040] & - & The "Open Project" window is displayed. No project is
						selected. The "Finish" button is disabled. \bl 
3M & Browse and select an existing project. & - & The "Project Name" field is now filled. The "Finish"
						button is enabled. \bl 
4M & Click on "Finish" button. & - & A progress bar is displayed indicating that the selected
						existing project is being opened. After, the existing project is
						opened. The project work area is displayed. [FR_TARGET_0110]
					 \bl 
\end{tabular}
\subsubsection*{Scenario SC19}
\begin{tabular}{p{1in}p{4in}}
{\bf Description:} & Cancel the project opening. \\
{\bf From steps:} & 2M,3M \\
{\bf To steps:} & END \\
\end{tabular}
 
\begin{tabular}{|p{0.8in}|p{1.6in}|p{1.6in}|p{1.6in}|}
\hline
Id & User Action & Condition & System Response  \bl 
1A & Click on "Cancel" button. & - & The tool goes to the blank work area. No project is
						opened. [FR_TARGET_0225] \bl 
\end{tabular}
\subsubsection*{Scenario SC20}
\begin{tabular}{p{1in}p{4in}}
{\bf Description:} & Open using shortcut. \\
{\bf From steps:} & START \\
{\bf To steps:} & 3M \\
\end{tabular}
 
\begin{tabular}{|p{0.8in}|p{1.6in}|p{1.6in}|p{1.6in}|}
\hline
Id & User Action & Condition & System Response  \bl 
1B & Press CTRL+O. & - & The "Open Project" window is displayed. No project is
						selected. The "Finish" button is disabled. [FR_TARGET_0040]
					 \bl 
\end{tabular}
\subsection*{Use case UC_25}
\begin{itemize}
\item {\bf Name: }Opening an Existing Project from an Already Opened Project
			
\item {\bf Description: }A project is already opened.
\end{itemize}
\subsubsection*{Scenario SC21}
\begin{tabular}{p{1in}p{4in}}
{\bf Description:} & Open an existing project from an already opened
					project. \\
{\bf From steps:} & START \\
{\bf To steps:} & UC_20#3M \\
\end{tabular}
 
\begin{tabular}{|p{0.8in}|p{1.6in}|p{1.6in}|p{1.6in}|}
\hline
Id & User Action & Condition & System Response  \bl 
1M & Go to "File" option in the menu bar.   & TaRGeT is already started up. A project is already
						opened. &  The "Open Project" option is enabled.  \bl 
2M & Select the "Open Project" option. [FR_TARGET_0040]  & - & A dialog box is displayed informing that the current
						project will be closed in order to open another one.  \bl 
3M & Click on "Yes" button. & - & The "Open Project" screen is displayed.  \bl 
\end{tabular}
\subsubsection*{Scenario SC22}
\begin{tabular}{p{1in}p{4in}}
{\bf Description:} & Cancel the project opening. \\
{\bf From steps:} & 2M \\
{\bf To steps:} & END \\
\end{tabular}
 
\begin{tabular}{|p{0.8in}|p{1.6in}|p{1.6in}|p{1.6in}|}
\hline
Id & User Action & Condition & System Response  \bl 
1A & Click on "No" button. & - & The dialog box is closed and the current project work
						area is displayed. \bl 
\end{tabular}
\subsection*{Use case UC_26}
\begin{itemize}
\item {\bf Name: }Refreshing a Project
\item {\bf Description: }Refresh a project.
\end{itemize}
\subsubsection*{Scenario SC23}
\begin{tabular}{p{1in}p{4in}}
{\bf Description:} & Refresh a project. \\
{\bf From steps:} & START \\
{\bf To steps:} & UC_90#1M \\
\end{tabular}
 
\begin{tabular}{|p{0.8in}|p{1.6in}|p{1.6in}|p{1.6in}|}
\hline
Id & User Action & Condition & System Response  \bl 
1M & Go to "Project" option in the menu bar.   & - &  A menu with two options shall appear: "Refresh
						Automatically" and "Refresh Project". "Refresh Automatically"
						option is checked. [FR_TARGET_0015] \bl 
\end{tabular}
\subsubsection*{Scenario SC24}
\begin{tabular}{p{1in}p{4in}}
{\bf Description:} & "Refresh automatically" option disabled. \\
{\bf From steps:} & 1M \\
{\bf To steps:} & END \\
\end{tabular}
 
\begin{tabular}{|p{0.8in}|p{1.6in}|p{1.6in}|p{1.6in}|}
\hline
Id & User Action & Condition & System Response  \bl 
1A & Click in "Refresh Automatically" option. & TaRGeT is already started up. A project is already
						opened.  & "Refresh Automatically" option is not checked. \bl 
2A & Open an already imported Use Case Document with the
						Microsoft Word (outside the TaRGeT). & There is at least one imported Use Case Document.
						[FR_TARGET_0100]  & The document is opened in the default viewer outside the
						TaRGeT. \bl 
3A & Modify any field in Use Case Document and save the
						document. & - & The document is modified.  \bl 
4A & Go back to TaRGeT work area. & - & The project is not refreshed. \bl 
5A & Go to "Project" in the menu bar and select "Refresh
						project" option. & - & A progress bar is displayed while the project is being
						refreshed. [FR_TARGET_0110, FR_TARGET_0015] \bl 
\end{tabular}
\subsubsection*{Scenario SC25}
\begin{tabular}{p{1in}p{4in}}
{\bf Description:} & "Refresh automatically" option enabled. \\
{\bf From steps:} & 1M \\
{\bf To steps:} & END \\
\end{tabular}
 
\begin{tabular}{|p{0.8in}|p{1.6in}|p{1.6in}|p{1.6in}|}
\hline
Id & User Action & Condition & System Response  \bl 
1B & Go to "Project" in the menu bar and select "Refresh
						project" option. & - & The project is refreshed. [FR_TARGET_0015] \bl 
\end{tabular}
\subsubsection*{Scenario SC26}
\begin{tabular}{p{1in}p{4in}}
{\bf Description:} & Using the Refresh shortcut. \\
{\bf From steps:} & 3A \\
{\bf To steps:} & END \\
\end{tabular}
 
\begin{tabular}{|p{0.8in}|p{1.6in}|p{1.6in}|p{1.6in}|}
\hline
Id & User Action & Condition & System Response  \bl 
1C & Press CTRL+R. & - & A progress bar is displayed while the project is being
						refreshed. [FR_TARGET_0110, FR_TARGET_0015] \bl 
\end{tabular}
\subsection*{Use case UC_30}
\begin{itemize}
\item {\bf Name: }Importing Valid Use Case Documents
\item {\bf Description: } Import Use Case Documents.
\end{itemize}
\subsubsection*{Scenario SC27}
\begin{tabular}{p{1in}p{4in}}
{\bf Description:} & Import Use Case Documents. \\
{\bf From steps:} & START \\
{\bf To steps:} & END \\
\end{tabular}
 
\begin{tabular}{|p{0.8in}|p{1.6in}|p{1.6in}|p{1.6in}|}
\hline
Id & User Action & Condition & System Response  \bl 
1M & Go to "Artifacts" option in the menu bar. & TaRGeT is started up. A project is already opened.
					 & The "Import Use Case Documents" option is enabled.
					 \bl 
2M & Select "Import Use Case Documents" option. [FR_TARGET_0100]
					 & - & The "Import Documents" window is displayed. The "Finish"
						button is disabled.  \bl 
3M & Select "Add Document" button and browse a valid Use Case
						Document.  & No document is imported. & The selected document path is inserted in the "Documents
						To Import" list. The "Finish" button is enabled. \bl 
4M & Click on "Finish" button. & - & A progress bar is displayed indicating that the
						document(s) is (are) being processed and imported.[FR_TARGET_0110]
					 \bl 
5M & Wait the progress bar.  & - & The imported document(s) is (are) placed in a specific
						folder and displayed in the "Artifacts" View. The imported use
						cases are displayed in the "Use Cases" View. [FR_TARGET_0010,
						FR_TARGET_0015] \bl 
\end{tabular}
\subsubsection*{Scenario SC28}
\begin{tabular}{p{1in}p{4in}}
{\bf Description:} & Import another document. \\
{\bf From steps:} & 3M \\
{\bf To steps:} & 4M \\
\end{tabular}
 
\begin{tabular}{|p{0.8in}|p{1.6in}|p{1.6in}|p{1.6in}|}
\hline
Id & User Action & Condition & System Response  \bl 
1A & Select "Add Document" button and browse another valid Use
						Case Document. The document must not have duplicated feature ids,
						when compared with the already browsed document.  & Another document has to be inserted on the "Documents To
						Import" list. & The selected document path is inserted in the "Documents
						To Import" list. \bl 
\end{tabular}
\subsubsection*{Scenario SC29}
\begin{tabular}{p{1in}p{4in}}
{\bf Description:} & Cancel the importing. \\
{\bf From steps:} & 2M,3M \\
{\bf To steps:} & END \\
\end{tabular}
 
\begin{tabular}{|p{0.8in}|p{1.6in}|p{1.6in}|p{1.6in}|}
\hline
Id & User Action & Condition & System Response  \bl 
1B & Click on "Cancel" button. & - & The focus goes back to the project work area. \bl 
\end{tabular}
\subsubsection*{Scenario SC30}
\begin{tabular}{p{1in}p{4in}}
{\bf Description:} & Import through shortcut. \\
{\bf From steps:} & START \\
{\bf To steps:} & 3M \\
\end{tabular}
 
\begin{tabular}{|p{0.8in}|p{1.6in}|p{1.6in}|p{1.6in}|}
\hline
Id & User Action & Condition & System Response  \bl 
1C & Press CTRL+I. & - & The "Import Documents" window is displayed. The "Finish"
						button is disabled. \bl 
\end{tabular}
\subsection*{Use case UC_35}
\begin{itemize}
\item {\bf Name: }Importing a Use Case Document With Duplicated Feature ID
\item {\bf Description: }Use Case Document has duplicated feature ID.
			
\end{itemize}
\subsubsection*{Scenario SC31}
\begin{tabular}{p{1in}p{4in}}
{\bf Description:} & Imported Use Case Document has duplicated feature ID.
				 \\
{\bf From steps:} & UC_30#2M \\
{\bf To steps:} & END \\
\end{tabular}
 
\begin{tabular}{|p{0.8in}|p{1.6in}|p{1.6in}|p{1.6in}|}
\hline
Id & User Action & Condition & System Response  \bl 
1M & Select "Add Document" button and browse a Use Case Document
						with duplicated feature ID. This means that the document contains
						at least two features with the same ID.  & There is no imported document. & The selected document path is inserted in the "Documents
						To Import" list. The "Finish" button is enabled. \bl 
2M & Click on "Finish" button. & - & A progress bar is displayed indicating that the document
						is being processed. [FR_TARGET_0110]  \bl 
3M & Wait the progress bar.  & - & An error dialog is displayed indicating the document that
						has duplicated feature id [FR_TARGET_0125] \bl 
4M & Click on "OK" button. & - & The selected document(s) will not be imported. The focus
						goes back to "Import Documents" window. \bl 
\end{tabular}
\subsection*{Use case UC_40}
\begin{itemize}
\item {\bf Name: }Importing a Use Case Document With Duplicated Use Case ID
			
\item {\bf Description: }Use Case Document has duplicated use case ID.
			
\end{itemize}
\subsubsection*{Scenario SC32}
\begin{tabular}{p{1in}p{4in}}
{\bf Description:} & Imported Use Case Document has duplicated use case ID.
				 \\
{\bf From steps:} & UC_30#2M \\
{\bf To steps:} & END \\
\end{tabular}
 
\begin{tabular}{|p{0.8in}|p{1.6in}|p{1.6in}|p{1.6in}|}
\hline
Id & User Action & Condition & System Response  \bl 
1M & Select "Add Document" button and browse a Use Case Document
						with duplicated use case ID. This means that the document contains
						a feature that has at least two use cases with the same ID.
					 & No document is imported. & The selected document path is inserted in the "Documents
						To Import" list. The "Finish" button is enabled. \bl 
2M & Click on "Finish" button. & - & A progress bar is displayed indicating that the document
						is being processed. [FR_TARGET_0110] \bl 
3M & Wait the progress bar.  & - & An error dialog is displayed indicating the document that
						has duplicated use case ID. [FR_TARGET_0125]  \bl 
4M & Click on "OK" button. & - & The selected document(s) will not be imported. The focus
						goes back to "Import Documents" window. \bl 
\end{tabular}
\subsection*{Use case UC_45}
\begin{itemize}
\item {\bf Name: }Importing a Use Case Document With Invalid Structure
\item {\bf Description: }Use Case Document has invalid structure.
\end{itemize}
\subsubsection*{Scenario SC33}
\begin{tabular}{p{1in}p{4in}}
{\bf Description:} & Imported Use Case Document has an invalid structure.
				 \\
{\bf From steps:} & UC_30#2M \\
{\bf To steps:} & END \\
\end{tabular}
 
\begin{tabular}{|p{0.8in}|p{1.6in}|p{1.6in}|p{1.6in}|}
\hline
Id & User Action & Condition & System Response  \bl 
1M & Select "Add Document" button and browse a Use Case Document
						with invalid structure. & No document is imported. & The selected document path is inserted in the "Documents
						To Import" list. The "Finish" button is enabled. \bl 
2M & Click on "Finish" button. & - & A progress bar is displayed indicating that the document
						is being processed. [FR_TARGET_0110]  \bl 
3M & Wait the progress bar. & - & An error dialog is displayed indicating the document that
						has an error. [FR_TARGET_0125]  \bl 
4M & Click on "OK" button. & - & The selected document(s) will not be imported. The focus
						goes back to "Import Documents" window. \bl 
\end{tabular}
\subsubsection*{Scenario SC34}
\begin{tabular}{p{1in}p{4in}}
{\bf Description:} & Import another document. \\
{\bf From steps:} & 1M \\
{\bf To steps:} & 2M \\
\end{tabular}
 
\begin{tabular}{|p{0.8in}|p{1.6in}|p{1.6in}|p{1.6in}|}
\hline
Id & User Action & Condition & System Response  \bl 
1A & Click on "Add Document" button and browse another valid Use
						Case Document. & Another document has to be imported. & The selected document path is inserted in the "Documents
						To Import" list. \bl 
\end{tabular}
\subsubsection*{Scenario SC35}
\begin{tabular}{p{1in}p{4in}}
{\bf Description:} & Cancel the importing. \\
{\bf From steps:} & 4M \\
{\bf To steps:} & END \\
\end{tabular}
 
\begin{tabular}{|p{0.8in}|p{1.6in}|p{1.6in}|p{1.6in}|}
\hline
Id & User Action & Condition & System Response  \bl 
1B & Click on "Cancel" button. & - & The focus goes back to the project work area. \bl 
\end{tabular}
\subsection*{Use case UC_50}
\begin{itemize}
\item {\bf Name: }Importing a Use Case Document with an Empty Mandatory Field
			
\item {\bf Description: }Use Case Document has an empty mandatory field.
			
\end{itemize}
\subsubsection*{Scenario SC36}
\begin{tabular}{p{1in}p{4in}}
{\bf Description:} & Selected Use Case Document has an empty mandatory
					field. \\
{\bf From steps:} & UC_30#2M \\
{\bf To steps:} & END \\
\end{tabular}
 
\begin{tabular}{|p{0.8in}|p{1.6in}|p{1.6in}|p{1.6in}|}
\hline
Id & User Action & Condition & System Response  \bl 
1M & Select "Add Document" button and browse a Use Case Document
						with an empty mandatory field.  & No document is imported. & The selected document path is inserted in the "Documents
						To Import" list. The "Finish" button is enabled. \bl 
2M & Click on "Finish" button. & - & An error dialog is displayed indicating that the document
						has an error in the xml structure. [FR_TARGET_0125] \bl 
3M & Click on "OK" button. & - & The selected document(s) will not be imported. The focus
						goes back to "Import Documents" window. \bl 
\end{tabular}
\subsection*{Use case UC_55}
\begin{itemize}
\item {\bf Name: }Importing a Use Case Document With Invalid Step ID References
			
\item {\bf Description: }Use Case Document has invalid id references.
			
\end{itemize}
\subsubsection*{Scenario SC37}
\begin{tabular}{p{1in}p{4in}}
{\bf Description:} & The selected Use Case Document contains invalid id
					references. \\
{\bf From steps:} & UC_30#2M \\
{\bf To steps:} & END \\
\end{tabular}
 
\begin{tabular}{|p{0.8in}|p{1.6in}|p{1.6in}|p{1.6in}|}
\hline
Id & User Action & Condition & System Response  \bl 
1M & Select "Add Document" button and browse a Use Case Document
						with invalid ID references.  & No document is imported. & The selected document path is inserted in the "Documents
						To Import" list. The "Finish" button is enabled. \bl 
2M & Click on "Finish" button. & - & A progress bar is displayed indicating that the document
						is being processed. [FR_TARGET_0110] \bl 
3M & Wait the progress bar. & - & A warning message is displayed indicating that the
						document contains errors. \bl 
4M & Click on "OK" button. & - & The imported document is placed in the documents folder
						and is displayed in the "Artifacts" View. The imported use cases
						are displayed in the "Use Cases" View. A special icon indicates
						the use cases with invalid references. Also the error view
						displays details about invalid id references and/or duplicated
						ids. [FR_TARGET_0010, FR_TARGET_0120] \bl 
\end{tabular}
\subsubsection*{Scenario SC38}
\begin{tabular}{p{1in}p{4in}}
{\bf Description:} & Import another valid document. \\
{\bf From steps:} & 1M \\
{\bf To steps:} & 2M \\
\end{tabular}
 
\begin{tabular}{|p{0.8in}|p{1.6in}|p{1.6in}|p{1.6in}|}
\hline
Id & User Action & Condition & System Response  \bl 
1A & Select "Add Document" button and browse another valid Use
						Case Document.  & Another document has to be imported. & The selected document path is inserted in the "Documents
						To Import" list. \bl 
\end{tabular}
\subsection*{Use case UC_60}
\begin{itemize}
\item {\bf Name: }Importing a Use Case Document With Duplicated Step IDs
\item {\bf Description: }Use Case Document has duplicated Step IDs.
\end{itemize}
\subsubsection*{Scenario SC39}
\begin{tabular}{p{1in}p{4in}}
{\bf Description:} & The selected Use Case Document contains duplicated Step
					Ids. \\
{\bf From steps:} & UC_30#2M \\
{\bf To steps:} & END \\
\end{tabular}
 
\begin{tabular}{|p{0.8in}|p{1.6in}|p{1.6in}|p{1.6in}|}
\hline
Id & User Action & Condition & System Response  \bl 
1M & Select "Browser" button and browse a Use Case Document with
						duplicated step ids. The document contains at least two steps in a
						use case with the same ID.  & No document is imported. & The selected document path is inserted in the "Documents
						To Import" list. The "Finish" button is enabled. \bl 
2M & Click on "Finish" button. & - & A progress bar is displayed indicating that the document
						is being processed. [FR_TARGET_0110] \bl 
3M & Wait the progress bar. & - & A warning message is displayed indicating that the
						document contains errors. \bl 
4M & Click on "OK" button. & - & The imported document is placed in the documents folder
						and is displayed in the "Artifacts" View. The imported use cases
						are displayed in the "Use Cases" View. A special icon indicates
						the use cases with invalid references. Also the error view
						displays details about invalid id references and/or duplicated
						ids. [FR_TARGET_0010, FR_TARGET_0120] \bl 
\end{tabular}
\subsubsection*{Scenario SC40}
\begin{tabular}{p{1in}p{4in}}
{\bf Description:} & Select another document (valid). \\
{\bf From steps:} & 1M \\
{\bf To steps:} & 2M \\
\end{tabular}
 
\begin{tabular}{|p{0.8in}|p{1.6in}|p{1.6in}|p{1.6in}|}
\hline
Id & User Action & Condition & System Response  \bl 
1A & Select "Add Document" button and browse another valid Use
						Case Document.  & - & The selected document path is inserted in the "Documents
						To Import" list. \bl 
\end{tabular}
\subsection*{Use case UC_63}
\begin{itemize}
\item {\bf Name: }Importing a Use Case Document with a Feature Already Imported.
			
\item {\bf Description: }The documents have at least one feature with same name
				and ID. Another Use Case Document is already imported.
\end{itemize}
\subsubsection*{Scenario SC41}
\begin{tabular}{p{1in}p{4in}}
{\bf Description:} & A Use Case Document is already imported. The Use Case
					Documents have at least one common feature with same ID and name.
				 \\
{\bf From steps:} & UC_30#2M \\
{\bf To steps:} & END \\
\end{tabular}
 
\begin{tabular}{|p{0.8in}|p{1.6in}|p{1.6in}|p{1.6in}|}
\hline
Id & User Action & Condition & System Response  \bl 
1M & Select "Add Document" button and browse a valid Use Case
						Document. The document and the already imported document have at
						least one common feature with same ID and name and the use cases
						IDs are different.  & There is at least one valid document imported in the
						project. & The "Import Documents" window is displayed. The "Finish"
						button is enabled. \bl 
2M & Click on "Finish" button. & - & A progress bar is displayed indicating that the
						document(s) is (are) being processed and imported.
						[FR_TARGET_0110]  \bl 
3M & Wait the progress bar.  & - & The imported document is placed in a specific folder and
						displayed in the "Artifacts" view. The imported use cases are
						displayed in the "Use Cases" view. The feature icon in the "Use
						Cases" view shows that the feature was merged. [FR_TARGET_0010,
						FR_TARGET_0100] \bl 
\end{tabular}
\subsubsection*{Scenario SC42}
\begin{tabular}{p{1in}p{4in}}
{\bf Description:} & A Use Case Document is already imported. The Use Case
					Documents have at least one common feature with same ID and
					different names. \\
{\bf From steps:} & UC_30#2M \\
{\bf To steps:} & END \\
\end{tabular}
 
\begin{tabular}{|p{0.8in}|p{1.6in}|p{1.6in}|p{1.6in}|}
\hline
Id & User Action & Condition & System Response  \bl 
1A & Select "Add Document" button and browse a valid Use Case
						Document. The document and the already imported document have at
						least one common feature with same ID but different names and the
						use cases IDs are different.  & There is at least one valid document imported in the
						project. & The "Import Documents" window is displayed. The "Finish"
						button is enabled. \bl 
2A & Click on "Finish" button. & - & A progress bar is displayed indicating that the
						document(s) is (are) being processed and imported.
						[FR_TARGET_0110]  \bl 
3A & Wait the progress bar.  & - & The imported document is placed in a specific folder and
						displayed in the "Artifacts" View. The imported use cases are
						displayed in the "Use Cases" View. The feature icon in the "Use
						Cases" view shows that the feature was merged. A new warning is
						displayed in the "Error" view indicating that the features have
						same ID but different names. [FR_TARGET_0010, FR_TARGET_0100,
						FR_TARGET_0120] \bl 
\end{tabular}
\subsubsection*{Scenario SC43}
\begin{tabular}{p{1in}p{4in}}
{\bf Description:} & The document contains a feature with same ID of another
					feature that belongs to a previously imported document.
				 \\
{\bf From steps:} & UC_30#2M \\
{\bf To steps:} & END \\
\end{tabular}
 
\begin{tabular}{|p{0.8in}|p{1.6in}|p{1.6in}|p{1.6in}|}
\hline
Id & User Action & Condition & System Response  \bl 
1B & Select "Add Document" button and browse a valid Use Case
						Document. The document and the already imported document have at
						least one common feature with same ID. The features have at least
						one use case with same ID. & There is at least one valid imported document.
					 & The selected document path is inserted in the "Documents
						To Import" list. \bl 
2B & Click on "Finish" button. & - & A progress bar is displayed indicating that the document
						is being processed. [FR_TARGET_0110] \bl 
3B & Wait the progress bar. & - & An error dialog is displayed indicating the document that
						has an error. [FR_TARGET_0125]  \bl 
4B & Click on "OK" button. & - & The selected document(s) will not be imported. The focus
						goes back to "Import Documents" window. \bl 
\end{tabular}
\subsection*{Use case UC_65}
\begin{itemize}
\item {\bf Name: }Viewing Use Case Documents with Microsoft Word
\item {\bf Description: } Viewing Use Case Documents with Microsoft Word.
			
\end{itemize}
\subsubsection*{Scenario SC44}
\begin{tabular}{p{1in}p{4in}}
{\bf Description:} & Double-click on the document. \\
{\bf From steps:} & START \\
{\bf To steps:} & END \\
\end{tabular}
 
\begin{tabular}{|p{0.8in}|p{1.6in}|p{1.6in}|p{1.6in}|}
\hline
Id & User Action & Condition & System Response  \bl 
1M & Double-click on a Use Case Document in the "Artifacts"
						view.  & A project is opened in TaRGeT. The project has at least
						one imported document. & The Microsoft Word is invoked and the document is opened.
						[FR_TARGET_0005] \bl 
\end{tabular}
\subsection*{Use case UC_70}
\begin{itemize}
\item {\bf Name: }Viewing Valid Use Case in HTML Format
\item {\bf Description: } Viewing Use Case Documents in HTML format.
			
\end{itemize}
\subsubsection*{Scenario SC45}
\begin{tabular}{p{1in}p{4in}}
{\bf Description:} & View valid Use Case Documents in HTML format in the
					main panel, by double-clicking the use case. \\
{\bf From steps:} & START \\
{\bf To steps:} & END \\
\end{tabular}
 
\begin{tabular}{|p{0.8in}|p{1.6in}|p{1.6in}|p{1.6in}|}
\hline
Id & User Action & Condition & System Response  \bl 
1M & Double-click on a use case in the "Use Cases" view.
					 & A project is opened in TaRGeT. The project has at least
						one imported valid Use Case Document.  & The selected Use Case is displayed in HTML format in the
						main panel. [FR_TARGET_0130] \bl 
\end{tabular}
\subsubsection*{Scenario SC46}
\begin{tabular}{p{1in}p{4in}}
{\bf Description:} & View Use Case Documents in HTML format in the main
					panel, by right-clicking the use case. \\
{\bf From steps:} & START \\
{\bf To steps:} & END \\
\end{tabular}
 
\begin{tabular}{|p{0.8in}|p{1.6in}|p{1.6in}|p{1.6in}|}
\hline
Id & User Action & Condition & System Response  \bl 
1A & Right-click on a use case in the "Use Cases" view. & A project is opened in TaRGeT. The project has at least
						one imported valid Use Case Document. & A drop down menu is displayed with two options.
					 \bl 
2A & Select "Open in default view" option. & - & The selected Use Case is displayed in HTML format in the
						main panel. [FR_TARGET_0130]  \bl 
\end{tabular}
\subsubsection*{Scenario SC47}
\begin{tabular}{p{1in}p{4in}}
{\bf Description:} & View Use Case Documents in HTML format in the default
					browser, by right-clicking the use case. \\
{\bf From steps:} & START \\
{\bf To steps:} & END \\
\end{tabular}
 
\begin{tabular}{|p{0.8in}|p{1.6in}|p{1.6in}|p{1.6in}|}
\hline
Id & User Action & Condition & System Response  \bl 
1B & Right-click on a use case in the "Use Cases" view. & A project is opened in TaRGeT. The project has at least
						one imported valid Use Case Document. & A drop down menu is displayed with two options.
					 \bl 
2B & Select "Open with default browser" option. & - & The selected Use Case is displayed in HTML format in the
						default browser. [FR_TARGET_0130]  \bl 
\end{tabular}
\subsection*{Use case UC_71}
\begin{itemize}
\item {\bf Name: }Viewing Invalid Use Case in HTML Format
\item {\bf Description: }Viewing Use Case Documents in HTML format when they have
				some error.
\end{itemize}
\subsubsection*{Scenario SC48}
\begin{tabular}{p{1in}p{4in}}
{\bf Description:} & View Use Case Documents with duplicated step ID in HTML
					format in the main panel, by double-clicking the use case.
				 \\
{\bf From steps:} & START \\
{\bf To steps:} & END \\
\end{tabular}
 
\begin{tabular}{|p{0.8in}|p{1.6in}|p{1.6in}|p{1.6in}|}
\hline
Id & User Action & Condition & System Response  \bl 
1M & Double-click on a use case with duplicated step ID in the
						"Use Cases" view.  & A project is opened in TaRGeT. The project has at least
						one imported use case with duplicated step ID.  & The selected Use Case is displayed in HTML format in the
						main panel. The duplicated steps IDs are highlighted.
						[FR_TARGET_0130] \bl 
\end{tabular}
\subsubsection*{Scenario SC49}
\begin{tabular}{p{1in}p{4in}}
{\bf Description:} & View Use Case Documents with invalid step ID reference
					in HTML format in the main panel, by right-clicking the use case.
				 \\
{\bf From steps:} & START \\
{\bf To steps:} & END \\
\end{tabular}
 
\begin{tabular}{|p{0.8in}|p{1.6in}|p{1.6in}|p{1.6in}|}
\hline
Id & User Action & Condition & System Response  \bl 
1A & Right-click on a use case with invalid step ID reference in
						the "Use Cases" view. & A project is opened in TaRGeT. The project has at least
						one imported use case with invalid step ID reference. & A drop down menu is displayed with two options.
					 \bl 
2A & Select "Open in default view" option. & - & The selected Use Case is displayed in HTML format in the
						main panel. The invalid step ID reference is highlighted.
						[FR_TARGET_0130]  \bl 
\end{tabular}
\subsubsection*{Scenario SC50}
\begin{tabular}{p{1in}p{4in}}
{\bf Description:} & View Use Case Documents with invalid step ID reference
					in HTML format in the default browser, by right-clicking the use
					case. \\
{\bf From steps:} & START \\
{\bf To steps:} & END \\
\end{tabular}
 
\begin{tabular}{|p{0.8in}|p{1.6in}|p{1.6in}|p{1.6in}|}
\hline
Id & User Action & Condition & System Response  \bl 
1B & Right-click on a use case with invalid step ID reference in
						the "Use Cases" view. & A project is opened in TaRGeT. The project has at least
						one imported use case with invalid step ID reference. & A drop down menu is displayed with two options.
					 \bl 
2B & Select "Open with default browser" option. & - & The selected Use Case is displayed in HTML format in the
						default browser. The invalid step ID reference is highlighted.
						[FR_TARGET_0130] \bl 
\end{tabular}
\subsection*{Use case UC_75}
\begin{itemize}
\item {\bf Name: }Viewing a Generated Test Suite with Microsoft Excel
\item {\bf Description: } Viewing Test Suite Documents in TaRGeT with Microsoft
				Excel.
\end{itemize}
\subsubsection*{Scenario SC51}
\begin{tabular}{p{1in}p{4in}}
{\bf Description:} & View Test Suite Document by double-clicking.
				 \\
{\bf From steps:} & START \\
{\bf To steps:} & END \\
\end{tabular}
 
\begin{tabular}{|p{0.8in}|p{1.6in}|p{1.6in}|p{1.6in}|}
\hline
Id & User Action & Condition & System Response  \bl 
1M & Double-click on a test suite document in the "Artifacts"
						view. & A project is opened in TaRGeT. The project has at least
						one already generated test suite. & The Microsoft Excel is invoked and the document is
						opened. The Test Cases and Traceability Matrixes appear in the
						document.[FR_TARGET_0005, FR_TARGET 0243, FR_TARGET 0175,
						FR_TARGET 0240,] \bl 
\end{tabular}
\subsection*{Use case UC_80}
\begin{itemize}
\item {\bf Name: }Renaming a Document
\item {\bf Description: }Renaming Use Case or Test Suite Documents in TaRGeT.
			
\end{itemize}
\subsubsection*{Scenario SC52}
\begin{tabular}{p{1in}p{4in}}
{\bf Description:} & Rename a test suite document. \\
{\bf From steps:} & START \\
{\bf To steps:} & END \\
\end{tabular}
 
\begin{tabular}{|p{0.8in}|p{1.6in}|p{1.6in}|p{1.6in}|}
\hline
Id & User Action & Condition & System Response  \bl 
1M & Right-click on a generated test suite document in the
						"Artifacts" view.  & A project is opened in TaRGeT. The project has at least
						one already generated test suite. & A drop down menu is displayed. \bl 
2M & Click on "Rename" option in the drop down menu.
						[FR_TARGET_0001] & - & A dialog is popped up with a text field. \bl 
3M & Change the document name to a valid name and click on "OK"
						button. [FR_TARGET_0025] & - & The document is renamed. \bl 
\end{tabular}
\subsubsection*{Scenario SC53}
\begin{tabular}{p{1in}p{4in}}
{\bf Description:} & Rename a Use Case Document. \\
{\bf From steps:} & START \\
{\bf To steps:} & END \\
\end{tabular}
 
\begin{tabular}{|p{0.8in}|p{1.6in}|p{1.6in}|p{1.6in}|}
\hline
Id & User Action & Condition & System Response  \bl 
1A & Right-click on a Use Case Document in the "Artifacts" view.
					 & A project is opened in TaRGeT. The project has at least
						one imported Use Case Document. & A drop down menu is displayed. \bl 
2A & Click on "Rename" option in the drop down menu.
						[FR_TARGET_0001] & - & A dialog is popped up with a text field. \bl 
3A & Change the document name to a valid name and click on "OK"
						button. [FR_TARGET_0025] & - & The document is renamed. \bl 
\end{tabular}
\subsubsection*{Scenario SC54}
\begin{tabular}{p{1in}p{4in}}
{\bf Description:} & Select to rename more than one document. \\
{\bf From steps:} & START \\
{\bf To steps:} & END \\
\end{tabular}
 
\begin{tabular}{|p{0.8in}|p{1.6in}|p{1.6in}|p{1.6in}|}
\hline
Id & User Action & Condition & System Response  \bl 
1B & In the "Artifacts" view, Select more than one document and
						right-click on the selection. [FR_TARGET_0001] & A project is opened in TaRGeT. The project has at least
						two documents (test suite or Use Case Documents). & A drop down menu is displayed. The "Rename" option is
						disabled. \bl 
\end{tabular}
\subsubsection*{Scenario SC55}
\begin{tabular}{p{1in}p{4in}}
{\bf Description:} & Type an invalid document name. \\
{\bf From steps:} & 2M \\
{\bf To steps:} & END \\
\end{tabular}
 
\begin{tabular}{|p{0.8in}|p{1.6in}|p{1.6in}|p{1.6in}|}
\hline
Id & User Action & Condition & System Response  \bl 
1C & Try to change the document name to an invalid name (e.g. a
						name containing "<" or "*" characters). [FR_TARGET_0025]
					 & - & A message is displayed indicating that the name is
						invalid. The "OK" button is disabled. \bl 
2C & Click on "Cancel" button. & - & The focus goes back to the main window. \bl 
\end{tabular}
\subsubsection*{Scenario SC56}
\begin{tabular}{p{1in}p{4in}}
{\bf Description:} & Type an invalid document name and retype a valid name.
				 \\
{\bf From steps:} & 2M \\
{\bf To steps:} & END \\
\end{tabular}
 
\begin{tabular}{|p{0.8in}|p{1.6in}|p{1.6in}|p{1.6in}|}
\hline
Id & User Action & Condition & System Response  \bl 
1D & Try to change the document name to an invalid name (e.g. a
						name containing "<" or "*" characters). [FR_TARGET_0025]
					 & - & A message is displayed indicating that the name is
						invalid. The "OK" button is disabled. \bl 
2D & Erase the invalid name and type a valid name. & - & The OK button is enabled. \bl 
3D & Click on the OK button & - & The document is renamed. \bl 
\end{tabular}
\subsubsection*{Scenario SC57}
\begin{tabular}{p{1in}p{4in}}
{\bf Description:} & Cancel the rename. \\
{\bf From steps:} & 2D \\
{\bf To steps:} & END \\
\end{tabular}
 
\begin{tabular}{|p{0.8in}|p{1.6in}|p{1.6in}|p{1.6in}|}
\hline
Id & User Action & Condition & System Response  \bl 
1E & Click on "Cancel" button. & - & The focus goes back to the main window. \bl 
\end{tabular}
\subsubsection*{Scenario SC58}
\begin{tabular}{p{1in}p{4in}}
{\bf Description:} & Type a document name that is being used by another use
					case document. \\
{\bf From steps:} & START \\
{\bf To steps:} & 2C,2D \\
\end{tabular}
 
\begin{tabular}{|p{0.8in}|p{1.6in}|p{1.6in}|p{1.6in}|}
\hline
Id & User Action & Condition & System Response  \bl 
1F & Right-click on a Use Case Document in the "Artifacts" view.
					 & A project is opened in TaRGeT. The project has at least
						two use cases. & A drop down menu is displayed. The "Rename" option is
						enabled. \bl 
2F & Click on "Rename" option in the drop down menu.
						[FR_TARGET_0001] & - & A dialog is popped up with a text field. \bl 
3F & Type a name that is already in use. & - & A message is displayed indicating that the name is
						already in use. The "OK" button is disabled. \bl 
\end{tabular}
\subsubsection*{Scenario SC59}
\begin{tabular}{p{1in}p{4in}}
{\bf Description:} & Type a test suite file name that is being used by
					another test suite file. \\
{\bf From steps:} & START \\
{\bf To steps:} & 2F \\
\end{tabular}
 
\begin{tabular}{|p{0.8in}|p{1.6in}|p{1.6in}|p{1.6in}|}
\hline
Id & User Action & Condition & System Response  \bl 
1G & Right-click on a test case document in the "Artifacts"
						view.  & A project is opened in TaRGeT. The project has at least
						two test suites. & A drop down menu is displayed. The "Rename" option is
						enabled. \bl 
\end{tabular}
\subsubsection*{Scenario SC60}
\begin{tabular}{p{1in}p{4in}}
{\bf Description:} & Type an empty name. \\
{\bf From steps:} & 2M \\
{\bf To steps:} & 2C \\
\end{tabular}
 
\begin{tabular}{|p{0.8in}|p{1.6in}|p{1.6in}|p{1.6in}|}
\hline
Id & User Action & Condition & System Response  \bl 
1H & Type an empty name. [FR_TARGET_0025] & - & A message is displayed indicating that the name must not
						be empty. The "OK" button is disabled. \bl 
\end{tabular}
\subsection*{Use case UC_85}
\begin{itemize}
\item {\bf Name: }Deleting a Document
\item {\bf Description: }Deleting Use Cases or Test Suite Documents in TaRGeT.
			
\end{itemize}
\subsubsection*{Scenario SC61}
\begin{tabular}{p{1in}p{4in}}
{\bf Description:} & Delete a test suite document. \\
{\bf From steps:} & START \\
{\bf To steps:} & END \\
\end{tabular}
 
\begin{tabular}{|p{0.8in}|p{1.6in}|p{1.6in}|p{1.6in}|}
\hline
Id & User Action & Condition & System Response  \bl 
1M & Right-click on a generated test suite document in the
						"Artifacts" view.  & A project is opened in TaRGeT. The project has at least
						one already generated test suite. & A drop down menu is displayed. \bl 
2M & Click on "Delete" option in the drop down menu.
						[FR_TARGET_0003] & - & A dialog is displayed asking a confirmation. \bl 
3M & Click on "Yes" button. & - & The document(s) is (are) deleted. The project work area
						is refreshed. [FR_TARGET_0015] \bl 
\end{tabular}
\subsubsection*{Scenario SC62}
\begin{tabular}{p{1in}p{4in}}
{\bf Description:} & Delete a Use Case Document that is not referred by
					another document. \\
{\bf From steps:} & START \\
{\bf To steps:} & END \\
\end{tabular}
 
\begin{tabular}{|p{0.8in}|p{1.6in}|p{1.6in}|p{1.6in}|}
\hline
Id & User Action & Condition & System Response  \bl 
1B & Right-click on a Use Case Document in the "Artifacts" view.
					 & A project is opened in TaRGeT. The project has at least
						one imported Use Case Document. The imported documents do not
						refer to a use case of another imported document.  & A drop down menu is displayed. \bl 
2B & Click on "Delete" option in the drop down menu.
						[FR_TARGET_0003] & - & A dialog is displayed asking a confirmation. \bl 
3B & Click on "Yes" button. & - & The document(s) is(are) deleted. The project work area is
						refreshed. [FR_TARGET_0015] \bl 
\end{tabular}
\subsubsection*{Scenario SC63}
\begin{tabular}{p{1in}p{4in}}
{\bf Description:} & Delete a Use Case Document that is referred by another
					document. \\
{\bf From steps:} & START \\
{\bf To steps:} & END \\
\end{tabular}
 
\begin{tabular}{|p{0.8in}|p{1.6in}|p{1.6in}|p{1.6in}|}
\hline
Id & User Action & Condition & System Response  \bl 
1C & Right-click on a Use Case Document in the "Artifacts" view.
						The selected document is referred by a use case of another
						imported document.  & A project is opened in TaRGeT. The project has at least
						two imported Use Case Documents. A use case is referred by another
						use case from a different document. & A drop down menu is displayed. \bl 
2C & Click on "Delete" option in the drop down menu.
						[FR_TARGET_0003] & - & A dialog is displayed asking a confirmation and informing
						that the document is being referred by another document.
					 \bl 
3C & Click on "Yes" button. & - & The document(s) is(are) deleted. The project work area is
						refreshed. Some invalid reference errors are displayed in "Error"
						view. [FR_TARGET_0015, FR_TARGET_0120] \bl 
\end{tabular}
\subsubsection*{Scenario SC64}
\begin{tabular}{p{1in}p{4in}}
{\bf Description:} & Delete more than one generated test suite.
				 \\
{\bf From steps:} & START \\
{\bf To steps:} & 2M \\
\end{tabular}
 
\begin{tabular}{|p{0.8in}|p{1.6in}|p{1.6in}|p{1.6in}|}
\hline
Id & User Action & Condition & System Response  \bl 
1D & In the "Artifacts" view, Select more than one test suite
						document and right-click on the selection.  & A project is opened in TaRGeT. The project has at least
						two generated test suites. & A drop down menu is displayed. \bl 
\end{tabular}
\subsubsection*{Scenario SC65}
\begin{tabular}{p{1in}p{4in}}
{\bf Description:} & Delete one test suite and one Use Case Document that is
					being referenced. \\
{\bf From steps:} & START \\
{\bf To steps:} & END \\
\end{tabular}
 
\begin{tabular}{|p{0.8in}|p{1.6in}|p{1.6in}|p{1.6in}|}
\hline
Id & User Action & Condition & System Response  \bl 
1E & In the "Artifacts" view, Select one test suite document and
						a Use Case Document, and right-click on the selection.  & A project is opened in TaRGeT. The project has at least
						two imported Use Case Document and at least one generated test
						suite. The imported Use Case Document is referenced by another
						imported document. & A drop down menu is displayed. \bl 
2E & Click on "Delete" option in the drop down menu.
						[FR_TARGET_0003] & - & A dialog is displayed asking a confirmation and informing
						that the document is being referred by another document.
					 \bl 
3E & Click on "Yes" button. & - & The document(s) is(are) deleted. The project work area is
						refreshed. Some invalid reference errors are displayed in "Error"
						view. [FR_TARGET_0015, FR_TARGET_0120] \bl 
\end{tabular}
\subsubsection*{Scenario SC66}
\begin{tabular}{p{1in}p{4in}}
{\bf Description:} & Delete two Use Case Documents. \\
{\bf From steps:} & START \\
{\bf To steps:} & 2C \\
\end{tabular}
 
\begin{tabular}{|p{0.8in}|p{1.6in}|p{1.6in}|p{1.6in}|}
\hline
Id & User Action & Condition & System Response  \bl 
1F & In the "Artifacts" view, Select one test suite two Use Case
						Documents, and right-click on the selection. One of the selected
						documents refers to the other.  & A project is opened in TaRGeT. The project has two
						imported Use Case Document. One imported Use Case Document is
						referenced by the other. & A drop down menu is displayed. \bl 
\end{tabular}
\subsubsection*{Scenario SC67}
\begin{tabular}{p{1in}p{4in}}
{\bf Description:} & Cancel the document deletion. \\
{\bf From steps:} & 2M,2C \\
{\bf To steps:} & END \\
\end{tabular}
 
\begin{tabular}{|p{0.8in}|p{1.6in}|p{1.6in}|p{1.6in}|}
\hline
Id & User Action & Condition & System Response  \bl 
1A & Click on "No" button. & - & The document(s) is (are) not deleted. \bl 
\end{tabular}
\subsection*{Use case UC_90}
\begin{itemize}
\item {\bf Name: }Updating the Use Case Documents outside the TaRGeT UI
\item {\bf Description: }Updating Use Case Documents outside TaRGeT UI.
			
\end{itemize}
\subsubsection*{Scenario SC68}
\begin{tabular}{p{1in}p{4in}}
{\bf Description:} & Edit a Use Case Document with duplicated step ID
					outside TaRGeT UI. \\
{\bf From steps:} & START \\
{\bf To steps:} & END \\
\end{tabular}
 
\begin{tabular}{|p{0.8in}|p{1.6in}|p{1.6in}|p{1.6in}|}
\hline
Id & User Action & Condition & System Response  \bl 
1M & Open an already imported Use Case Document with the
						Microsoft Word (outside the TaRGeT). The document has a duplicated
						step ID. [FR_TARGET_0005] & TaRGeT is started up and a project is opened. There is
						at least one imported Use Case Document with duplicated step ID in
						the project. & The document is opened in the default viewer outside the
						TaRGeT. \bl 
2M & Edit the duplicated ID, save the document and close it.
					 & - & The document is fixed.  \bl 
3M & Go back to the TaRGeT window. & - & A progress bar is displayed while the project is being
						refreshed. [FR_TARGET_0110, FR_TARGET_0015] \bl 
4M & Wait the progress bar. & - & The fixed duplicated ID error is removed from the "Error"
						view.  \bl 
\end{tabular}
\subsection*{Use case UC_95}
\begin{itemize}
\item {\bf Name: }Rejecting a crashed Use Case Documents
\item {\bf Description: }Rejecting crashed Use Case Documents.
\end{itemize}
\subsubsection*{Scenario SC69}
\begin{tabular}{p{1in}p{4in}}
{\bf Description:} & Damage an imported Use Case Document. \\
{\bf From steps:} & START \\
{\bf To steps:} & END \\
\end{tabular}
 
\begin{tabular}{|p{0.8in}|p{1.6in}|p{1.6in}|p{1.6in}|}
\hline
Id & User Action & Condition & System Response  \bl 
1M & Double-click on a Use Case Document in the "Artifacts"
						view.  & A project is opened in TaRGeT. There is one imported and
						valid Use Case Document. & The Microsoft Word is invoked and the document is opened.
						[FR_TARGET_0005] \bl 
2M & Change the document, damaging its structure, and go back to
						the TaRGeT window.  & - & A progress bar is displayed while the project is being
						refreshed. [FR_TARGET_0110, FR_TARGET_0015] \bl 
3M & Wait the progress bar. & - & A new error is displayed in the "Error" view indicating
						that the document is damaged. No information related to this
						document is displayed in the "Use Case" view. An error icon is
						assigned to the document in the "Artifacts" view. [FR_TARGET_0120,
						FR_TARGET_0125]  \bl 
\end{tabular}
\subsection*{Use case UC_100}
\begin{itemize}
\item {\bf Name: }Rejecting a Use Case Document With Duplicated Feature ID
\item {\bf Description: }Rejecting Use Case Documents that assign to a feature an
				ID that is being used by a feature of another document.
			
\end{itemize}
\subsubsection*{Scenario SC70}
\begin{tabular}{p{1in}p{4in}}
{\bf Description:} & Assign to a feature an ID that is being used by a
					feature of another document. \\
{\bf From steps:} & START \\
{\bf To steps:} & END \\
\end{tabular}
 
\begin{tabular}{|p{0.8in}|p{1.6in}|p{1.6in}|p{1.6in}|}
\hline
Id & User Action & Condition & System Response  \bl 
1M & Double-click on a Use Case Document in the "Artifacts"
						view. & A project is opened in TaRGeT. There is two imported and
						valid Use Case Document. & The Microsoft Word is invoked and the document is opened.
						[FR_TARGET_0005] \bl 
2M & Change the document, by assigning to a feature an ID that
						is being used by a feature of another document, and go back to the
						TaRGeT window.  & - & A progress bar is displayed while the project is being
						refreshed. [FR_TARGET_0110, FR_TARGET_0015] \bl 
3M & Wait the progress bar. & - & A new error is displayed in the "Error" view indicating
						that one of the documents contains a duplicated feature ID. The
						document is rejected and no information related to the document is
						displayed in the "Use Case" view. An error icon is assigned to the
						document in the "Artifacts" view. [FR_TARGET_0120, FR_TARGET_0125]
					 \bl 
\end{tabular}
\subsection*{Use case UC_105}
\begin{itemize}
\item {\bf Name: }Rejecting a Use Case Document With Duplicated Use Case ID
			
\item {\bf Description: }Rejecting Use Case Documents that assign to a use case
				an ID that is being used by another use case of the same feature.
			
\end{itemize}
\subsubsection*{Scenario SC71}
\begin{tabular}{p{1in}p{4in}}
{\bf Description:} & Assign to a use case an ID that is being used by
					another use of the same feature. \\
{\bf From steps:} & START \\
{\bf To steps:} & END \\
\end{tabular}
 
\begin{tabular}{|p{0.8in}|p{1.6in}|p{1.6in}|p{1.6in}|}
\hline
Id & User Action & Condition & System Response  \bl 
1M & Double-click on a Use Case Document in the "Artifacts"
						view. & A project is opened in TaRGeT. There is one imported and
						valid Use Case Document. The document has at least one feature
						with more than one use case. & The Microsoft Word is invoked and the document is opened.
						[FR_TARGET_0005] \bl 
2M & Change the document, by assigning to a use case an ID that
						is being used by another use case of the same feature, and go back
						to the TaRGeT window.  & - & A progress bar is displayed while the project is being
						refreshed. [FR_TARGET_0110, FR_TARGET_0015] \bl 
3M & Wait the progress bar. & - & A new error is displayed in the "Error" view indicating
						that one of the documents contains a duplicated use case ID. The
						document is rejected and no information related to the document is
						displayed in the "Use Case" view. An error icon is assigned to the
						document in the "Artifacts" view. [FR_TARGET_0120, FR_TARGET_0125]
					 \bl 
\end{tabular}
\subsection*{Use case UC_106}
\begin{itemize}
\item {\bf Name: }Searching in Use Case Document
\item {\bf Description: }Searching in Use Case Document.
\end{itemize}
\subsubsection*{Scenario SC72}
\begin{tabular}{p{1in}p{4in}}
{\bf Description:} & Perform a search when some valid Use Case Documents are
					already imported. \\
{\bf From steps:} & START \\
{\bf To steps:} & END \\
\end{tabular}
 
\begin{tabular}{|p{0.8in}|p{1.6in}|p{1.6in}|p{1.6in}|}
\hline
Id & User Action & Condition & System Response  \bl 
1M & Choose "Tools" option in the menu bar. & TaRGeT is started up. A project is already opened. There
						is at least one document imported. No error is listed in the
						"Error" list. & A drop down menu is displayed. "Search" option is
						available. \bl 
2M & Choose "Search" option in the drop down menu. & - & "Search" window is displayed. [FR_TARGET_0135] \bl 
3M & Type a query in the input area. & - & The "Find" field is filled. \bl 
4M & Click on "Search" button. & - & The search results are displayed in "Search Results"
						View. Verify if the results are in accordance to the query.
					 \bl 
\end{tabular}
\subsubsection*{Scenario SC73}
\begin{tabular}{p{1in}p{4in}}
{\bf Description:} & Cancel the search. \\
{\bf From steps:} & 3M \\
{\bf To steps:} & END \\
\end{tabular}
 
\begin{tabular}{|p{0.8in}|p{1.6in}|p{1.6in}|p{1.6in}|}
\hline
Id & User Action & Condition & System Response  \bl 
1A & Click on "Cancel" button.  & - & The focus goes back to the work area. \bl 
\end{tabular}
\subsubsection*{Scenario SC74}
\begin{tabular}{p{1in}p{4in}}
{\bf Description:} & Using the specific field "Use Case Identifier".
				 \\
{\bf From steps:} & 2M \\
{\bf To steps:} & END \\
\end{tabular}
 
\begin{tabular}{|p{0.8in}|p{1.6in}|p{1.6in}|p{1.6in}|}
\hline
Id & User Action & Condition & System Response  \bl 
1B & "Type a query for serching based on use case Id, i.e.
						. A
						use case with id
						must be contained in the project. & - & The "Find" field is filled. \bl 
2B & Click on "Search" button. & - & At least one use case is displayed in "Search Results"
						View. [FR_TARGET_0135] \bl 
\end{tabular}
\subsubsection*{Scenario SC75}
\begin{tabular}{p{1in}p{4in}}
{\bf Description:} & Using the specific field "From Step". \\
{\bf From steps:} & 2M \\
{\bf To steps:} & 4M \\
\end{tabular}
 
\begin{tabular}{|p{0.8in}|p{1.6in}|p{1.6in}|p{1.6in}|}
\hline
Id & User Action & Condition & System Response  \bl 
1C & Type "from step: START" in the input area. & - & The "Find" field is filled. \bl 
\end{tabular}
\subsubsection*{Scenario SC76}
\begin{tabular}{p{1in}p{4in}}
{\bf Description:} & Using shortcut. \\
{\bf From steps:} & START \\
{\bf To steps:} & 3M \\
\end{tabular}
 
\begin{tabular}{|p{0.8in}|p{1.6in}|p{1.6in}|p{1.6in}|}
\hline
Id & User Action & Condition & System Response  \bl 
1D & Press CTRL+F & - & "Search" window is displayed. [FR_TARGET_0135] \bl 
\end{tabular}
\subsection*{Use case UC_107}
\begin{itemize}
\item {\bf Name: }Viewing Search Results
\item {\bf Description: }Viewing search results.
\end{itemize}
\subsubsection*{Scenario SC77}
\begin{tabular}{p{1in}p{4in}}
{\bf Description:} & View found Use Case Documents in HTML format by
					double-clicking the use case. \\
{\bf From steps:} & UC_106#4M \\
{\bf To steps:} & END \\
\end{tabular}
 
\begin{tabular}{|p{0.8in}|p{1.6in}|p{1.6in}|p{1.6in}|}
\hline
Id & User Action & Condition & System Response  \bl 
1M & Double-click on a use case in the "Search Results" view.
					 & - & The selected Use Case is displayed in HTML format in the
						main panel. [FR_TARGET_0130]  \bl 
\end{tabular}
\subsection*{Use case UC_130}
\begin{itemize}
\item {\bf Name: }Generating Test Suites Changing Test Case Field Parameters
			
\item {\bf Description: }Configuring test case fields.
\end{itemize}
\subsubsection*{Scenario SC138}
\begin{tabular}{p{1in}p{4in}}
{\bf Description:} & Configuring test cases parameters. \\
{\bf From steps:} & START \\
{\bf To steps:} & END \\
\end{tabular}
 
\begin{tabular}{|p{0.8in}|p{1.6in}|p{1.6in}|p{1.6in}|}
\hline
Id & User Action & Condition & System Response  \bl 
1M & Go to "Tools" in the menu bar.  & There is an already created project with at least one
						imported use case without errors. [FR_TARGET_0100] & A drop down menu is displayed with "Search",
						"Preferences" and "On The Fly Generation" options. The
						"Preferences" option is available. \bl 
2M & Select "Preferences" option. & - & The Preferences wizard is displayed with "Test Case ID",
						"Test Case Initial ID", "Empty Field", "Objective Prefix", "Print
						Use Case Description", "Print Flow Description" and "Keep
						Requirements" fields. \bl 
\end{tabular}
\subsubsection*{Scenario SC139}
\begin{tabular}{p{1in}p{4in}}
{\bf Description:} & Cancel the preferences. \\
{\bf From steps:} & 2B,2C,2D,2E,2F,2G,2H,2I,2J,2K \\
{\bf To steps:} & END \\
\end{tabular}
 
\begin{tabular}{|p{0.8in}|p{1.6in}|p{1.6in}|p{1.6in}|}
\hline
Id & User Action & Condition & System Response  \bl 
1A & Press "Cancel" button. & - & The field was not changed, the preferences screen is
						closed and the focus goes back to the main window.
						[FR_TARGET_0231] \bl 
\end{tabular}
\subsubsection*{Scenario SC140}
\begin{tabular}{p{1in}p{4in}}
{\bf Description:} & Changing "Objective Prefix" field. \\
{\bf From steps:} & 2M \\
{\bf To steps:} & END \\
\end{tabular}
 
\begin{tabular}{|p{0.8in}|p{1.6in}|p{1.6in}|p{1.6in}|}
\hline
Id & User Action & Condition & System Response  \bl 
1B & Type a wanted prefix in the "Objective prefix" field and
						press OK button. & The "Print Flow Description" field in the Preferences
						window cannot be "None" & The preferences configuration is changed.
						[FR_TARGET_0231] \bl 
2B & Go to "On The Fly Generation" and select each test case
						from the list. & - & All test cases contain their objective field with the
						prefix defined by the user. [FR_TARGET_0235, ,FR_TARGET_0237]
					 \bl 
\end{tabular}
\subsubsection*{Scenario SC141}
\begin{tabular}{p{1in}p{4in}}
{\bf Description:} & Changing "Empty Field" parameter. \\
{\bf From steps:} & 2M \\
{\bf To steps:} & END \\
\end{tabular}
 
\begin{tabular}{|p{0.8in}|p{1.6in}|p{1.6in}|p{1.6in}|}
\hline
Id & User Action & Condition & System Response  \bl 
1C & Type a wanted content in the "Empty Field" field and press
						OK button. & The "Print Flow Description" and "Print Use case
						description" fields is configured to "None" & The preferences configuration is changed.
						[FR_TARGET_0231] \bl 
2C & Go to "On The Fly Generation" and select each test case
						from the list. & - & In all test cases, the fields that are empty in the Use
						Case Document are filled with this content defined by the user.
						[FR_TARGET_0235, ,FR_TARGET_0237] \bl 
\end{tabular}
\subsubsection*{Scenario SC142}
\begin{tabular}{p{1in}p{4in}}
{\bf Description:} & Changing "Test Case Id" parameter to a fixed value.
				 \\
{\bf From steps:} & 2M \\
{\bf To steps:} & END \\
\end{tabular}
 
\begin{tabular}{|p{0.8in}|p{1.6in}|p{1.6in}|p{1.6in}|}
\hline
Id & User Action & Condition & System Response  \bl 
1D & Type "<tc_featureid>_TESTE_<tc_id>" in the
						"Test Case Id" field and press OK button. & - & The preferences configuration is changed.
						[FR_TARGET_0231] \bl 
2D & Go to "On The Fly Generation" and select each test case
						from the list. & - & All Test Case Id fields are filled according the standard
						defined by the user. [FR_TARGET_0235, ,FR_TARGET_0237] \bl 
\end{tabular}
\subsubsection*{Scenario SC143}
\begin{tabular}{p{1in}p{4in}}
{\bf Description:} & Changing "Test Case initial Id" parameter \\
{\bf From steps:} & 2M \\
{\bf To steps:} & END \\
\end{tabular}
 
\begin{tabular}{|p{0.8in}|p{1.6in}|p{1.6in}|p{1.6in}|}
\hline
Id & User Action & Condition & System Response  \bl 
1E & Type an initial test case id (number) in the "Test Case
						Initial Id" field and press OK button. & - & The preferences configuration is changed.
						[FR_TARGET_0231] \bl 
2E & Go to "On The Fly Generation" and select each test case
						from the list. & - & All Test Case Ids are filled according the test case
						initial id (crescent order) defined by the user. [FR_TARGET_0235,
						,FR_TARGET_0237] \bl 
\end{tabular}
\subsubsection*{Scenario SC144}
\begin{tabular}{p{1in}p{4in}}
{\bf Description:} & Changing "Print Use Case Description" parameter to
					"All". \\
{\bf From steps:} & 2M \\
{\bf To steps:} & END \\
\end{tabular}
 
\begin{tabular}{|p{0.8in}|p{1.6in}|p{1.6in}|p{1.6in}|}
\hline
Id & User Action & Condition & System Response  \bl 
1F & Choose the "ALL" option in the "Print Use Case Description"
						field and press OK button. & - & The preferences configuration is changed.
						[FR_TARGET_0232] \bl 
2F & Go to "On The Fly Generation" and select each test case
						from the list. & - & All descriptions of all Use cases related to the test
						case are concatenated in the Case Description field for each test
						case. [FR_TARGET_0235, ,FR_TARGET_0237] \bl 
\end{tabular}
\subsubsection*{Scenario SC145}
\begin{tabular}{p{1in}p{4in}}
{\bf Description:} & Changing "Print Use Case Description" parameter to
					"Last". \\
{\bf From steps:} & 2M \\
{\bf To steps:} & END \\
\end{tabular}
 
\begin{tabular}{|p{0.8in}|p{1.6in}|p{1.6in}|p{1.6in}|}
\hline
Id & User Action & Condition & System Response  \bl 
1G & Choose the "Last" option in the "Print Use Case
						Description" field and press OK button. & - & The preferences configuration is changed.
						[FR_TARGET_0232] \bl 
2G & Go to "On The Fly Generation" and select each test case
						from the list. & - & Just the description of last Use case related to each
						test case is shown in the Case Description field. [FR_TARGET_0235,
						,FR_TARGET_0237] \bl 
\end{tabular}
\subsubsection*{Scenario SC146}
\begin{tabular}{p{1in}p{4in}}
{\bf Description:} & Changing "Print Use Case Description" parameter to
					"None". \\
{\bf From steps:} & 2M \\
{\bf To steps:} & END \\
\end{tabular}
 
\begin{tabular}{|p{0.8in}|p{1.6in}|p{1.6in}|p{1.6in}|}
\hline
Id & User Action & Condition & System Response  \bl 
1H & Choose the "None" option in the "Print Use Case
						Description" field and press OK button. & - & The preferences configuration is changed.
						[FR_TARGET_0232] \bl 
2H & Go to "On The Fly Generation" and select each test case
						from the list. & - & The Case Description field does not contain any
						description and it is filled with "Empty Field" suggested in
						Preferences window. [FR_TARGET_0235, ,FR_TARGET_0237,
						FR_TARGET_231] \bl 
\end{tabular}
\subsubsection*{Scenario SC147}
\begin{tabular}{p{1in}p{4in}}
{\bf Description:} & Changing "Print Flow Description" parameter to "All".
				 \\
{\bf From steps:} & 2M \\
{\bf To steps:} & END \\
\end{tabular}
 
\begin{tabular}{|p{0.8in}|p{1.6in}|p{1.6in}|p{1.6in}|}
\hline
Id & User Action & Condition & System Response  \bl 
1I & Choose the "ALL" option in the "Print Flow Description"
						field and press OK button. & - & The preferences configuration is changed.
						[FR_TARGET_0232] \bl 
2I & Go to "On The Fly Generation" and select each test case
						from the list. & - & All descriptions of all Flows related to each test case
						are concatenated after "Objective Prefix" in the respective field.
						[FR_TARGET_0235, ,FR_TARGET_0237] \bl 
\end{tabular}
\subsubsection*{Scenario SC148}
\begin{tabular}{p{1in}p{4in}}
{\bf Description:} & Changing "Print Flow Description" parameter to "Last".
				 \\
{\bf From steps:} & 2M \\
{\bf To steps:} & END \\
\end{tabular}
 
\begin{tabular}{|p{0.8in}|p{1.6in}|p{1.6in}|p{1.6in}|}
\hline
Id & User Action & Condition & System Response  \bl 
1J & Choose the "Last" option in the "Print Flow Description"
						field and press OK button. & - & The preferences configuration is changed.
						[FR_TARGET_0232] \bl 
2J & Go to "On The Fly Generation" and select each test case
						from the list. & - & Just description of last Flow related to each test case
						is shown after "Objective Prefix" in the respective field.
						[FR_TARGET_0235, ,FR_TARGET_0237] \bl 
\end{tabular}
\subsubsection*{Scenario SC149}
\begin{tabular}{p{1in}p{4in}}
{\bf Description:} & Changing "Print Flow Description" parameter to "None".
				 \\
{\bf From steps:} & 2M \\
{\bf To steps:} & END \\
\end{tabular}
 
\begin{tabular}{|p{0.8in}|p{1.6in}|p{1.6in}|p{1.6in}|}
\hline
Id & User Action & Condition & System Response  \bl 
1K & Choose the "None" option in the "Print Flow Description"
						field and press OK button. & - & The preferences configuration is changed.
						[FR_TARGET_0232] \bl 
2K & Go to "On The Fly Generation" and select each test case
						from the list. & - & The Flow Description field does not contain any
						description and it is filled with "Empty Field" suggested in
						Preferences window. [FR_TARGET_0235, ,FR_TARGET_0237,
						FR_TARGET_231] \bl 
\end{tabular}
\subsubsection*{Scenario SC150}
\begin{tabular}{p{1in}p{4in}}
{\bf Description:} & Check "Keep Requirements" parameter. \\
{\bf From steps:} & 2M \\
{\bf To steps:} & END \\
\end{tabular}
 
\begin{tabular}{|p{0.8in}|p{1.6in}|p{1.6in}|p{1.6in}|}
\hline
Id & User Action & Condition & System Response  \bl 
1L & Check the "Keep Requirement" option and press OK button.
					 & - & The preferences configuration is changed.[FR_TARGET_0233]
					 \bl 
2L & Go to "On The Fly Generation" and select each test case
						from the list. & - & The requirements are shown into each step that is
						associated, besides to be shown at Requirements list.
						[FR_TARGET_0160, FR_TARGET_0170, ,FR_TARGET_0175] \bl 
\end{tabular}
\subsection*{Use case UC_135}
\begin{itemize}
\item {\bf Name: }Viewing the About Window
\item {\bf Description: }Displaying the About Window.
\end{itemize}
\subsubsection*{Scenario SC160}
\begin{tabular}{p{1in}p{4in}}
{\bf Description:} & Access the about window. \\
{\bf From steps:} & START \\
{\bf To steps:} & END \\
\end{tabular}
 
\begin{tabular}{|p{0.8in}|p{1.6in}|p{1.6in}|p{1.6in}|}
\hline
Id & User Action & Condition & System Response  \bl 
1M & Choose "Help" option in the menu bar.  & The TaRGeT is already started up. & A drop down menu is displayed. [FR_TARGET_0205]
					 \bl 
2M & Choose "About TaRGeT" option in the drop down menu.
					 & - & "About" window is displayed. The content of the windows
						is detailed in TaRGeT requirements document. [FR_TARGET_0210]
					 \bl 
\end{tabular}
\subsection*{Use case UC_140}
\begin{itemize}
\item {\bf Name: }Viewing the Help Window
\item {\bf Description: } Displaying the Help Window.
\end{itemize}
\subsubsection*{Scenario SC161}
\begin{tabular}{p{1in}p{4in}}
{\bf Description:} & Access the help window. \\
{\bf From steps:} & START \\
{\bf To steps:} & END \\
\end{tabular}
 
\begin{tabular}{|p{0.8in}|p{1.6in}|p{1.6in}|p{1.6in}|}
\hline
Id & User Action & Condition & System Response  \bl 
1M & Choose "Help" option in the menu bar. & The TaRGeT is already started up. & A drop down menu is displayed.  \bl 
2M & Choose "Help Contents" option in the drop down menu.
					 & - & "Help" window is displayed. [FR_TARGET_0205] \bl 
\end{tabular}
\subsection*{Use case UC_110}
\begin{itemize}
\item {\bf Name: }Generating Test Suites through the On The Fly Generation
\item {\bf Description: }Generating Test Cases through the On The Fly Generation.
			
\end{itemize}
\subsubsection*{Scenario SC78}
\begin{tabular}{p{1in}p{4in}}
{\bf Description:} & Generate test cases in a spreadsheet format.
				 \\
{\bf From steps:} & START \\
{\bf To steps:} & END \\
\end{tabular}
 
\begin{tabular}{|p{0.8in}|p{1.6in}|p{1.6in}|p{1.6in}|}
\hline
Id & User Action & Condition & System Response  \bl 
1M & Choose the "Tools" option in the menu bar.  & The TaRGeT is already started up. There is an already
						created project with at least one imported use case without
						errors. [FR_TARGET_0100] & A drop down menu is displayed with "Search",
						"Preferences" and "On The Fly Generation" options. The "On The Fly
						Generation" option is available.  \bl 
2M & Choose the "On The Fly Generation" option in the drop down
						menu. & - & The "On The Fly Generation Editor" window is displayed.
						The "Test Selection Page" is displayed. [FR_TARGET_0234]
					 \bl 
3M & Go to "Test Cases" tab in the "On The Fly Generation
						Editor" view and click on the "Save Test Suite" button. & No filter is selected. [FR_TARGET_0150] & The "Test Cases Generation Summary" pop up is displayed
						informing the number of generated test cases and the total time
						required to complete the process. [FR_TARGET_0160] \bl 
4M & Click on the "ok" button. & - & The test suite is generated with the default name and the
						correct structure. The new test suite is in the "Artifacts" view
						in the TestCases folder. Notice that the generated test cases are
						according to the test cases list of the "Test Cases viewer" tab.
						[FR_TARGET_0237, FR_TARGET_0239, FR_TARGET_0170] \bl 
\end{tabular}
\subsubsection*{Scenario SC79}
\begin{tabular}{p{1in}p{4in}}
{\bf Description:} & Generating tests through a document with errors.
				 \\
{\bf From steps:} & START \\
{\bf To steps:} & END \\
\end{tabular}
 
\begin{tabular}{|p{0.8in}|p{1.6in}|p{1.6in}|p{1.6in}|}
\hline
Id & User Action & Condition & System Response  \bl 
1A & Choose the "Tools" option in the menu bar.  & The TaRGeT is already started up. A project is opened
						and some Use Case Documentation is already imported. There is at
						least one error in the "Error" list. & A drop down menu is displayed. The "On The Fly
						Generation" option is available. \bl 
2A & Choose the "On The Fly Generation" option in the drop down
						menu.  & - & A dialog is displayed indicating that there is/are
						error(s) in the project and it is not possible to generate tests.
						[FR_TARGET_0239] \bl 
3A & Click on "Ok" button. & - & The focus comes back to the work area. \bl 
\end{tabular}
\subsubsection*{Scenario SC80}
\begin{tabular}{p{1in}p{4in}}
{\bf Description:} & Generating tests using a template. \\
{\bf From steps:} & 2M \\
{\bf To steps:} & UC_110#4M \\
\end{tabular}
 
\begin{tabular}{|p{0.8in}|p{1.6in}|p{1.6in}|p{1.6in}|}
\hline
Id & User Action & Condition & System Response  \bl 
1B & Go to "Test Cases" tab in the "On The Fly Generation
						Editor" view and click on the "Save Test Suite" button. & No filter is selected. [FR_TARGET_0150] & The "Import Template" pop up is displayed asking you if
						you want to import a spreadsheet template to generate the test
						suit based on it.  \bl 
2B & Press "Yes" button. & - & A "Browse File" window is showed to select the .xls file.
					 \bl 
3B & Choose a .xls template. & There is available a spreadsheet template exported from
						Test Central. & The "Test Cases Generation Summary" pop up is displayed
						informing the number of generated test cases and the total time
						required to complete the process.  \bl 
\end{tabular}
\subsubsection*{Scenario SC81}
\begin{tabular}{p{1in}p{4in}}
{\bf Description:} & Generating tests without a template. \\
{\bf From steps:} & 1B \\
{\bf To steps:} & 4M \\
\end{tabular}
 
\begin{tabular}{|p{0.8in}|p{1.6in}|p{1.6in}|p{1.6in}|}
\hline
Id & User Action & Condition & System Response  \bl 
1C & Press "No" button. & - & A "Browse File" window is showed to select the .xls file.
					 \bl 
\end{tabular}
\subsection*{Use case UC_111}
\begin{itemize}
\item {\bf Name: }Generating Test Suites using Requirement Filter
\item {\bf Description: }Selecting test cases by Requirement Filter.
			
\end{itemize}
\subsubsection*{Scenario SC82}
\begin{tabular}{p{1in}p{4in}}
{\bf Description:} & Filtering test suit generation by requirements.
				 \\
{\bf From steps:} & UC_110#2M \\
{\bf To steps:} & END \\
\end{tabular}
 
\begin{tabular}{|p{0.8in}|p{1.6in}|p{1.6in}|p{1.6in}|}
\hline
Id & User Action & Condition & System Response  \bl 
1M & In "Selection" tab in "On The Fly Generation Editor",
						select some requirements (not all) in the Requirements Selection
						area.  & The imported use cases refer to more than one
						requirement. & Some requirements are selected. [FR_TARGET_0145]
					 \bl 
2M & Go to "Test Cases Viewer" tab. & - &  Verify that test cases in the "Generated Test Cases"
						list are according to the selected filter. [FR_TARGET_0236,
						FR_TARGET_0237] \bl 
\end{tabular}
\subsubsection*{Scenario SC83}
\begin{tabular}{p{1in}p{4in}}
{\bf Description:} & All requirements are selected. \\
{\bf From steps:} & UC_110#2M \\
{\bf To steps:} & 2M,UC_121#1M \\
\end{tabular}
 
\begin{tabular}{|p{0.8in}|p{1.6in}|p{1.6in}|p{1.6in}|}
\hline
Id & User Action & Condition & System Response  \bl 
1A & In the tab "Selection" in the "On The Fly Generation
						Editor", click on "Select all" button in the page of Requirements.
					 & The imported use cases have one or more than one step
						which refers to distinct requirements & All requirements are selected. [FR_TARGET_0145]
					 \bl 
\end{tabular}
\subsubsection*{Scenario SC84}
\begin{tabular}{p{1in}p{4in}}
{\bf Description:} & Deselect all requirements selection. \\
{\bf From steps:} & 1M \\
{\bf To steps:} & 2M \\
\end{tabular}
 
\begin{tabular}{|p{0.8in}|p{1.6in}|p{1.6in}|p{1.6in}|}
\hline
Id & User Action & Condition & System Response  \bl 
1D & In the tab "Selection" in the "On The Fly Generation
						Editor", click on "Deselect all" button in the page of
						Requirements.  & - & All requirements are deselected. [FR_TARGET_0145]
					 \bl 
\end{tabular}
\subsubsection*{Scenario SC85}
\begin{tabular}{p{1in}p{4in}}
{\bf Description:} & There is no reference to any requirement in the
					imported use cases document(s). \\
{\bf From steps:} & START \\
{\bf To steps:} & END \\
\end{tabular}
 
\begin{tabular}{|p{0.8in}|p{1.6in}|p{1.6in}|p{1.6in}|}
\hline
Id & User Action & Condition & System Response  \bl 
1E & Choose "Tools" option in the menu bar.  & No requirement is referenced in the imported use cases
						document(s). & A drop down menu is displayed. "On The Fly Generation"
						option is available. \bl 
2E & Choose "On The Fly Generation" option in the drop down
						menu. & - & The "On The Fly Generation Editor" window is displayed.
						The "Requirements List" is disabled. The "Select All" and
						"Deselect All" buttons are disabled. [FR_TARGET_0145] \bl 
\end{tabular}
\subsection*{Use case UC_112}
\begin{itemize}
\item {\bf Name: }Generating Test Suites using Use Cases Filter
\item {\bf Description: }Selecting test cases by Use Cases Filter.
\end{itemize}
\subsubsection*{Scenario SC86}
\begin{tabular}{p{1in}p{4in}}
{\bf Description:} & Filtering test suit generation by use cases.
				 \\
{\bf From steps:} & UC_110#2M \\
{\bf To steps:} & UC_111#2M \\
\end{tabular}
 
\begin{tabular}{|p{0.8in}|p{1.6in}|p{1.6in}|p{1.6in}|}
\hline
Id & User Action & Condition & System Response  \bl 
1M & In the tab "Selection" in the "On The Fly Generation
						Editor", select one or more (press CTRL) Use Cases available in
						the document.  & The imported use cases document have at least use case.
					 & The use cases are selected.  \bl 
2M & Click on "Add Use case" button.  & - & One or more Use cases are added in the "Selected Use
						Cases" area. [FR_TARGET_0153] \bl 
\end{tabular}
\subsubsection*{Scenario SC87}
\begin{tabular}{p{1in}p{4in}}
{\bf Description:} & Remove Use Cases from filter. \\
{\bf From steps:} & 2M \\
{\bf To steps:} & UC_111#2M \\
\end{tabular}
 
\begin{tabular}{|p{0.8in}|p{1.6in}|p{1.6in}|p{1.6in}|}
\hline
Id & User Action & Condition & System Response  \bl 
1A & In the tab "Selection" in the "On The Fly Generation
						Editor", select some use cases on the "Selected Use Cases" area.
					 & - & The use cases are selected. \bl 
2A & Click on the "Remove Use Case" button.  & - & The selected use cases were removed from "Selected Use
						Cases" area. [FR_TARGET_0153] \bl 
\end{tabular}
\subsection*{Use case UC_115}
\begin{itemize}
\item {\bf Name: }Generating Test Suites with Test Purposes
\item {\bf Description: }Selecting test cases by Test Purposes Filter.
			
\end{itemize}
\subsubsection*{Scenario SC88}
\begin{tabular}{p{1in}p{4in}}
{\bf Description:} & Filtering a test suite generation by a test purpose
				 \\
{\bf From steps:} & UC_110#2M \\
{\bf To steps:} & UC_111#2M \\
\end{tabular}
 
\begin{tabular}{|p{0.8in}|p{1.6in}|p{1.6in}|p{1.6in}|}
\hline
Id & User Action & Condition & System Response  \bl 
1M & Go to "Test Purpose Creation" and select *-* step in the
						"Steps" area and click on "Select" button.    & - & A "*" symbol is added to "Current Test Purpose" field.
					 \bl 
2M & In the "Steps" area, select a step and click on "Select"
						button. & - & A reference to this step is added to the "Current Test
						Purpose" field. [FR_TARGET_0151] \bl 
3M & In the "Steps" area, select *-* step and click on "Select"
						button.    & - & A "*" symbol is added to "Current Test Purpose" field.
					 \bl 
4M & Click on "OK" button. & - & The test purpose is added to the "Created Test Purposes"
						list. The "Current Test Purpose" list is empty. \bl 
\end{tabular}
\subsubsection*{Scenario SC89}
\begin{tabular}{p{1in}p{4in}}
{\bf Description:} & Move the step to lower level. \\
{\bf From steps:} & START \\
{\bf To steps:} & 3M \\
\end{tabular}
 
\begin{tabular}{|p{0.8in}|p{1.6in}|p{1.6in}|p{1.6in}|}
\hline
Id & User Action & Condition & System Response  \bl 
1A & Go to "On The Fly Generation" -> "Selection" tab ->
						"Test Purpose Creation" and in "Steps" area, select a step and
						click on "Select" button. & - & A reference to this step is added to the "Current Test
						Purpose" field. [FR_TARGET_0151] \bl 
2A & Go to "Test Purpose Creation" and select *-* step in the
						"Steps" area and click on "Select" button.    & - & A "*" symbol is added to "Current Test Purpose" field.
					 \bl 
3A & Select the first step in the "Current Test Purpose" field
						and click in the "Down" button. & - & The selected test is in a lower level.  \bl 
\end{tabular}
\subsubsection*{Scenario SC90}
\begin{tabular}{p{1in}p{4in}}
{\bf Description:} & Move the step to upper level. \\
{\bf From steps:} & 2A \\
{\bf To steps:} & 3M \\
\end{tabular}
 
\begin{tabular}{|p{0.8in}|p{1.6in}|p{1.6in}|p{1.6in}|}
\hline
Id & User Action & Condition & System Response  \bl 
1B & Select the last step in the "Current Test Purpose" field
						and click in the "Up" button. & - & The selected test is in upper level. \bl 
\end{tabular}
\subsubsection*{Scenario SC91}
\begin{tabular}{p{1in}p{4in}}
{\bf Description:} & Clear one or more step(s) from "Current Test Purpose"
					field. \\
{\bf From steps:} & 3M \\
{\bf To steps:} & END \\
\end{tabular}
 
\begin{tabular}{|p{0.8in}|p{1.6in}|p{1.6in}|p{1.6in}|}
\hline
Id & User Action & Condition & System Response  \bl 
1C & Select one or more steps on "Current Test Purpose" area and
						click on "Clean" button.  & - & The selected step(s) was(were) removed from "Current Test
						Purpose" field.  \bl 
\end{tabular}
\subsubsection*{Scenario SC92}
\begin{tabular}{p{1in}p{4in}}
{\bf Description:} & Remove a test purpose from the "Created Test Purposes"
					list. \\
{\bf From steps:} & 4M \\
{\bf To steps:} & END \\
\end{tabular}
 
\begin{tabular}{|p{0.8in}|p{1.6in}|p{1.6in}|p{1.6in}|}
\hline
Id & User Action & Condition & System Response  \bl 
1D & Select a test purpose on "Created Test Purposes" list.
					 & - & The test purpose is highlighted. [FR_TARGET_0151]
					 \bl 
2D & Click on "Clean" button on "Created Test Purposes" area.
					 & - & The test purpose was removed from the "Created Test
						Purposes" list. \bl 
\end{tabular}
\subsubsection*{Scenario SC93}
\begin{tabular}{p{1in}p{4in}}
{\bf Description:} & Remove all test purposes from the created test purposes
					list. \\
{\bf From steps:} & START \\
{\bf To steps:} & END \\
\end{tabular}
 
\begin{tabular}{|p{0.8in}|p{1.6in}|p{1.6in}|p{1.6in}|}
\hline
Id & User Action & Condition & System Response  \bl 
1E & Click on "Clear All" button on "Created Test Purposes"
						area.  & There is more than one test purpose on "Test purposes"
						list. & The "Created Test Purposes" list is empty.
						[FR_TARGET_0151] \bl 
\end{tabular}
\subsection*{Use case UC_116}
\begin{itemize}
\item {\bf Name: }Generating Test Suites with Path Coverage
\item {\bf Description: }Selecting test cases from Path Coverage.
\end{itemize}
\subsubsection*{Scenario SC94}
\begin{tabular}{p{1in}p{4in}}
{\bf Description:} & Path coverage percentage is 50%. \\
{\bf From steps:} & UC_110#2M \\
{\bf To steps:} & UC_111#2M \\
\end{tabular}
 
\begin{tabular}{|p{0.8in}|p{1.6in}|p{1.6in}|p{1.6in}|}
\hline
Id & User Action & Condition & System Response  \bl 
1M & Go to "Path Coverage" area and set to 50.  & - & "Path Coverage" is 50%. [FR_TARGET_0152] \bl 
\end{tabular}
\subsection*{Use case UC_117}
\begin{itemize}
\item {\bf Name: }Visualizing Test Cases in Test Cases Viewer
\item {\bf Description: }Visualize the test cases before the test suite
				generation.
\end{itemize}
\subsubsection*{Scenario SC95}
\begin{tabular}{p{1in}p{4in}}
{\bf Description:} & Visualizing Test Cases. \\
{\bf From steps:} & UC_110#2M \\
{\bf To steps:} & END \\
\end{tabular}
 
\begin{tabular}{|p{0.8in}|p{1.6in}|p{1.6in}|p{1.6in}|}
\hline
Id & User Action & Condition & System Response  \bl 
1M & Go to the "Test Cases" tab in the "On The Fly Generation
						Editor" view. & No filter is selected. & The "Common Test Cases" and the "New Test Cases" boxes
						are checked. [FR_TARGET_0235]  \bl 
2M & Verify the test cases in the "Generated Test Cases" list
					 & - & All test cases appear in the "Generated Test Cases" list.
					 \bl 
\end{tabular}
\subsubsection*{Scenario SC96}
\begin{tabular}{p{1in}p{4in}}
{\bf Description:} & Visualizing the "Common Test Cases". \\
{\bf From steps:} & UC_110#2M \\
{\bf To steps:} & END \\
\end{tabular}
 
\begin{tabular}{|p{0.8in}|p{1.6in}|p{1.6in}|p{1.6in}|}
\hline
Id & User Action & Condition & System Response  \bl 
1A & Filter the test suite selecting some requirements in the
						"Selection" tab.  & - & The filter(s) are selected. [FR_TARGET_0234] \bl 
2A & Go to the "Test Cases" tab in "On The Fly Generation
						Editor" view. & - & The "Common Test Cases" and the "New Test Cases" boxes
						are checked. [FR_TARGET_0235]  \bl 
3A & Deselect the "New Test Cases" box. & - & Just the "Common Test Cases" selected by the filters in
						the Selected tab appear in the "Generated Test Cases" list with
						black font. [FR_TARGET_0237] \bl 
\end{tabular}
\subsubsection*{Scenario SC97}
\begin{tabular}{p{1in}p{4in}}
{\bf Description:} & Visualizing the "Removed Test Cases" and the "Common
					Test Cases". \\
{\bf From steps:} & UC_111#2M \\
{\bf To steps:} & END \\
\end{tabular}
 
\begin{tabular}{|p{0.8in}|p{1.6in}|p{1.6in}|p{1.6in}|}
\hline
Id & User Action & Condition & System Response  \bl 
1B & Select the "Removed Test Cases" box. & - & The removed test cases are added to the "Generated Test
						Cases" list with a red font. \bl 
\end{tabular}
\subsubsection*{Scenario SC98}
\begin{tabular}{p{1in}p{4in}}
{\bf Description:} & Visualizing the "Removed Test Cases". \\
{\bf From steps:} & 1B \\
{\bf To steps:} & END \\
\end{tabular}
 
\begin{tabular}{|p{0.8in}|p{1.6in}|p{1.6in}|p{1.6in}|}
\hline
Id & User Action & Condition & System Response  \bl 
1C & Deselect "Common Test Cases" box. & - & Just the test cases removed by the filters appear in the
						"Generated Test Cases" list with a red font. \bl 
\end{tabular}
\subsubsection*{Scenario SC99}
\begin{tabular}{p{1in}p{4in}}
{\bf Description:} & Visualizing the lists of "Common Test Cases" and "New
					test Cases" after saving a filter. \\
{\bf From steps:} & UC_118#1M \\
{\bf To steps:} & END \\
\end{tabular}
 
\begin{tabular}{|p{0.8in}|p{1.6in}|p{1.6in}|p{1.6in}|}
\hline
Id & User Action & Condition & System Response  \bl 
1D & Select a filter configuration saved previously. & - & The filter configuration is selected.  \bl 
2D & Go to "Selection" tab and select some requirements that are
						equal to saved filter and some requirements different from saved
						filter. & - & The filter(s) are selected. [FR_TARGET_0234] \bl 
3D & Go to the "Test Cases" tab in the "On The Fly Generation
						Editor" view. & - & The "Common Test Cases" and "New Test Cases" boxes are
						checked. [FR_TARGET_0235]  \bl 
5D & Go to "Generated Test Cases" list.  & - & The common test cases between the first selected filter
						and the second selected filter that is loaded in the "Selection"
						tab, appear with black font in the "Generated Test Cases" list.
						The New Test Cases appear with green font in the "Generated Test
						Cases" list.  \bl 
\end{tabular}
\subsubsection*{Scenario SC100}
\begin{tabular}{p{1in}p{4in}}
{\bf Description:} & Visualizing the "Removed Test Cases" after saving a
					filter \\
{\bf From steps:} & UC_118#1A \\
{\bf To steps:} & END \\
\end{tabular}
 
\begin{tabular}{|p{0.8in}|p{1.6in}|p{1.6in}|p{1.6in}|}
\hline
Id & User Action & Condition & System Response  \bl 
1E & Click on the "Load Former Filter Selection" button.
					 & - & The new filter configuration is loaded and the
						correspondent test cases appear in the "Generated Test Cases
						"list. [FR_TARGET_0237, FR_TARGET_0236] \bl 
2E & Select the default filter configuration "All test Cases".
					 & - & The filter configuration is selected. The common test
						cases between the selected filter and the filter that is loaded in
						the "Selection" tab appear with black font in the "Generated Test
						Cases" list.  \bl 
3E & Select the "Removed Test Cases" box. & - & The removed test cases appear in the "Generated Test
						Cases" list with a red font. \bl 
\end{tabular}
\subsubsection*{Scenario SC101}
\begin{tabular}{p{1in}p{4in}}
{\bf Description:} & Deselecting Common Test Cases box. \\
{\bf From steps:} & 2M \\
{\bf To steps:} & END \\
\end{tabular}
 
\begin{tabular}{|p{0.8in}|p{1.6in}|p{1.6in}|p{1.6in}|}
\hline
Id & User Action & Condition & System Response  \bl 
1F & Deselect the "Common Test Cases" box. & - & No test cases appear in the "Generated Test Cases" list.
					 \bl 
\end{tabular}
\subsection*{Use case UC_118}
\begin{itemize}
\item {\bf Name: }Saving a filter configuration in the Test Cases Viewer
\item {\bf Description: }Saving and loading a filter configuration in the test
				cases viewer.
\end{itemize}
\subsubsection*{Scenario SC111}
\begin{tabular}{p{1in}p{4in}}
{\bf Description:} & Saving a filter configuration. \\
{\bf From steps:} & UC_117#2A \\
{\bf To steps:} & END \\
\end{tabular}
 
\begin{tabular}{|p{0.8in}|p{1.6in}|p{1.6in}|p{1.6in}|}
\hline
Id & User Action & Condition & System Response  \bl 
1M & Click on the "Save Current Filter Selection" button.
					 & - & The new filter configuration is saved and appears in the
						common list. [FR_TARGET_0236] \bl 
\end{tabular}
\subsubsection*{Scenario SC112}
\begin{tabular}{p{1in}p{4in}}
{\bf Description:} & Loading a saved filter configuration. \\
{\bf From steps:} & 1M \\
{\bf To steps:} & END \\
\end{tabular}
 
\begin{tabular}{|p{0.8in}|p{1.6in}|p{1.6in}|p{1.6in}|}
\hline
Id & User Action & Condition & System Response  \bl 
1A & Select a filter configuration saved previously. & - & The filter configuration is selected.  \bl 
2A & Click on the "Load Former Filter Selection" button.
					 & - & The new filter configuration is loaded and the
						correspondent test cases appear in the "Generated Test Cases
						"list. [FR_TARGET_0237, FR_TARGET_0236] \bl 
\end{tabular}
\subsubsection*{Scenario SC113}
\begin{tabular}{p{1in}p{4in}}
{\bf Description:} & Loading the default filter configuration. \\
{\bf From steps:} & 2A \\
{\bf To steps:} & END \\
\end{tabular}
 
\begin{tabular}{|p{0.8in}|p{1.6in}|p{1.6in}|p{1.6in}|}
\hline
Id & User Action & Condition & System Response  \bl 
1B & Select the default filter configuration "All test Cases".
					 & - & The filter configuration is selected. \bl 
2B & Click on the "Load Former Filter Selection" button.
					 & - & All test cases appear in the "Generated test cases" list.
					 \bl 
\end{tabular}
\subsection*{Use case UC_119}
\begin{itemize}
\item {\bf Name: }Visualizing Test Cases in table format in Test Cases Viewer
			
\item {\bf Description: }Visualizing test cases in spreadsheet format.
			
\end{itemize}
\subsubsection*{Scenario SC114}
\begin{tabular}{p{1in}p{4in}}
{\bf Description:} & View a test case in the table. \\
{\bf From steps:} & UC_117#2M \\
{\bf To steps:} & END \\
\end{tabular}
 
\begin{tabular}{|p{0.8in}|p{1.6in}|p{1.6in}|p{1.6in}|}
\hline
Id & User Action & Condition & System Response  \bl 
1M & Click on any test case in the "Generated Test Cases" list.
					 & - & The chosen test case appears in the "Selected test Cases"
						view in the spreadsheet template. [FR_TARGET_0237] \bl 
\end{tabular}
\subsubsection*{Scenario SC115}
\begin{tabular}{p{1in}p{4in}}
{\bf Description:} & Select more than one test case to view in the table
					format. \\
{\bf From steps:} & UC_117#2M \\
{\bf To steps:} & END \\
\end{tabular}
 
\begin{tabular}{|p{0.8in}|p{1.6in}|p{1.6in}|p{1.6in}|}
\hline
Id & User Action & Condition & System Response  \bl 
1A & Click on more than one (using CTRL) test case in the
						"Generated Test Cases" list. & - & The first test case selected in crescent order that
						appears on list is shown in the "Selected test Cases" view in the
						spreadsheet standard. [FR_TARGET_0237] \bl 
\end{tabular}
\subsubsection*{Scenario SC116}
\begin{tabular}{p{1in}p{4in}}
{\bf Description:} & Deselect a test case. \\
{\bf From steps:} & 1A \\
{\bf To steps:} & END \\
\end{tabular}
 
\begin{tabular}{|p{0.8in}|p{1.6in}|p{1.6in}|p{1.6in}|}
\hline
Id & User Action & Condition & System Response  \bl 
1B & Deselect the first test case selected (using CTRL) in the
						"Generated Test Cases" list. & - & The spreadsheet view is updated, displaying another
						selected test case that appears first in he list in the "Generated
						test Cases" list.  \bl 
\end{tabular}
\subsection*{Use case UC_121}
\begin{itemize}
\item {\bf Name: }Visualizing Test Cases in the Traceability Matrixes
\item {\bf Description: }Visualizing test cases in the traceability matrixes.
			
\end{itemize}
\subsubsection*{Scenario SC117}
\begin{tabular}{p{1in}p{4in}}
{\bf Description:} & Visualizing Test Cases. \\
{\bf From steps:} & UC_110#2M \\
{\bf To steps:} & END \\
\end{tabular}
 
\begin{tabular}{|p{0.8in}|p{1.6in}|p{1.6in}|p{1.6in}|}
\hline
Id & User Action & Condition & System Response  \bl 
1M & Go to the "Traceability Matrix" tab in the "On The Fly
						Generation Editor" view. & No filter is selected & The "Common Test Cases" and "New Test Cases" boxes are
						checked. [FR_TARGET_0235]  \bl 
2M & Verify the test cases in the matrixes list & - & All test cases appear in the matrixes list. \bl 
\end{tabular}
\subsubsection*{Scenario SC118}
\begin{tabular}{p{1in}p{4in}}
{\bf Description:} & Visualizing the "Common Test Cases". \\
{\bf From steps:} & UC_110#2M \\
{\bf To steps:} & END \\
\end{tabular}
 
\begin{tabular}{|p{0.8in}|p{1.6in}|p{1.6in}|p{1.6in}|}
\hline
Id & User Action & Condition & System Response  \bl 
1A & Filter the test suite selecting some requirements in the
						"Test Selection Page". & - & The filter(s) are selected. [FR_TARGET_0234] \bl 
2A & Go to the "Traceability Matrix" tab in the "On The Fly
						Generation Editor" view. & - & The "Common Test Cases" and "New Test Cases" boxes are
						checked. [FR_TARGET_0235]  \bl 
3A & Deselect the "New Test Cases" box. & - & Just the "Common Test Cases" selected by the filters in
						the Selected tab appear in the matrixes list with black font.
						[FR_TARGET_0237] \bl 
\end{tabular}
\subsubsection*{Scenario SC119}
\begin{tabular}{p{1in}p{4in}}
{\bf Description:} & Visualizing the "Removed Test Cases" and "Common Test
					Cases". \\
{\bf From steps:} & UC_111#1M \\
{\bf To steps:} & END \\
\end{tabular}
 
\begin{tabular}{|p{0.8in}|p{1.6in}|p{1.6in}|p{1.6in}|}
\hline
Id & User Action & Condition & System Response  \bl 
1B & Go to the "Traceability Matrix" tab in the "On The Fly
						Generation Editor" view. & - & Verify that test cases in the "Traceability Matrix " list
						are according to the selected filter. [FR_TARGET_0240]  \bl 
2B & Select the "Removed Test Cases" box. & - & The removed test cases are added to the matrixes list
						with a red font. \bl 
\end{tabular}
\subsubsection*{Scenario SC120}
\begin{tabular}{p{1in}p{4in}}
{\bf Description:} & Visualizing the "Removed Test Cases". \\
{\bf From steps:} & 2B \\
{\bf To steps:} & END \\
\end{tabular}
 
\begin{tabular}{|p{0.8in}|p{1.6in}|p{1.6in}|p{1.6in}|}
\hline
Id & User Action & Condition & System Response  \bl 
1C & Deselect "Common Test Cases" box. & - & Just the test cases removed by the filters appear in the
						matrixes list with a red font. \bl 
\end{tabular}
\subsubsection*{Scenario SC121}
\begin{tabular}{p{1in}p{4in}}
{\bf Description:} & Visualizing the list of "Common Test Cases" and "New
					test Cases" after saving a filter. \\
{\bf From steps:} & 2A \\
{\bf To steps:} & END \\
\end{tabular}
 
\begin{tabular}{|p{0.8in}|p{1.6in}|p{1.6in}|p{1.6in}|}
\hline
Id & User Action & Condition & System Response  \bl 
1D & Click on the "Save Current Filter Selection". & - & The new filter configuration is saved and appears in the
						common list. [FR_TARGET_0236] \bl 
2D & Select a filter configuration saved previously. & - & The filter configuration is selected.  \bl 
3D & Go to "Test Selection Page" and select some requirements
						that are equal to the first selected filter and some requirements
						different from first selected filter. & - & The second filter(s) are selected. [FR_TARGET_0234]
					 \bl 
4D & Go to the "Traceability Matrix" tab in the "On The Fly
						Generation Editor" view. & - & The "Common Test Cases" and "New Test Cases" boxes are
						checked. [FR_TARGET_0235]  \bl 
5D & Verify the matrixes list.  & - & The common test cases between the first selected filter
						and the second selected filter that is loaded in the "Selection"
						tab, appear with black font in the matrixes list. The New Test
						Cases appear with green font in the matrixes list. \bl 
\end{tabular}
\subsubsection*{Scenario SC122}
\begin{tabular}{p{1in}p{4in}}
{\bf Description:} & Visualizing the "Removed Test Cases" after saving a
					filter \\
{\bf From steps:} & 2D \\
{\bf To steps:} & END \\
\end{tabular}
 
\begin{tabular}{|p{0.8in}|p{1.6in}|p{1.6in}|p{1.6in}|}
\hline
Id & User Action & Condition & System Response  \bl 
1E & Click on the "Load Former Filter Selection" button.
					 & - & The new filter configuration is loaded and the
						correspondent test cases appear in the matrixes list.
						[FR_TARGET_0237, FR_TARGET_0236] \bl 
2E & Select the default filter configuration "All test Cases".
					 & - & The filter configuration is selected. The common test
						cases between the selected filter and the filter that is loaded in
						the "Selection" tab, appear with black font in the matrixes list.
					 \bl 
3E & Select the "Removed Test Cases" box. & - & The removed test cases are added to the matrixes list
						with a red font. \bl 
\end{tabular}
\subsubsection*{Scenario SC123}
\begin{tabular}{p{1in}p{4in}}
{\bf Description:} & Deselecting Common Test Cases box. \\
{\bf From steps:} & 2M \\
{\bf To steps:} & END \\
\end{tabular}
 
\begin{tabular}{|p{0.8in}|p{1.6in}|p{1.6in}|p{1.6in}|}
\hline
Id & User Action & Condition & System Response  \bl 
1F & Deselect the "Common Test Cases" box. & - & No test cases appear in the matrixes list. \bl 
\end{tabular}
\subsection*{Use case UC_122}
\begin{itemize}
\item {\bf Name: }Saving a filter configuration in the Traceability Matrixes
			
\item {\bf Description: }Saving and loading a filter configuration.
\end{itemize}
\subsubsection*{Scenario SC124}
\begin{tabular}{p{1in}p{4in}}
{\bf Description:} & Saving a filter configuration in the traceability
					matrixes. \\
{\bf From steps:} & UC_121#2A \\
{\bf To steps:} & END \\
\end{tabular}
 
\begin{tabular}{|p{0.8in}|p{1.6in}|p{1.6in}|p{1.6in}|}
\hline
Id & User Action & Condition & System Response  \bl 
1M & Click on the "Save Current Filter Selection". & - & The new filter configuration is saved and appears in the
						common list. \bl 
\end{tabular}
\subsubsection*{Scenario SC125}
\begin{tabular}{p{1in}p{4in}}
{\bf Description:} & Loading a saved filter configuration in the
					traceability matrixes. \\
{\bf From steps:} & 1M \\
{\bf To steps:} & END \\
\end{tabular}
 
\begin{tabular}{|p{0.8in}|p{1.6in}|p{1.6in}|p{1.6in}|}
\hline
Id & User Action & Condition & System Response  \bl 
1A & Select a filter configuration saved previously. & - & The filter configuration is selected. \bl 
2A & Click on the "Load Former Filter Selection" button.
					 & - & The new filter configuration is loaded and the
						correspondent test cases appear in the matrixes. \bl 
\end{tabular}
\subsubsection*{Scenario SC126}
\begin{tabular}{p{1in}p{4in}}
{\bf Description:} & Loading the default filter configuration. \\
{\bf From steps:} & 2A \\
{\bf To steps:} & END \\
\end{tabular}
 
\begin{tabular}{|p{0.8in}|p{1.6in}|p{1.6in}|p{1.6in}|}
\hline
Id & User Action & Condition & System Response  \bl 
1B & Select the default filter configuration "All test Cases".
					 & - & The filter configuration is selected. \bl 
2B & Click on the "Load Former Filter Selection" button.
					 & - & All test cases appear in the matrixes. \bl 
\end{tabular}
\subsection*{Use case UC_123}
\begin{itemize}
\item {\bf Name: }Visualizing Test Cases in table format in the Traceability
				Matrixes
\item {\bf Description: }Viewing test cases in the spreadsheet format.
			
\end{itemize}
\subsubsection*{Scenario SC127}
\begin{tabular}{p{1in}p{4in}}
{\bf Description:} & View a test case in the spreadsheet format.
				 \\
{\bf From steps:} & UC_121#2M \\
{\bf To steps:} & END \\
\end{tabular}
 
\begin{tabular}{|p{0.8in}|p{1.6in}|p{1.6in}|p{1.6in}|}
\hline
Id & User Action & Condition & System Response  \bl 
1M & Click in any test case in the matrixes list. & - & The details of the selected test case appear in the
						"Selected test Cases" view in the spreadsheet standard. \bl 
\end{tabular}
\subsection*{Use case UC_124}
\begin{itemize}
\item {\bf Name: }Including Test Cases Permanently in test suite
\item {\bf Description: }Including test cases permanently in the test suit,
				independently of filters selection.
\end{itemize}
\subsubsection*{Scenario SC128}
\begin{tabular}{p{1in}p{4in}}
{\bf Description:} & Include permanently a test case that is already
					included by the filter selection. \\
{\bf From steps:} & START \\
{\bf To steps:} & END \\
\end{tabular}
 
\begin{tabular}{|p{0.8in}|p{1.6in}|p{1.6in}|p{1.6in}|}
\hline
Id & User Action & Condition & System Response  \bl 
1M & Go to the "Test Cases" tab in the "On The Fly Generation
						Editor" view. & Any filter has to be selected. [FR_TARGET_0234]
					 & The selected test cases by filter are displayed in the
						"Generated Test Cases" list. [FR_TARGET_0237] \bl 
2M & Check the "Removed Test Cases" box. & - & The removed test cases by filter are displayed in the
						"Generated Test Cases" list with a red font. \bl 
3M & Click with the right mouse button in a included test case.
					 & - & A drop down menu is displayed with the "Exclude Test Case
						Permanently" and "Include Test Case Permanently" options are
						available and "Cancel Test Case Exclusion" and "Cancel Test Case
						Inclusion" are unavailable.  \bl 
4M & Select "Include Test Case Permanently" option. & - & The test case appears "Included Test Cases" list of
						Inclusion/Exclusion tab and remains with a black font in the
						"Generated Test Case" list. [FR_TARGET_0241] \bl 
\end{tabular}
\subsubsection*{Scenario SC129}
\begin{tabular}{p{1in}p{4in}}
{\bf Description:} & Cancel the test case inclusion permanently.
				 \\
{\bf From steps:} & 4M \\
{\bf To steps:} & END \\
\end{tabular}
 
\begin{tabular}{|p{0.8in}|p{1.6in}|p{1.6in}|p{1.6in}|}
\hline
Id & User Action & Condition & System Response  \bl 
1A & Go back to the test case tab and click with the mouse right
						button in this included test case. & - & A drop down menu is displayed with the "Exclude Test Case
						Permanently" and "Cancel Test Case Inclusion" options are
						available and "Include Test Case Permanently" and "Cancel Test
						Case Exclusion" are unavailable. \bl 
2A & Select "Cancel Test Case Inclusion" option. & - & The test case is removed from "Included Test Cases" list
						of Inclusion/Exclusion tab. \bl 
\end{tabular}
\subsubsection*{Scenario SC130}
\begin{tabular}{p{1in}p{4in}}
{\bf Description:} & Exclude the test case included permanently.
				 \\
{\bf From steps:} & 4M \\
{\bf To steps:} & END \\
\end{tabular}
 
\begin{tabular}{|p{0.8in}|p{1.6in}|p{1.6in}|p{1.6in}|}
\hline
Id & User Action & Condition & System Response  \bl 
1B & Go back to the test case tab and click with the mouse right
						button in this included test case. & - & A drop down menu is displayed with the "Exclude Test Case
						Permanently" and "Cancel Test Case Inclusion" options are
						available and "Include Test Case Permanently" and "Cancel Test
						Case Exclusion" are unavailable. \bl 
2B & Select "Exclude Test Case Permanently" option. & - & The test case is removed from "Included Test Cases" list,
						is added to "Excluded Test Cases" list in the Inclusion/Exclusion
						tab and shown with a red font in the "Generated Test Cases list"
						in the Test Cases tab. \bl 
\end{tabular}
\subsubsection*{Scenario SC131}
\begin{tabular}{p{1in}p{4in}}
{\bf Description:} & Cancel the test case inclusion permanently in the
					Inclusion/Exclusion tab. \\
{\bf From steps:} & 4M \\
{\bf To steps:} & END \\
\end{tabular}
 
\begin{tabular}{|p{0.8in}|p{1.6in}|p{1.6in}|p{1.6in}|}
\hline
Id & User Action & Condition & System Response  \bl 
1C & Go to the Inclusion/Exclusion tab and click with the mouse
						right button in this included test case. & - & A drop down menu is displayed with the "Exclude Test Case
						Permanently" and "Cancel Test Case Inclusion" options are
						available.  \bl 
2C & Select "Cancel Test Case Inclusion" option. & - & The test case is removed from "Included Test Cases" list
						of Inclusion/Exclusion tab. \bl 
\end{tabular}
\subsubsection*{Scenario SC132}
\begin{tabular}{p{1in}p{4in}}
{\bf Description:} & Include a test case that is excluded by the filter
					selection. \\
{\bf From steps:} & 2M \\
{\bf To steps:} & END \\
\end{tabular}
 
\begin{tabular}{|p{0.8in}|p{1.6in}|p{1.6in}|p{1.6in}|}
\hline
Id & User Action & Condition & System Response  \bl 
1D & Click with the mouse right button in a excluded test case.
					 & - & A drop down menu is displayed with the "Exclude Test Case
						Permanently" and "Include Test Case Permanently" options are
						available and "Cancel Test Case Exclusion" and "Cancel Test Case
						Inclusion" are unavailable. [FR_TARGET_0241] \bl 
2D & Select "Include Test Case Permanently" option. & - & The test case appears in the "Included Test Cases" list
						of Inclusion/Exclusion tab and is shown with a black font in the
						"Generated Test Case" list. \bl 
\end{tabular}
\subsection*{Use case UC_126}
\begin{itemize}
\item {\bf Name: }Excluding Test Cases Permanently in test suite
\item {\bf Description: }Excluding test cases permanently in the test suit,
				independently of filters selection.
\end{itemize}
\subsubsection*{Scenario SC133}
\begin{tabular}{p{1in}p{4in}}
{\bf Description:} & Exclude permanently a test case that is already
					excluded by the filter selection. \\
{\bf From steps:} & START \\
{\bf To steps:} & END \\
\end{tabular}
 
\begin{tabular}{|p{0.8in}|p{1.6in}|p{1.6in}|p{1.6in}|}
\hline
Id & User Action & Condition & System Response  \bl 
1M & Go to the "Test Cases" tab in the "On The Fly Generation
						Editor" view. & Any filter has to be selected. [FR_TARGET_0234]
					 & The selected test cases by filter are displayed in the
						"Generated Test Cases" list. [FR_TARGET_0237] \bl 
2M & Check the "Removed Test Cases" box. & - & The removed test cases by filter are displayed in the
						"Generated Test Cases" list with a red font. \bl 
3M & Click with the mouse right button in a excluded test case.
					 & - & A drop down menu is displayed with the "Exclude Test Case
						Permanently" and "Include Test Case Permanently" options are
						available and "Cancel Test Case Exclusion" and "Cancel Test Case
						Inclusion" are unavailable. \bl 
4M & Select "Exclude Test Case Permanently" option. & - & The test case appears "Excluded Test Cases" list of
						Inclusion/Exclusion tab and remains with a red font in the
						"Generated Test Case" list. [FR_TARGET_0241] \bl 
\end{tabular}
\subsubsection*{Scenario SC134}
\begin{tabular}{p{1in}p{4in}}
{\bf Description:} & Cancel the test case exclusion permanently.
				 \\
{\bf From steps:} & 4M \\
{\bf To steps:} & END \\
\end{tabular}
 
\begin{tabular}{|p{0.8in}|p{1.6in}|p{1.6in}|p{1.6in}|}
\hline
Id & User Action & Condition & System Response  \bl 
1A & Go back to the test case tab and click with the mouse right
						button in this excluded test case. & - & A drop down menu is displayed with the "Include Test Case
						Permanently" and "Cancel Test Case Exclusion" options are
						available and "Exclude Test Case Permanently" and "Cancel Test
						Case Inclusion" are unavailable. \bl 
2A & Select "Cancel Test Case Exclusion" option. & - & The test case is removed from "Excluded Test Cases" list
						of Inclusion/Exclusion tab. \bl 
\end{tabular}
\subsubsection*{Scenario SC135}
\begin{tabular}{p{1in}p{4in}}
{\bf Description:} & Exclude the test case included permanently.
				 \\
{\bf From steps:} & 4M \\
{\bf To steps:} & END \\
\end{tabular}
 
\begin{tabular}{|p{0.8in}|p{1.6in}|p{1.6in}|p{1.6in}|}
\hline
Id & User Action & Condition & System Response  \bl 
1B & Go back to the test case tab and click with the mouse right
						button in this excluded test case. & - & A drop down menu is displayed with the "Include Test Case
						Permanently" and "Cancel Test Case Exclusion" options are
						available and "Exclude Test Case Permanently" and "Cancel Test
						Case Inclusion" are unavailable. \bl 
2B & Select "Include Test Case Permanently" option. & - & The test case is removed from "Excluded Test Cases" list,
						is added to "Included Test Cases" list in the Inclusion/Exclusion
						tab and shown with a black font in the "Generated Test Cases list"
						in the Test Cases tab. \bl 
\end{tabular}
\subsubsection*{Scenario SC136}
\begin{tabular}{p{1in}p{4in}}
{\bf Description:} & Cancel the test case exclusion permanently in the
					Inclusion/Exclusion tab. \\
{\bf From steps:} & 4M \\
{\bf To steps:} & END \\
\end{tabular}
 
\begin{tabular}{|p{0.8in}|p{1.6in}|p{1.6in}|p{1.6in}|}
\hline
Id & User Action & Condition & System Response  \bl 
1C & Go to the Inclusion/Exclusion tab and click with the mouse
						right button in this excluded test case. & - & A drop down menu is displayed with the "Include Test Case
						Permanently" and "Cancel Test Case Exclusion" options are
						available. \bl 
2C & Select "Cancel Test Case Exclusion" option. & - & The test case is removed from "Included Test Cases" list
						of Inclusion/Exclusion tab. \bl 
\end{tabular}
\subsubsection*{Scenario SC137}
\begin{tabular}{p{1in}p{4in}}
{\bf Description:} & Exclude a test case that is included by the filter
					selection. \\
{\bf From steps:} & 4M \\
{\bf To steps:} & END \\
\end{tabular}
 
\begin{tabular}{|p{0.8in}|p{1.6in}|p{1.6in}|p{1.6in}|}
\hline
Id & User Action & Condition & System Response  \bl 
1D & Click with the mouse right button in a included test case.
					 & - & A drop down menu is displayed with the "Exclude Test Case
						Permanently" and "Include Test Case Permanently" options are
						available and "Cancel Test Case Exclusion" and "Cancel Test Case
						Inclusion" are unavailable. \bl 
2D & Select "Exclude Test Case Permanently" option. & - & The test case appears in the "Excluded Test Cases" list
						of Inclusion/Exclusion tab and is shown with a red font in the
						"Generated Test Case" list. \bl 
\end{tabular}
\subsection*{Use case UC_145}
\begin{itemize}
\item {\bf Name: }Starting Consistency Management
\item {\bf Description: }Starting Consistency Management Editor.
\end{itemize}
\subsubsection*{Scenario SC151}
\begin{tabular}{p{1in}p{4in}}
{\bf Description:} & Starting Consistency Management Editor. \\
{\bf From steps:} & UC_110#2M \\
{\bf To steps:} & END \\
\end{tabular}
 
\begin{tabular}{|p{0.8in}|p{1.6in}|p{1.6in}|p{1.6in}|}
\hline
Id & User Action & Condition & System Response  \bl 
1M & Go to "Test Cases" tab and mark the option "Start
						Consistency Managememnt before saving." & The On The Fly Editor is Opened. & A window is displayed asking the user to select a
						generated test suite to compare. [FR_TARGET_245]. \bl 
2M & Select a test suite and press the "Finish" button. & - & The Consistency Management Editor is showed and the
						system displays one table containing the new test cases generated
						by the On The Fly and another table containing the percentage of
						similarity with the test cases of the previous selected test
						suite. [FR_TARGET_244]The system automatically associates a new
						test case to an old one that has a similarity level of 100%.
						[FR_TARGET_246] \bl 
\end{tabular}
\subsubsection*{Scenario SC152}
\begin{tabular}{p{1in}p{4in}}
{\bf Description:} & No test suites available. \\
{\bf From steps:} & UC_110#2M \\
{\bf To steps:} & END \\
\end{tabular}
 
\begin{tabular}{|p{0.8in}|p{1.6in}|p{1.6in}|p{1.6in}|}
\hline
Id & User Action & Condition & System Response  \bl 
1A & Go to "Test Cases" tab and mark the option "Start
						Consistency Managememnt before saving." & The On The Fly Editor is Opened. There is no test suite
						in the project folder. & A window is displayed asking the user to select a test
						suite to compare, but none test suite is displayed in the list and
						a message is shown informing that there is no test suites
						available. The "Finish" button is disabled.  \bl 
2A & Press the "Cancel" button. & - & The window is closed and the On The Fly Editor is
						displayed. \bl 
\end{tabular}
\subsection*{Use case UC_146}
\begin{itemize}
\item {\bf Name: }Associating Test Cases
\item {\bf Description: }Associating similar test cases in the Consistency
				Management Editor.
\end{itemize}
\subsubsection*{Scenario SC153}
\begin{tabular}{p{1in}p{4in}}
{\bf Description:} & Associating similar test cases in the Consistency
					Management Editor. \\
{\bf From steps:} & UC_145#2M \\
{\bf To steps:} & END \\
\end{tabular}
 
\begin{tabular}{|p{0.8in}|p{1.6in}|p{1.6in}|p{1.6in}|}
\hline
Id & User Action & Condition & System Response  \bl 
1M & Select one test case from the new test cases table.
					 & There are no modifications between the new and the old
						test suites. & The system shows in the old test cases table the
						similarity percentage between the selected new test case and the
						old ones. The "Delete" and "New Id" options are also displayed.
					 \bl 
2M & Select one test case from the old test case table. & - & The system displays two spreadsheets showing the fields
						of the selected test cases in the "Compare Test Cases" area. The
						system marks with blue color the different fields and with yellow
						color the extra steps. \bl 
3M & Press the "Save Test Suite" button. & - & The system displays a message informing that the test
						cases were merged and a new test suite has been generated.
					 \bl 
\end{tabular}
\subsubsection*{Scenario SC154}
\begin{tabular}{p{1in}p{4in}}
{\bf Description:} & A new test case has been created. \\
{\bf From steps:} & UC_145#2M \\
{\bf To steps:} & END \\
\end{tabular}
 
\begin{tabular}{|p{0.8in}|p{1.6in}|p{1.6in}|p{1.6in}|}
\hline
Id & User Action & Condition & System Response  \bl 
1A & Select one test case from the New Test Cases table.
					 & There is at least one modification between the new test
						suite and the old test suite.  & The system marks the new or modified test case with red
						color. The system shows in the old test cases table the similarity
						levels between the selected new test case and the old ones
						different from 100%. The "Delete" and "New Id" options are also
						displayed and the "New Id" option is checked. \bl 
2A & Press "Save Test Suite" button. & - & The system displays a message informing that the test
						cases were merged and a new test suite has been generated. The
						generated test suite must contain the new test case and the new id
						that has been attributed to it. \bl 
\end{tabular}
\subsubsection*{Scenario SC155}
\begin{tabular}{p{1in}p{4in}}
{\bf Description:} & Removing a test case. \\
{\bf From steps:} & 1M \\
{\bf To steps:} & END \\
\end{tabular}
 
\begin{tabular}{|p{0.8in}|p{1.6in}|p{1.6in}|p{1.6in}|}
\hline
Id & User Action & Condition & System Response  \bl 
1B & Check the "Delete" option in the old test case table and
						press the "Save Test Suite" button. & - & The system informs that the selected test case will be
						removed from the test suite and asks if the user wants to proceed.
					 \bl 
2B & Press the "OK" button. & - & The system displays a message informing that the test
						cases were merged and a new test suite has been generated. The
						generated test suite must not contain the deleted test case.
					 \bl 
\end{tabular}
\subsubsection*{Scenario SC156}
\begin{tabular}{p{1in}p{4in}}
{\bf Description:} & Associate two new test cases to the same old test case
				 \\
{\bf From steps:} & UC_145#2M \\
{\bf To steps:} & END \\
\end{tabular}
 
\begin{tabular}{|p{0.8in}|p{1.6in}|p{1.6in}|p{1.6in}|}
\hline
Id & User Action & Condition & System Response  \bl 
2C & Associate two test cases from "New Test Cases" table to the
						same test case in the "Old Test Case" table. & - & The system shows a warning message informing that there
						is another test case already associated with the selected old one.
					 \bl 
3C & Press the "OK" button and press the "Save Test Suite"
						option. & - & The system displays a message informing that the test
						cases were merged and a new test suite has been generated. The
						generated test suite must contain two test cases with the same id.
					 \bl 
\end{tabular}
\subsection*{Use case UC_147}
\begin{itemize}
\item {\bf Name: }Setting up Next New Id
\item {\bf Description: }Configure the next new id to new test case.
			
\end{itemize}
\subsubsection*{Scenario SC157}
\begin{tabular}{p{1in}p{4in}}
{\bf Description:} & Associating similar test cases in the Consistency
					Management Editor. \\
{\bf From steps:} & UC_146#2A \\
{\bf To steps:} & END \\
\end{tabular}
 
\begin{tabular}{|p{0.8in}|p{1.6in}|p{1.6in}|p{1.6in}|}
\hline
Id & User Action & Condition & System Response  \bl 
1M & In the "Preferences" area, press the "Change" button.
					 & - & The system enables the "Next New Id" field for edition.
					 \bl 
2M & Fill the "Next New Id" field with a desired value and
						presses the "Save" button. & - & The system disables the "Next New Id" field for edition
						and save the new filled value. \bl 
3M & Press the "Save Test Suite" option. & - & The system displays a message informing that the test
						cases were merged and a new test suite has been generated. The
						generated test suite must contain the test case marked as new with
						the configured id.  \bl 
\end{tabular}
\subsection*{Use case UC_148}
\begin{itemize}
\item {\bf Name: }Changing compared Test Suite
\item {\bf Description: }Change the test suite file that is being compared to the
				New test Suite generated by On The Fly.
\end{itemize}
\subsubsection*{Scenario SC158}
\begin{tabular}{p{1in}p{4in}}
{\bf Description:} & Changing Test Suite file. \\
{\bf From steps:} & UC_145#2M \\
{\bf To steps:} & END \\
\end{tabular}
 
\begin{tabular}{|p{0.8in}|p{1.6in}|p{1.6in}|p{1.6in}|}
\hline
Id & User Action & Condition & System Response  \bl 
1M & Above Old Test Cases table, press the "Change" button.
					 & There are one more generated test suites different from
						suite generated by On the Fly (New Test suite). & The system shows a window with all saved test suites and
						ask to user select a test suite to compare. \bl 
2M & Choose a test suite different from current. & - & The system shows a progress bar indicating the test suite
						loading process and opens another Consistency Management window
						comparing the selected Test suite file. \bl 
\end{tabular}
\subsection*{Use case UC_149}
\begin{itemize}
\item {\bf Name: }Automatic Association configuration
\item {\bf Description: }Configuring Automatic Association.
\end{itemize}
\subsubsection*{Scenario SC159}
\begin{tabular}{p{1in}p{4in}}
{\bf Description:} & Changing Test Suite file. \\
{\bf From steps:} & UC_146#1A \\
{\bf To steps:} & END \\
\end{tabular}
 
\begin{tabular}{|p{0.8in}|p{1.6in}|p{1.6in}|p{1.6in}|}
\hline
Id & User Action & Condition & System Response  \bl 
1M & In Automatic Association area, fill the "First most
						similar" with a value a little less than the similarity percentage
						of first test case most similar to new test case. & There is a new modified test that has one test much
						similar with it (similarity percentage value is high) and the
						others are very different from it (similarity percentage value is
						lower than the first most similar). & The First Most Similar percentage is changed. \bl 
2M & In Automatic Association area, fill the "Second most
						similar" with a value a higher than the similarity percentage of
						second test case most similar to new test case. & - & The Second Most Similar percentage is changed. \bl 
3M & Press "Refresh" button & - & The first most similar test case is automatically checked
						in the Old Test Cases table. \bl 
\end{tabular}
\end{document}