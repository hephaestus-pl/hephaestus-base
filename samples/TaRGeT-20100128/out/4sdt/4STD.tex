%This Latex file is machine-generated by the Hephaestus

\documentclass[a4paper,11pt]{article}
\newcommand{\bl}{\\ \hline}
\title{Use Case Model Generated from TaRGeT}
\begin{document}
\maketitle
\section*{Use cases}
\subsection*{Use case UC-05}
\begin{itemize}
\item {\bf Name: }Creating a Project from Blank Work Area
\item {\bf Description: }Create a TaRGeT project.
\end{itemize}
\subsubsection*{Scenario UC-05-SC-1}
\begin{tabular}{p{1in}p{4in}}
{\bf Description:} & Create a new project. \\
{\bf From steps:} & START \\
{\bf To steps:} & END \\
\end{tabular}
 
\begin{tabular}{|p{0.8in}|p{1.6in}|p{1.6in}|p{1.6in}|}
\hline
Id & User Action & Condition & System Response  \bl 
1M & Go to "File" option in the menu bar. & The TaRGeT is already started up. No project is opened. & The "New Project" option is enabled in the drop down menu. The Close Project option is disabled. \bl 
2M & Select the "New Project" option. [FR-TARGET-0020] & - & The "New Project" screen is displayed. The default project name is displayed in the "Project Name" field. The default project location is displayed in the "Destination Folder" field. \bl 
3M & Click on "Finish" button. & - & A progress bar is displayed. A new project work area is displayed. The project folder is created in the specified location. [FR-TARGET-0110, FR-TARGET-0010] \bl 
\end{tabular}
\subsubsection*{Scenario UC-05-SC-2}
\begin{tabular}{p{1in}p{4in}}
{\bf Description:} & Type a valid project name. \\
{\bf From steps:} & 2M \\
{\bf To steps:} & 3M \\
\end{tabular}
 
\begin{tabular}{|p{0.8in}|p{1.6in}|p{1.6in}|p{1.6in}|}
\hline
Id & User Action & Condition & System Response  \bl 
1A & Type a valid project name. & - & The "Project Name" field is filled. [FR-TARGET-0025] \bl 
\end{tabular}
\subsubsection*{Scenario UC-05-SC-3}
\begin{tabular}{p{1in}p{4in}}
{\bf Description:} & Cancel the project creation. \\
{\bf From steps:} & 2M,1A,1C \\
{\bf To steps:} & END \\
\end{tabular}
 
\begin{tabular}{|p{0.8in}|p{1.6in}|p{1.6in}|p{1.6in}|}
\hline
Id & User Action & Condition & System Response  \bl 
1B & Click on the "Cancel" button. & - & The tool goes back to the window it was before the project creation. \bl 
\end{tabular}
\subsubsection*{Scenario UC-05-SC-4}
\begin{tabular}{p{1in}p{4in}}
{\bf Description:} & Browse destination folder. \\
{\bf From steps:} & 2M,1A \\
{\bf To steps:} & 3M \\
\end{tabular}
 
\begin{tabular}{|p{0.8in}|p{1.6in}|p{1.6in}|p{1.6in}|}
\hline
Id & User Action & Condition & System Response  \bl 
1C & Browse and select the location in which the project shall be created. & - & The "Destination Folder" field is filled. \bl 
\end{tabular}
\subsubsection*{Scenario UC-05-SC-5}
\begin{tabular}{p{1in}p{4in}}
{\bf Description:} & Create the project using shortcut. \\
{\bf From steps:} & START \\
{\bf To steps:} & END \\
\end{tabular}
 
\begin{tabular}{|p{0.8in}|p{1.6in}|p{1.6in}|p{1.6in}|}
\hline
Id & User Action & Condition & System Response  \bl 
1H & Press CTRL+N. & - & The "New Project" screen is displayed. The default project name is displayed in the "Project Name" field. The default project location is displayed in the "Destination Folder" field. [FR-TARGET-0020] \bl 
2H & Click on "Finish" button. & - & A progress bar is displayed. A new project work area is displayed. The project folder is created in the specified location. [FR-TARGET-0110, FR-TARGET-0010] \bl 
\end{tabular}
\subsubsection*{Scenario UC-05-SC-6}
\begin{tabular}{p{1in}p{4in}}
{\bf Description:} & Type an invalid project name. \\
{\bf From steps:} & 2M \\
{\bf To steps:} & END \\
\end{tabular}
 
\begin{tabular}{|p{0.8in}|p{1.6in}|p{1.6in}|p{1.6in}|}
\hline
Id & User Action & Condition & System Response  \bl 
1D & Type an invalid project name (e.g. with one of the following characters ":\/<>?*). & - & A message is displayed indicating that some characters are not allowed. The "Finish" button is disabled. [FR-TARGET-0025] \bl 
\end{tabular}
\subsubsection*{Scenario UC-05-SC-7}
\begin{tabular}{p{1in}p{4in}}
{\bf Description:} & No project name is typed. \\
{\bf From steps:} & 2M \\
{\bf To steps:} & END \\
\end{tabular}
 
\begin{tabular}{|p{0.8in}|p{1.6in}|p{1.6in}|p{1.6in}|}
\hline
Id & User Action & Condition & System Response  \bl 
1E & Type a blank project name. [FR-TARGET-0025] & - & A message is displayed indicating that a name must be specified. The "Finish" button is disabled. \bl 
\end{tabular}
\subsubsection*{Scenario UC-05-SC-8}
\begin{tabular}{p{1in}p{4in}}
{\bf Description:} & Project directory already exists. \\
{\bf From steps:} & 2M \\
{\bf To steps:} & END \\
\end{tabular}
 
\begin{tabular}{|p{0.8in}|p{1.6in}|p{1.6in}|p{1.6in}|}
\hline
Id & User Action & Condition & System Response  \bl 
1F & Type a valid project name. [FR-TARGET-0025] & - & The "Project Name" field is filled. \bl 
2F & Browse and select the location in which there is a directory with the same name of the project. & - & The "Destination Folder" field is filled. \bl 
3F & Click on "Finish" button. & - & A dialog is displayed indicating that the project folder already exists. The dialog asks if the user you want to proceed and erase all content of the folder. [FR-TARGET-0020] \bl 
4F & Click on "Yes" button. & - & All the folder contents is deleted and the new project is created. \bl 
\end{tabular}
\subsubsection*{Scenario UC-05-SC-9}
\begin{tabular}{p{1in}p{4in}}
{\bf Description:} & Do not delete all folder contents. \\
{\bf From steps:} & 3F \\
{\bf To steps:} & END \\
\end{tabular}
 
\begin{tabular}{|p{0.8in}|p{1.6in}|p{1.6in}|p{1.6in}|}
\hline
Id & User Action & Condition & System Response  \bl 
1G & Click on "No" button. & - & The focus goes back to the "New Project" window. \bl 
\end{tabular}
\subsection*{Use case UC-20}
\begin{itemize}
\item {\bf Name: }Opening an Existing Project from Blank Work Area
\item {\bf Description: }Open an existing project.
\end{itemize}
\subsubsection*{Scenario UC-20-SC-1}
\begin{tabular}{p{1in}p{4in}}
{\bf Description:} & Open an existing project. \\
{\bf From steps:} & START \\
{\bf To steps:} & END \\
\end{tabular}
 
\begin{tabular}{|p{0.8in}|p{1.6in}|p{1.6in}|p{1.6in}|}
\hline
Id & User Action & Condition & System Response  \bl 
1M & Go to "File" option in the menu bar. & TaRGeT is already started up. No project is opened. & The "Open Project" option is enabled. \bl 
2M & Select the "Open Project" option. [FR-TARGET-0040] & - & The "Open Project" window is displayed. No project is selected. The "Finish" button is disabled. \bl 
3M & Browse and select an existing project. & - & The "Project Name" field is now filled. The "Finish" button is enabled. \bl 
4M & Click on "Finish" button. & - & A progress bar is displayed indicating that the selected existing project is being opened. After, the existing project is opened. The project work area is displayed. [FR-TARGET-0110] \bl 
\end{tabular}
\subsubsection*{Scenario UC-20-SC-2}
\begin{tabular}{p{1in}p{4in}}
{\bf Description:} & Cancel the project opening. \\
{\bf From steps:} & 2M,3M \\
{\bf To steps:} & END \\
\end{tabular}
 
\begin{tabular}{|p{0.8in}|p{1.6in}|p{1.6in}|p{1.6in}|}
\hline
Id & User Action & Condition & System Response  \bl 
1A & Click on "Cancel" button. & - & The tool goes to the blank work area. No project is opened. [FR-TARGET-0225] \bl 
\end{tabular}
\subsubsection*{Scenario UC-20-SC-3}
\begin{tabular}{p{1in}p{4in}}
{\bf Description:} & Open using shortcut. \\
{\bf From steps:} & START \\
{\bf To steps:} & 3M \\
\end{tabular}
 
\begin{tabular}{|p{0.8in}|p{1.6in}|p{1.6in}|p{1.6in}|}
\hline
Id & User Action & Condition & System Response  \bl 
1B & Press CTRL+O. & - & The "Open Project" window is displayed. No project is selected. The "Finish" button is disabled. [FR-TARGET-0040] \bl 
\end{tabular}
\subsection*{Use case UC-25}
\begin{itemize}
\item {\bf Name: }Opening an Existing Project from an Already Opened Project
\item {\bf Description: }A project is already opened.
\end{itemize}
\subsubsection*{Scenario UC-25-SC-1}
\begin{tabular}{p{1in}p{4in}}
{\bf Description:} & Open an existing project from an already opened project. \\
{\bf From steps:} & START \\
{\bf To steps:} & UC-20-3M \\
\end{tabular}
 
\begin{tabular}{|p{0.8in}|p{1.6in}|p{1.6in}|p{1.6in}|}
\hline
Id & User Action & Condition & System Response  \bl 
1M & Go to "File" option in the menu bar. & TaRGeT is already started up. A project is already opened. & The "Open Project" option is enabled. \bl 
2M & Select the "Open Project" option. [FR-TARGET-0040] & - & A dialog box is displayed informing that the current project will be closed in order to open another one. \bl 
3M & Click on "Yes" button. & - & The "Open Project" screen is displayed. \bl 
\end{tabular}
\subsubsection*{Scenario UC-25-SC-2}
\begin{tabular}{p{1in}p{4in}}
{\bf Description:} & Cancel the project opening. \\
{\bf From steps:} & 2M \\
{\bf To steps:} & END \\
\end{tabular}
 
\begin{tabular}{|p{0.8in}|p{1.6in}|p{1.6in}|p{1.6in}|}
\hline
Id & User Action & Condition & System Response  \bl 
1A & Click on "No" button. & - & The dialog box is closed and the current project work area is displayed. \bl 
\end{tabular}
\subsection*{Use case UC-26}
\begin{itemize}
\item {\bf Name: }Refreshing a Project
\item {\bf Description: }Refresh a project.
\end{itemize}
\subsubsection*{Scenario UC-26-SC-1}
\begin{tabular}{p{1in}p{4in}}
{\bf Description:} & Refresh a project. \\
{\bf From steps:} & START \\
{\bf To steps:} & UC-90-1M \\
\end{tabular}
 
\begin{tabular}{|p{0.8in}|p{1.6in}|p{1.6in}|p{1.6in}|}
\hline
Id & User Action & Condition & System Response  \bl 
1M & Go to "Project" option in the menu bar. & - & A menu with two options shall appear:  "Refresh Automatically" and "Refresh Project". "Refresh Automatically" option is checked. [FR-TARGET-0015] \bl 
\end{tabular}
\subsubsection*{Scenario UC-26-SC-2}
\begin{tabular}{p{1in}p{4in}}
{\bf Description:} & "Refresh automatically" option disabled. \\
{\bf From steps:} & 1M \\
{\bf To steps:} & END \\
\end{tabular}
 
\begin{tabular}{|p{0.8in}|p{1.6in}|p{1.6in}|p{1.6in}|}
\hline
Id & User Action & Condition & System Response  \bl 
1A & Click in "Refresh Automatically" option. & TaRGeT is already started up. A project is already opened. & "Refresh Automatically" option is not checked. \bl 
2A & Open an already imported Use Case Document with the Microsoft Word (outside the TaRGeT). & There is at least one imported Use Case Document. [FR-TARGET-0100] & The document is opened in the default viewer outside the TaRGeT. \bl 
3A & Modify any field in Use Case Document and save the document. & - & The document is modified. \bl 
4A & Go back to TaRGeT work area. & - & The project is not refreshed. \bl 
5A & Go to "Project" in the menu bar and select "Refresh project" option. & - & A progress bar is displayed while the project is being refreshed. [FR-TARGET-0110, FR-TARGET-0015] \bl 
\end{tabular}
\subsubsection*{Scenario UC-26-SC-3}
\begin{tabular}{p{1in}p{4in}}
{\bf Description:} & "Refresh automatically" option enabled. \\
{\bf From steps:} & 1M \\
{\bf To steps:} & END \\
\end{tabular}
 
\begin{tabular}{|p{0.8in}|p{1.6in}|p{1.6in}|p{1.6in}|}
\hline
Id & User Action & Condition & System Response  \bl 
1B & Go to "Project" in the menu bar and select "Refresh project" option. & - & The project is refreshed. [FR-TARGET-0015] \bl 
\end{tabular}
\subsubsection*{Scenario UC-26-SC-4}
\begin{tabular}{p{1in}p{4in}}
{\bf Description:} & Using the Refresh shortcut. \\
{\bf From steps:} & 3A \\
{\bf To steps:} & END \\
\end{tabular}
 
\begin{tabular}{|p{0.8in}|p{1.6in}|p{1.6in}|p{1.6in}|}
\hline
Id & User Action & Condition & System Response  \bl 
1C & Press CTRL+R. & - & A progress bar is displayed while the project is being refreshed. [FR-TARGET-0110, FR-TARGET-0015] \bl 
\end{tabular}
\subsection*{Use case UC-30}
\begin{itemize}
\item {\bf Name: }Importing Valid Use Case Documents
\item {\bf Description: }Import Use Case Documents.
\end{itemize}
\subsubsection*{Scenario UC-30-SC-1}
\begin{tabular}{p{1in}p{4in}}
{\bf Description:} & Import Use Case Documents. \\
{\bf From steps:} & START \\
{\bf To steps:} & END \\
\end{tabular}
 
\begin{tabular}{|p{0.8in}|p{1.6in}|p{1.6in}|p{1.6in}|}
\hline
Id & User Action & Condition & System Response  \bl 
1M & Go to "Artifacts" option in the menu bar. & TaRGeT is started up. A project is already opened. & The "Import Use Case Documents" option is enabled. \bl 
2M & Select "Import Use Case Documents" option. [FR-TARGET-0100] & - & The "Import Documents" window is displayed. The "Finish" button is disabled. \bl 
3M & Select "Add Document" button and browse a valid Use Case Document. & No document is imported. & 
		      The selected document path is inserted in the "Documents To Import" list. The "Finish" button is enabled. @addDocument
		     \bl 
4M & Click on "Finish" button. & - & A progress bar is displayed indicating that the document(s) is (are) being processed and imported.[FR-TARGET-0110] \bl 
5M & Wait the progress bar. & - & The imported document(s) is (are) placed in a specific folder and displayed in the "Artifacts" View. The imported use cases are displayed in the "Use Cases" View. @endOfImportDocument [FR-TARGET-0010, FR-TARGET-0015] \bl 
\end{tabular}
\subsubsection*{Scenario UC-30-SC-2}
\begin{tabular}{p{1in}p{4in}}
{\bf Description:} & Import another document. \\
{\bf From steps:} & 3M \\
{\bf To steps:} & 4M \\
\end{tabular}
 
\begin{tabular}{|p{0.8in}|p{1.6in}|p{1.6in}|p{1.6in}|}
\hline
Id & User Action & Condition & System Response  \bl 
1A & Select "Add Document" button and browse another valid Use Case Document. The document must not have duplicated feature ids, when compared with the already browsed document. & Another document has to be inserted on the "Documents To Import" list. & The selected document path is inserted in the "Documents To Import" list. \bl 
\end{tabular}
\subsubsection*{Scenario UC-30-SC-3}
\begin{tabular}{p{1in}p{4in}}
{\bf Description:} & Cancel the importing. \\
{\bf From steps:} & 2M,3M \\
{\bf To steps:} & END \\
\end{tabular}
 
\begin{tabular}{|p{0.8in}|p{1.6in}|p{1.6in}|p{1.6in}|}
\hline
Id & User Action & Condition & System Response  \bl 
1B & Click on "Cancel" button. & - & The focus goes back to the project work area. \bl 
\end{tabular}
\subsubsection*{Scenario UC-30-SC-4}
\begin{tabular}{p{1in}p{4in}}
{\bf Description:} & Import through shortcut. \\
{\bf From steps:} & START \\
{\bf To steps:} & 3M \\
\end{tabular}
 
\begin{tabular}{|p{0.8in}|p{1.6in}|p{1.6in}|p{1.6in}|}
\hline
Id & User Action & Condition & System Response  \bl 
1C & Press CTRL+I. & - & The "Import Documents" window is displayed. The "Finish" button is disabled. \bl 
\end{tabular}
\subsection*{Use case UC-35}
\begin{itemize}
\item {\bf Name: }Importing a Use Case Document With Duplicated Feature ID
\item {\bf Description: }Use Case Document has duplicated feature ID.
\end{itemize}
\subsubsection*{Scenario UC-35-SC-1}
\begin{tabular}{p{1in}p{4in}}
{\bf Description:} & Imported Use Case Document has duplicated feature ID. \\
{\bf From steps:} & UC-30-2M \\
{\bf To steps:} & END \\
\end{tabular}
 
\begin{tabular}{|p{0.8in}|p{1.6in}|p{1.6in}|p{1.6in}|}
\hline
Id & User Action & Condition & System Response  \bl 
1M & Select "Add Document" button and browse a Use Case Document with duplicated feature ID. This means that the document contains at least two features with the same ID. & There is no imported document. & The selected document path is inserted in the "Documents To Import" list. The "Finish" button is enabled. \bl 
2M & Click on "Finish" button. & - & A progress bar is displayed indicating that the document is being processed. [FR-TARGET-0110] \bl 
3M & Wait the progress bar. & - & An error dialog is displayed indicating that the document that has duplicated feature id [FR-TARGET-0125] \bl 
4M & Click on "OK" button. & - & The selected document(s) will not be imported. The focus goes back to "Import Documents" window. \bl 
\end{tabular}
\subsection*{Use case UC-40}
\begin{itemize}
\item {\bf Name: }Importing a Use Case Document With Duplicated Use Case ID
\item {\bf Description: }Use Case Document has duplicated use case ID.
\end{itemize}
\subsubsection*{Scenario UC-40-SC-1}
\begin{tabular}{p{1in}p{4in}}
{\bf Description:} & Imported Use Case Document has duplicated use case ID. \\
{\bf From steps:} & UC-30-2M \\
{\bf To steps:} & END \\
\end{tabular}
 
\begin{tabular}{|p{0.8in}|p{1.6in}|p{1.6in}|p{1.6in}|}
\hline
Id & User Action & Condition & System Response  \bl 
1M & Select "Add Document" button and browse a Use Case Document with duplicated use case ID. This means that the document contains a feature that has at least two use cases with the same ID. & No document is imported. & The selected document path is inserted in the "Documents To Import" list. The "Finish" button is enabled. \bl 
2M & Click on "Finish" button. & - & A progress bar is displayed indicating that the document is being processed. [FR-TARGET-0110] \bl 
3M & Wait the progress bar. & - & An error dialog is displayed indicating the document that has duplicated use case ID. [FR-TARGET-0125] \bl 
4M & Click on "OK" button. & - & The selected document(s) will not be imported. The focus goes back to "Import Documents" window. \bl 
\end{tabular}
\subsection*{Use case UC-45}
\begin{itemize}
\item {\bf Name: }Importing a Use Case Document With Invalid Structure
\item {\bf Description: }Use Case Document has invalid structure.
\end{itemize}
\subsubsection*{Scenario UC-45-SC-1}
\begin{tabular}{p{1in}p{4in}}
{\bf Description:} & Imported Use Case Document has an invalid structure. \\
{\bf From steps:} & UC-30-2M \\
{\bf To steps:} & END \\
\end{tabular}
 
\begin{tabular}{|p{0.8in}|p{1.6in}|p{1.6in}|p{1.6in}|}
\hline
Id & User Action & Condition & System Response  \bl 
1M & Select "Add Document" button and browse a Use Case Document with invalid structure. & No document is imported. & The selected document path is inserted in the "Documents To Import" list. The "Finish" button is enabled. \bl 
2M & Click on "Finish" button. & - & A progress bar is displayed indicating that the document is being processed. [FR-TARGET-0110] \bl 
3M & Wait the progress bar. & - & An error dialog is displayed indicating the document that has an error. [FR-TARGET-0125] \bl 
4M & Click on "OK" button. & - & The selected document(s) will not be imported. The focus goes back to "Import Documents" window. \bl 
\end{tabular}
\subsubsection*{Scenario UC-45-SC-2}
\begin{tabular}{p{1in}p{4in}}
{\bf Description:} & Import another document. \\
{\bf From steps:} & 1M \\
{\bf To steps:} & 2M \\
\end{tabular}
 
\begin{tabular}{|p{0.8in}|p{1.6in}|p{1.6in}|p{1.6in}|}
\hline
Id & User Action & Condition & System Response  \bl 
1A & Click on "Add Document" button and browse another valid Use Case Document. & Another document has to be imported. & The selected document path is inserted in the "Documents To Import" list. \bl 
\end{tabular}
\subsubsection*{Scenario UC-45-SC-3}
\begin{tabular}{p{1in}p{4in}}
{\bf Description:} & Cancel the importing. \\
{\bf From steps:} & 4M \\
{\bf To steps:} & END \\
\end{tabular}
 
\begin{tabular}{|p{0.8in}|p{1.6in}|p{1.6in}|p{1.6in}|}
\hline
Id & User Action & Condition & System Response  \bl 
1B & Click on "Cancel" button. & - & The focus goes back to the project work area. \bl 
\end{tabular}
\subsection*{Use case UC-50}
\begin{itemize}
\item {\bf Name: }Importing a Use Case Document with an Empty Mandatory Field
\item {\bf Description: }Use Case Document has an empty mandatory field.
\end{itemize}
\subsubsection*{Scenario UC-50-SC-1}
\begin{tabular}{p{1in}p{4in}}
{\bf Description:} & Selected Use Case Document has an empty mandatory field. \\
{\bf From steps:} & UC-30-2M \\
{\bf To steps:} & END \\
\end{tabular}
 
\begin{tabular}{|p{0.8in}|p{1.6in}|p{1.6in}|p{1.6in}|}
\hline
Id & User Action & Condition & System Response  \bl 
1M & Select "Add Document" button and browse a Use Case Document with an empty mandatory field. & No document is imported. & The selected document path is inserted in the "Documents To Import" list. The "Finish" button is enabled. \bl 
2M & Click on "Finish" button. & - & An error dialog is displayed indicating that the document has an error in the xml structure. [FR-TARGET-0125] \bl 
3M & Click on "OK" button. & - & The selected document(s) will not be imported. The focus goes back to "Import Documents" window. \bl 
\end{tabular}
\subsection*{Use case UC-55}
\begin{itemize}
\item {\bf Name: }Importing a Use Case Document With Invalid Step ID References
\item {\bf Description: }Use Case Document has invalid id references.
\end{itemize}
\subsubsection*{Scenario UC-55-SC-1}
\begin{tabular}{p{1in}p{4in}}
{\bf Description:} & The selected Use Case Document contains invalid id references. \\
{\bf From steps:} & UC-30-2M \\
{\bf To steps:} & END \\
\end{tabular}
 
\begin{tabular}{|p{0.8in}|p{1.6in}|p{1.6in}|p{1.6in}|}
\hline
Id & User Action & Condition & System Response  \bl 
1M & Select "Add Document" button and browse a Use Case Document with invalid ID references. & No document is imported. & The selected document path is inserted in the "Documents To Import" list. The "Finish" button is enabled. \bl 
2M & Click on "Finish" button. & - & A progress bar is displayed indicating that the document is being processed. [FR-TARGET-0110] \bl 
3M & Wait the progress bar. & - & A warning message is displayed indicating that the document contains errors. \bl 
4M & Click on "OK" button. & - & The imported document is placed in the documents folder and is displayed in the "Artifacts" View. The imported use cases are displayed in the "Use Cases" View. A special icon indicates the use cases with invalid references. Also the error view displays details about invalid id references and/or duplicated ids. [FR-TARGET-0010, FR-TARGET-0120] \bl 
\end{tabular}
\subsubsection*{Scenario UC-55-SC-2}
\begin{tabular}{p{1in}p{4in}}
{\bf Description:} & Import another valid document. \\
{\bf From steps:} & 1M \\
{\bf To steps:} & 2M \\
\end{tabular}
 
\begin{tabular}{|p{0.8in}|p{1.6in}|p{1.6in}|p{1.6in}|}
\hline
Id & User Action & Condition & System Response  \bl 
1A & Select "Add Document" button and browse another valid Use Case Document. & Another document has to be imported. & The selected document path is inserted in the "Documents To Import" list. \bl 
\end{tabular}
\subsection*{Use case UC-63}
\begin{itemize}
\item {\bf Name: }Importing a Use Case Document with a Feature Already Imported.
\item {\bf Description: }The documents have at least one feature with same name and ID. Another Use Case Document is already imported.
\end{itemize}
\subsubsection*{Scenario UC-63-SC-1}
\begin{tabular}{p{1in}p{4in}}
{\bf Description:} & A Use Case Document is already imported. The Use Case Documents have at least one common feature with same ID and name. \\
{\bf From steps:} & UC-30-2M \\
{\bf To steps:} & END \\
\end{tabular}
 
\begin{tabular}{|p{0.8in}|p{1.6in}|p{1.6in}|p{1.6in}|}
\hline
Id & User Action & Condition & System Response  \bl 
1M & Select "Add Document" button and browse a valid Use Case Document. The document and the already imported document have at least one common feature with same ID and name and the use cases IDs are different. & There is at least one valid document imported in the project. & The "Import Documents" window is displayed. The "Finish" button is enabled. \bl 
2M & Click on "Finish" button. & - & A progress bar is displayed indicating that the document(s) is (are) being processed and imported. [FR-TARGET-0110] \bl 
3M & Wait the progress bar. & - & The imported document is placed in a specific folder and displayed in the "Artifacts" view. The imported use cases are displayed in the "Use Cases" view. The feature icon in the "Use Cases" view shows that the feature was merged. [FR-TARGET-0010, FR-TARGET-0100] \bl 
\end{tabular}
\subsubsection*{Scenario UC-63-SC-2}
\begin{tabular}{p{1in}p{4in}}
{\bf Description:} & A Use Case Document is already imported. The Use Case Documents have at least one common feature with same ID and different names. \\
{\bf From steps:} & UC-30-2M \\
{\bf To steps:} & END \\
\end{tabular}
 
\begin{tabular}{|p{0.8in}|p{1.6in}|p{1.6in}|p{1.6in}|}
\hline
Id & User Action & Condition & System Response  \bl 
1A & Select "Add Document" button and browse a valid Use Case Document. The document and the already imported document have at least one common feature with same ID but different names and the use cases IDs are different. & There is at least one valid document imported in the project. & The "Import Documents" window is displayed. The "Finish" button is enabled. \bl 
2A & Click on "Finish" button. & - & A progress bar is displayed indicating that the document(s) is (are) being processed and imported. [FR-TARGET-0110] \bl 
3A & Wait the progress bar. & - & The imported document is placed in a specific folder and displayed in the "Artifacts" View. The imported use cases are displayed in the "Use Cases" View. The feature icon in the "Use Cases" view shows that the feature was merged. A new warning is displayed in the "Error" view indicating that the features have same ID but different names. [FR-TARGET-0010, FR-TARGET-0100,  FR-TARGET-0120] \bl 
\end{tabular}
\subsubsection*{Scenario UC-63-SC-3}
\begin{tabular}{p{1in}p{4in}}
{\bf Description:} & The document contains a feature with same ID of another feature that belongs to a previously imported document. \\
{\bf From steps:} & UC-30-2M \\
{\bf To steps:} & END \\
\end{tabular}
 
\begin{tabular}{|p{0.8in}|p{1.6in}|p{1.6in}|p{1.6in}|}
\hline
Id & User Action & Condition & System Response  \bl 
1B & Select "Add Document" button and browse a valid Use Case Document. The document and the already imported document have at least one common feature with same ID. The features have at least one use case with same ID. & There is at least one valid imported document. & The selected document path is inserted in the "Documents To Import" list. \bl 
2B & Click on "Finish" button. & - & A progress bar is displayed indicating that the document is being processed. [FR-TARGET-0110] \bl 
3B & Wait the progress bar. & - & An error dialog is displayed indicating the document that has an error. [FR-TARGET-0125] \bl 
4B & Click on "OK" button. & - & The selected document(s) will not be imported. The focus goes back to "Import Documents" window. \bl 
\end{tabular}
\subsection*{Use case UC-70}
\begin{itemize}
\item {\bf Name: }Viewing Valid Use Case in HTML Format
\item {\bf Description: }Viewing Use Case Documents in HTML format.
\end{itemize}
\subsubsection*{Scenario UC-70-SC-1}
\begin{tabular}{p{1in}p{4in}}
{\bf Description:} & View valid Use Case Documents in HTML format in the main panel, by double-clicking the use case. \\
{\bf From steps:} & START \\
{\bf To steps:} & END \\
\end{tabular}
 
\begin{tabular}{|p{0.8in}|p{1.6in}|p{1.6in}|p{1.6in}|}
\hline
Id & User Action & Condition & System Response  \bl 
1M & Double-click on a use case in the "Use Cases" view. & A project is opened in TaRGeT. The project has at least one imported valid Use Case Document. & The selected Use Case is displayed in HTML format in the main panel. [FR-TARGET-0130] \bl 
\end{tabular}
\subsubsection*{Scenario UC-70-SC-2}
\begin{tabular}{p{1in}p{4in}}
{\bf Description:} & View Use Case Documents in HTML format in the main panel, by right-clicking the use case. \\
{\bf From steps:} & START \\
{\bf To steps:} & END \\
\end{tabular}
 
\begin{tabular}{|p{0.8in}|p{1.6in}|p{1.6in}|p{1.6in}|}
\hline
Id & User Action & Condition & System Response  \bl 
1A & Right-click on a use case in the "Use Cases" view. & A project is opened in TaRGeT. The project has at least one imported valid Use Case Document. & A drop down menu is displayed with two options. \bl 
2A & Select "Open in default view" option. & - & The selected Use Case is displayed in HTML format in the main panel. [FR-TARGET-0130] \bl 
\end{tabular}
\subsubsection*{Scenario UC-70-SC-3}
\begin{tabular}{p{1in}p{4in}}
{\bf Description:} & View Use Case Documents in HTML format in the default browser, by right-clicking the use case. \\
{\bf From steps:} & START \\
{\bf To steps:} & END \\
\end{tabular}
 
\begin{tabular}{|p{0.8in}|p{1.6in}|p{1.6in}|p{1.6in}|}
\hline
Id & User Action & Condition & System Response  \bl 
1B & Right-click on a use case in the "Use Cases" view. & A project is opened in TaRGeT. The project has at least one imported valid Use Case Document. & A drop down menu is displayed with two options. \bl 
2B & Select "Open with default browser" option. & - & The selected Use Case is displayed in HTML format in the default browser. [FR-TARGET-0130] \bl 
\end{tabular}
\subsection*{Use case UC-71}
\begin{itemize}
\item {\bf Name: }Viewing Invalid Use Case in HTML Format
\item {\bf Description: }Viewing Use Case Documents in HTML format when they have some error.
\end{itemize}
\subsubsection*{Scenario UC-71-SC-1}
\begin{tabular}{p{1in}p{4in}}
{\bf Description:} & View Use Case Documents with duplicated step ID in HTML format in the main panel, by double-clicking the use case. \\
{\bf From steps:} & START \\
{\bf To steps:} & END \\
\end{tabular}
 
\begin{tabular}{|p{0.8in}|p{1.6in}|p{1.6in}|p{1.6in}|}
\hline
Id & User Action & Condition & System Response  \bl 
1M & Double-click on a use case with duplicated step ID in the "Use Cases" view. & A project is opened in TaRGeT. The project has at least one imported use case with duplicated step ID. & The selected Use Case is displayed in HTML format in the main panel. The duplicated steps IDs are highlighted.   [FR-TARGET-0130] \bl 
\end{tabular}
\subsubsection*{Scenario UC-71-SC-2}
\begin{tabular}{p{1in}p{4in}}
{\bf Description:} & View Use Case Documents with invalid step ID reference in HTML format in the main panel, by right-clicking the use case. \\
{\bf From steps:} & START \\
{\bf To steps:} & END \\
\end{tabular}
 
\begin{tabular}{|p{0.8in}|p{1.6in}|p{1.6in}|p{1.6in}|}
\hline
Id & User Action & Condition & System Response  \bl 
1A & Right-click on a use case with invalid step ID reference in the "Use Cases" view. & A project is opened in TaRGeT. The project has at least one imported use case with invalid step ID reference. & A drop down menu is displayed with two options. \bl 
2A & Select "Open in default view" option. & - & The selected Use Case is displayed in HTML format in the main panel. The invalid step ID reference is highlighted. [FR-TARGET-0130] \bl 
\end{tabular}
\subsubsection*{Scenario UC-71-SC-3}
\begin{tabular}{p{1in}p{4in}}
{\bf Description:} & View Use Case Documents with invalid step ID reference in HTML format in the default browser, by right-clicking the use case. \\
{\bf From steps:} & START \\
{\bf To steps:} & END \\
\end{tabular}
 
\begin{tabular}{|p{0.8in}|p{1.6in}|p{1.6in}|p{1.6in}|}
\hline
Id & User Action & Condition & System Response  \bl 
1B & Right-click on a use case with invalid step ID reference in the "Use Cases" view. & A project is opened in TaRGeT. The project has at least one imported use case with invalid step ID reference. & A drop down menu is displayed with two options. \bl 
2B & Select "Open with default browser" option. & - & The selected Use Case is displayed in HTML format in the default browser. The invalid step ID reference is highlighted.  [FR-TARGET-0130] \bl 
\end{tabular}
\subsection*{Use case UC-80}
\begin{itemize}
\item {\bf Name: }Renaming a Document
\item {\bf Description: }Renaming Use Case or Test Suite Documents in TaRGeT.
\end{itemize}
\subsubsection*{Scenario UC-80-SC-1}
\begin{tabular}{p{1in}p{4in}}
{\bf Description:} & Rename a test suite document. \\
{\bf From steps:} & START \\
{\bf To steps:} & END \\
\end{tabular}
 
\begin{tabular}{|p{0.8in}|p{1.6in}|p{1.6in}|p{1.6in}|}
\hline
Id & User Action & Condition & System Response  \bl 
1M & Right-click on a generated test suite document in the "Artifacts" view. & A project is opened in TaRGeT. The project has at least one already generated test suite. & A drop down menu is displayed. \bl 
2M & Click on "Rename" option in the drop down menu. [FR-TARGET-0001] & - & A dialog is popped up with a text field. \bl 
3M & Change the document name to a valid name and click on "OK" button. [FR-TARGET-0025] & - & The document is renamed. \bl 
\end{tabular}
\subsubsection*{Scenario UC-80-SC-2}
\begin{tabular}{p{1in}p{4in}}
{\bf Description:} & Rename a Use Case Document. \\
{\bf From steps:} & START \\
{\bf To steps:} & END \\
\end{tabular}
 
\begin{tabular}{|p{0.8in}|p{1.6in}|p{1.6in}|p{1.6in}|}
\hline
Id & User Action & Condition & System Response  \bl 
1A & Right-click on a Use Case Document in the "Artifacts" view. & A project is opened in TaRGeT. The project has at least one imported Use Case Document. & A drop down menu is displayed. \bl 
2A & Click on "Rename" option in the drop down menu. [FR-TARGET-0001] & - & A dialog is popped up with a text field. \bl 
3A & Change the document name to a valid name and click on "OK" button. [FR-TARGET-0025] & - & The document is renamed. \bl 
\end{tabular}
\subsubsection*{Scenario UC-80-SC-3}
\begin{tabular}{p{1in}p{4in}}
{\bf Description:} & Select to rename more than one document. \\
{\bf From steps:} & START \\
{\bf To steps:} & END \\
\end{tabular}
 
\begin{tabular}{|p{0.8in}|p{1.6in}|p{1.6in}|p{1.6in}|}
\hline
Id & User Action & Condition & System Response  \bl 
1B & In the "Artifacts" view, Select more than one document and right-click on the selection. [FR-TARGET-0001] & A project is opened in TaRGeT. The project has at least two documents (test suite or Use Case Documents). & A drop down menu is displayed. The "Rename" option is disabled. \bl 
\end{tabular}
\subsubsection*{Scenario UC-80-SC-4}
\begin{tabular}{p{1in}p{4in}}
{\bf Description:} & Type an invalid document name. \\
{\bf From steps:} & 2M \\
{\bf To steps:} & END \\
\end{tabular}
 
\begin{tabular}{|p{0.8in}|p{1.6in}|p{1.6in}|p{1.6in}|}
\hline
Id & User Action & Condition & System Response  \bl 
1C & Try to change the document name to an invalid name (e.g. a name containing "<" or "*" characters). [FR-TARGET-0025] & - & A message is displayed indicating that the name is invalid. The "OK" button is disabled. \bl 
2C & Click on "Cancel" button. & - & The focus goes back to the main window. \bl 
\end{tabular}
\subsubsection*{Scenario UC-80-SC-5}
\begin{tabular}{p{1in}p{4in}}
{\bf Description:} & Type an invalid document name and retype a valid name. \\
{\bf From steps:} & 2M \\
{\bf To steps:} & END \\
\end{tabular}
 
\begin{tabular}{|p{0.8in}|p{1.6in}|p{1.6in}|p{1.6in}|}
\hline
Id & User Action & Condition & System Response  \bl 
1D & Try to change the document name to an invalid name (e.g. a name containing "<" or "*" characters). [FR-TARGET-0025] & - & A message is displayed indicating that the name is invalid. The "OK" button is disabled. \bl 
2D & Erase the invalid name and type a valid name. & - & The OK button is enabled. \bl 
3D & Click on the OK button & - & The document is renamed. \bl 
\end{tabular}
\subsubsection*{Scenario UC-80-SC-6}
\begin{tabular}{p{1in}p{4in}}
{\bf Description:} & Cancel the rename. \\
{\bf From steps:} & 2D \\
{\bf To steps:} & END \\
\end{tabular}
 
\begin{tabular}{|p{0.8in}|p{1.6in}|p{1.6in}|p{1.6in}|}
\hline
Id & User Action & Condition & System Response  \bl 
1E & Click on "Cancel" button. & - & The focus goes back to the main window. \bl 
\end{tabular}
\subsubsection*{Scenario UC-80-SC-7}
\begin{tabular}{p{1in}p{4in}}
{\bf Description:} & Type a document name that is being used by another use case document. \\
{\bf From steps:} & START \\
{\bf To steps:} & 2C,2D \\
\end{tabular}
 
\begin{tabular}{|p{0.8in}|p{1.6in}|p{1.6in}|p{1.6in}|}
\hline
Id & User Action & Condition & System Response  \bl 
1F & Right-click on a Use Case Document in the "Artifacts" view. & A project is opened in TaRGeT. The project has at least two use cases. & A drop down menu is displayed. The "Rename" option is enabled. \bl 
2F & Click on "Rename" option in the drop down menu. [FR-TARGET-0001] & - & A dialog is popped up with a text field. \bl 
3F & Type a name that is already in use. & - & A message is displayed indicating that the name is already in use. The "OK" button is disabled. \bl 
\end{tabular}
\subsubsection*{Scenario UC-80-SC-8}
\begin{tabular}{p{1in}p{4in}}
{\bf Description:} & Type a test suite file name that is being used by another test suite file. \\
{\bf From steps:} & START \\
{\bf To steps:} & 2F \\
\end{tabular}
 
\begin{tabular}{|p{0.8in}|p{1.6in}|p{1.6in}|p{1.6in}|}
\hline
Id & User Action & Condition & System Response  \bl 
1G & Right-click on a test case document in the "Artifacts" view. & A project is opened in TaRGeT. The project has at least two test suites. & A drop down menu is displayed. The "Rename" option is enabled. \bl 
\end{tabular}
\subsubsection*{Scenario UC-80-SC-9}
\begin{tabular}{p{1in}p{4in}}
{\bf Description:} & Type an empty name. \\
{\bf From steps:} & 2M \\
{\bf To steps:} & 2C \\
\end{tabular}
 
\begin{tabular}{|p{0.8in}|p{1.6in}|p{1.6in}|p{1.6in}|}
\hline
Id & User Action & Condition & System Response  \bl 
1H & Type an empty name. [FR-TARGET-0025] & - & A message is displayed indicating that the name must not be empty. The "OK" button is disabled. \bl 
\end{tabular}
\subsection*{Use case UC-85}
\begin{itemize}
\item {\bf Name: }Deleting a Document
\item {\bf Description: }Deleting Use Cases or Test Suite Documents in TaRGeT.
\end{itemize}
\subsubsection*{Scenario UC-85-SC-1}
\begin{tabular}{p{1in}p{4in}}
{\bf Description:} & Delete a test suite document. \\
{\bf From steps:} & START \\
{\bf To steps:} & END \\
\end{tabular}
 
\begin{tabular}{|p{0.8in}|p{1.6in}|p{1.6in}|p{1.6in}|}
\hline
Id & User Action & Condition & System Response  \bl 
1M & Right-click on a generated test suite document in the "Artifacts" view. & A project is opened in TaRGeT. The project has at least one already generated test suite. & A drop down menu is displayed. \bl 
2M & Click on "Delete" option in the drop down menu. [FR-TARGET-0003] & - & A dialog is displayed asking a confirmation. \bl 
3M & Click on "Yes" button. & - & The document(s) is (are) deleted. The project work area is refreshed. [FR-TARGET-0015] \bl 
\end{tabular}
\subsubsection*{Scenario UC-85-SC-2}
\begin{tabular}{p{1in}p{4in}}
{\bf Description:} & Delete a Use Case Document that is not referred by another document. \\
{\bf From steps:} & START \\
{\bf To steps:} & END \\
\end{tabular}
 
\begin{tabular}{|p{0.8in}|p{1.6in}|p{1.6in}|p{1.6in}|}
\hline
Id & User Action & Condition & System Response  \bl 
1B & Right-click on a Use Case Document in the "Artifacts" view. & A project is opened in TaRGeT. The project has at least one imported Use Case Document. The imported documents do not refer to a use case of another imported document. & A drop down menu is displayed. \bl 
2B & Click on "Delete" option in the drop down menu. [FR-TARGET-0003] & - & A dialog is displayed asking a confirmation. \bl 
3B & Click on "Yes" button. & - & The document(s) is(are) deleted. The project work area is refreshed. [FR-TARGET-0015] \bl 
\end{tabular}
\subsubsection*{Scenario UC-85-SC-3}
\begin{tabular}{p{1in}p{4in}}
{\bf Description:} & Delete a Use Case Document that is referred by another document. \\
{\bf From steps:} & START \\
{\bf To steps:} & END \\
\end{tabular}
 
\begin{tabular}{|p{0.8in}|p{1.6in}|p{1.6in}|p{1.6in}|}
\hline
Id & User Action & Condition & System Response  \bl 
1C & Right-click on a Use Case Document in the "Artifacts" view. The selected document is referred by a use case of another imported document. & A project is opened in TaRGeT. The project has at least two imported Use Case Documents. A use case is referred by another use case from a different document. & A drop down menu is displayed. \bl 
2C & Click on "Delete" option in the drop down menu. [FR-TARGET-0003] & - & A dialog is displayed asking a confirmation and informing that the document is being referred by another document. \bl 
3C & Click on "Yes" button. & - & The document(s) is(are) deleted. The project work area is refreshed. Some invalid reference errors are displayed in "Error" view. [FR-TARGET-0015, FR-TARGET-0120] \bl 
\end{tabular}
\subsubsection*{Scenario UC-85-SC-4}
\begin{tabular}{p{1in}p{4in}}
{\bf Description:} & Delete more than one generated test suite. \\
{\bf From steps:} & START \\
{\bf To steps:} & 2M \\
\end{tabular}
 
\begin{tabular}{|p{0.8in}|p{1.6in}|p{1.6in}|p{1.6in}|}
\hline
Id & User Action & Condition & System Response  \bl 
1D & In the "Artifacts" view, Select more than one test suite document and right-click on the selection. & A project is opened in TaRGeT. The project has at least two generated test suites. & A drop down menu is displayed. \bl 
\end{tabular}
\subsubsection*{Scenario UC-85-SC-5}
\begin{tabular}{p{1in}p{4in}}
{\bf Description:} & Delete one test suite and one Use Case Document that is being referenced. \\
{\bf From steps:} & START \\
{\bf To steps:} & END \\
\end{tabular}
 
\begin{tabular}{|p{0.8in}|p{1.6in}|p{1.6in}|p{1.6in}|}
\hline
Id & User Action & Condition & System Response  \bl 
1E & In the "Artifacts" view, Select one test suite document and a Use Case Document, and right-click on the selection. & A project is opened in TaRGeT. The project has at least two imported Use Case Document and at least one generated test suite. The imported Use Case Document is referenced by another imported document. & A drop down menu is displayed. \bl 
2E & Click on "Delete" option in the drop down menu. [FR-TARGET-0003] & - & A dialog is displayed asking a confirmation and informing that the document is being referred by another document. \bl 
3E & Click on "Yes" button. & - & The document(s) is(are) deleted. The project work area is refreshed. Some invalid reference errors are displayed in "Error" view. [FR-TARGET-0015, FR-TARGET-0120] \bl 
\end{tabular}
\subsubsection*{Scenario UC-85-SC-6}
\begin{tabular}{p{1in}p{4in}}
{\bf Description:} & Delete two Use Case Documents. \\
{\bf From steps:} & START \\
{\bf To steps:} & 2C \\
\end{tabular}
 
\begin{tabular}{|p{0.8in}|p{1.6in}|p{1.6in}|p{1.6in}|}
\hline
Id & User Action & Condition & System Response  \bl 
1F & In the "Artifacts" view, Select one test suite two Use Case Documents, and right-click on the selection. One of the selected documents refers to the other. & A project is opened in TaRGeT. The project has two imported Use Case Document. One imported Use Case Document is referenced by the other. & A drop down menu is displayed. \bl 
\end{tabular}
\subsubsection*{Scenario UC-85-SC-7}
\begin{tabular}{p{1in}p{4in}}
{\bf Description:} & Cancel the document deletion. \\
{\bf From steps:} & 2M,2C \\
{\bf To steps:} & END \\
\end{tabular}
 
\begin{tabular}{|p{0.8in}|p{1.6in}|p{1.6in}|p{1.6in}|}
\hline
Id & User Action & Condition & System Response  \bl 
1A & Click on "No" button. & - & The document(s) is (are) not deleted. \bl 
\end{tabular}
\subsection*{Use case UC-106}
\begin{itemize}
\item {\bf Name: }Searching in Use Case Document
\item {\bf Description: }Searching in Use Case Document.
\end{itemize}
\subsubsection*{Scenario UC-106-SC-1}
\begin{tabular}{p{1in}p{4in}}
{\bf Description:} & Perform a search when some valid Use Case Documents are already imported. \\
{\bf From steps:} & START \\
{\bf To steps:} & END \\
\end{tabular}
 
\begin{tabular}{|p{0.8in}|p{1.6in}|p{1.6in}|p{1.6in}|}
\hline
Id & User Action & Condition & System Response  \bl 
1M & Choose "Tools" option in the menu bar. & TaRGeT is started up. A project is already opened. There is at least one document imported. No error is listed in the "Error" list. & A drop down menu is displayed. "Search" option is available. \bl 
2M & Choose "Search" option in the drop down menu. & - & "Search" window is displayed. [FR-TARGET-0135] \bl 
3M & Type a query in the input area. & - & The "Find" field is filled. \bl 
4M & Click on "Search" button. & - & The search results are displayed in "Search Results" View. Verify if the results are in accordance to the query. \bl 
\end{tabular}
\subsubsection*{Scenario UC-106-SC-2}
\begin{tabular}{p{1in}p{4in}}
{\bf Description:} & Cancel the search. \\
{\bf From steps:} & 3M \\
{\bf To steps:} & END \\
\end{tabular}
 
\begin{tabular}{|p{0.8in}|p{1.6in}|p{1.6in}|p{1.6in}|}
\hline
Id & User Action & Condition & System Response  \bl 
1A & Click on "Cancel" button. & - & The focus goes back to the work area. \bl 
\end{tabular}
\subsubsection*{Scenario UC-106-SC-3}
\begin{tabular}{p{1in}p{4in}}
{\bf Description:} & Using the specific field "Use Case Identifier". \\
{\bf From steps:} & 2M \\
{\bf To steps:} & END \\
\end{tabular}
 
\begin{tabular}{|p{0.8in}|p{1.6in}|p{1.6in}|p{1.6in}|}
\hline
Id & User Action & Condition & System Response  \bl 
1B & "Type a query for serching based on use case Id, i.e. "ucid:<use-case-id>". A use case with id <use-case-id> must be contained in the project. & - & The "Find" field is filled. \bl 
2B & Click on "Search" button. & - & At least one use case is displayed in "Search Results" View.  [FR-TARGET-0135] \bl 
\end{tabular}
\subsubsection*{Scenario UC-106-SC-4}
\begin{tabular}{p{1in}p{4in}}
{\bf Description:} & Using the specific field "From Step". \\
{\bf From steps:} & 2M \\
{\bf To steps:} & 4M \\
\end{tabular}
 
\begin{tabular}{|p{0.8in}|p{1.6in}|p{1.6in}|p{1.6in}|}
\hline
Id & User Action & Condition & System Response  \bl 
1C & Type "from step: START" in the input area. & - & The "Find" field is filled. \bl 
\end{tabular}
\subsubsection*{Scenario UC-106-SC-5}
\begin{tabular}{p{1in}p{4in}}
{\bf Description:} & Using shortcut. \\
{\bf From steps:} & START \\
{\bf To steps:} & 3M \\
\end{tabular}
 
\begin{tabular}{|p{0.8in}|p{1.6in}|p{1.6in}|p{1.6in}|}
\hline
Id & User Action & Condition & System Response  \bl 
1D & Press CTRL+F & - & "Search" window is displayed. [FR-TARGET-0135] \bl 
\end{tabular}
\subsection*{Use case UC-107}
\begin{itemize}
\item {\bf Name: }Viewing Search Results
\item {\bf Description: }Viewing search results.
\end{itemize}
\subsubsection*{Scenario UC-107-SC-1}
\begin{tabular}{p{1in}p{4in}}
{\bf Description:} & View found Use Case Documents in HTML format by double-clicking the use case. \\
{\bf From steps:} & UC-106-4M \\
{\bf To steps:} & END \\
\end{tabular}
 
\begin{tabular}{|p{0.8in}|p{1.6in}|p{1.6in}|p{1.6in}|}
\hline
Id & User Action & Condition & System Response  \bl 
1M & Double-click on a use case in the "Search Results" view. & - & The selected Use Case is displayed in HTML format in the main panel. The search results are highlighted. [FR-TARGET-0130] \bl 
\end{tabular}
\subsection*{Use case UC-135}
\begin{itemize}
\item {\bf Name: }Viewing the About Window
\item {\bf Description: }Displaying the About Window.
\end{itemize}
\subsubsection*{Scenario UC-135-SC-1}
\begin{tabular}{p{1in}p{4in}}
{\bf Description:} & Access the about window. \\
{\bf From steps:} & START \\
{\bf To steps:} & END \\
\end{tabular}
 
\begin{tabular}{|p{0.8in}|p{1.6in}|p{1.6in}|p{1.6in}|}
\hline
Id & User Action & Condition & System Response  \bl 
1M & Choose "Help" option in the menu bar. & The TaRGeT is already started up. & A drop down menu is displayed. [FR-TARGET-0205] \bl 
2M & Choose "About TaRGeT" option in the drop down menu. & - & "About" window is displayed. The content of the windows is detailed in TaRGeT requirements document. [FR-TARGET-0210] \bl 
\end{tabular}
\subsection*{Use case UC-140}
\begin{itemize}
\item {\bf Name: }Viewing the Help Window
\item {\bf Description: }Displaying the Help Window.
\end{itemize}
\subsubsection*{Scenario UC-140-SC-1}
\begin{tabular}{p{1in}p{4in}}
{\bf Description:} & Access the help window. \\
{\bf From steps:} & START \\
{\bf To steps:} & END \\
\end{tabular}
 
\begin{tabular}{|p{0.8in}|p{1.6in}|p{1.6in}|p{1.6in}|}
\hline
Id & User Action & Condition & System Response  \bl 
1M & Choose "Help" option in the menu bar. & The TaRGeT is already started up. & A drop down menu is displayed. \bl 
2M & Choose "Help Contents" option in the drop down menu. & - & "Help" window is displayed. [FR-TARGET-0205] \bl 
\end{tabular}
\subsection*{Use case UC-01}
\begin{itemize}
\item {\bf Name: }Opening TaRGeT
\item {\bf Description: }It describes how the TaRGeT should behave when it is started up.
\end{itemize}
\subsubsection*{Scenario UC-01-SC-3}
\begin{tabular}{p{1in}p{4in}}
{\bf Description:} & Start up the TaRGeT when Java runtime environment is not installed in the machine. \\
{\bf From steps:} & START \\
{\bf To steps:} & END \\
\end{tabular}
 
\begin{tabular}{|p{0.8in}|p{1.6in}|p{1.6in}|p{1.6in}|}
\hline
Id & User Action & Condition & System Response  \bl 
1B & Try to start the TaRGeT. & The Java runtime environment is not installed in the machine (see requirements for further information about Java version). [ER-TARGET-0020] & A message is displayed informing that no JRE is installed. The TaRGeT is not started up. [FR-TARGET-0230] \bl 
\end{tabular}
\subsubsection*{Scenario UC-01-SC-1}
\begin{tabular}{p{1in}p{4in}}
{\bf Description:} & Start up the TaRGeT with .NET and Java runtime environment installed. \\
{\bf From steps:} & START \\
{\bf To steps:} & END \\
\end{tabular}
 
\begin{tabular}{|p{0.8in}|p{1.6in}|p{1.6in}|p{1.6in}|}
\hline
Id & User Action & Condition & System Response  \bl 
1M & Start the TaRGeT. & The .NET and Java runtime environment are installed in the machine (see requirements for further information about Java and .NET versions) [ER-TARGET-0020, ER-TARGET-0025] & A splash screen is displayed. Verify the requirements document to check the splash screen. [FR-TARGET-0220] \bl 
2M & Wait some seconds. & - & The TaRGeT is started. No TaRGeT project is opened. A background image is displayed (see requirements document for more details). [FR-TARGET-0225] \bl 
\end{tabular}
\subsubsection*{Scenario UC-01-SC-2}
\begin{tabular}{p{1in}p{4in}}
{\bf Description:} & Start up the TaRGeT when .NET runtime environment is not installed in the machine. \\
{\bf From steps:} & START \\
{\bf To steps:} & END \\
\end{tabular}
 
\begin{tabular}{|p{0.8in}|p{1.6in}|p{1.6in}|p{1.6in}|}
\hline
Id & User Action & Condition & System Response  \bl 
1A & Try to start the TaRGeT. & Java runtime environment is installed in the machine. The .NET runtime environment is not installed in the machine (see requirements for further information about Java and .NET versions). [ER-TARGET-0020, ER-TARGET-0025] & A message is displayed informing that an error has occurred and asking the user to see the log file. [FR-TARGET-0230] \bl 
\end{tabular}
\subsection*{Use case UC-130}
\begin{itemize}
\item {\bf Name: }Generating Test Suites Changing Test Case Field Parameters
\item {\bf Description: }Configuring test case fields.
\end{itemize}
\subsubsection*{Scenario UC-130-SC-1}
\begin{tabular}{p{1in}p{4in}}
{\bf Description:} & Configuring test cases parameters. \\
{\bf From steps:} & START \\
{\bf To steps:} & END \\
\end{tabular}
 
\begin{tabular}{|p{0.8in}|p{1.6in}|p{1.6in}|p{1.6in}|}
\hline
Id & User Action & Condition & System Response  \bl 
1M & Go to "Tools" in the menu bar. & There is an already created project with at least one imported use case without errors. [FR-TARGET-0100] & A drop down menu is displayed with "Search", "Preferences" and "On The Fly Generation" options. The "Preferences" option is available. \bl 
2M & Select "Preferences" option. & - & The Preferences wizard is displayed with "Test Case ID", "Test Case Initial ID", "Empty Field", "Objective Prefix", "Print Use Case Description", "Print Flow Description" and "Keep Requirements" fields. \bl 
\end{tabular}
\subsubsection*{Scenario UC-130-SC-2}
\begin{tabular}{p{1in}p{4in}}
{\bf Description:} & Cancel the preferences. \\
{\bf From steps:} & 2B,2C,2D,2E,2F,2G,2H,2I,2J,2K \\
{\bf To steps:} & END \\
\end{tabular}
 
\begin{tabular}{|p{0.8in}|p{1.6in}|p{1.6in}|p{1.6in}|}
\hline
Id & User Action & Condition & System Response  \bl 
1A & Press "Cancel" button. & - & The field was not changed, the preferences screen is closed and the focus goes back to the main window. [FR-TARGET-0231] \bl 
\end{tabular}
\subsubsection*{Scenario UC-130-SC-3}
\begin{tabular}{p{1in}p{4in}}
{\bf Description:} & Changing "Objective Prefix" field. \\
{\bf From steps:} & 2M \\
{\bf To steps:} & END \\
\end{tabular}
 
\begin{tabular}{|p{0.8in}|p{1.6in}|p{1.6in}|p{1.6in}|}
\hline
Id & User Action & Condition & System Response  \bl 
1B & Type a wanted prefix in the "Objective prefix" field and press OK button. & The "Print Flow Description" field in the Preferences window  cannot be "None" & The preferences configuration is changed. [FR-TARGET-0231] \bl 
2B & Go to "On The Fly Generation" and select each test case from the list. & - & All test cases contain their objective field with the prefix defined by the user. [FR-TARGET-0235, ,FR-TARGET-0237] \bl 
\end{tabular}
\subsubsection*{Scenario UC-130-SC-4}
\begin{tabular}{p{1in}p{4in}}
{\bf Description:} & Changing "Empty Field" parameter. \\
{\bf From steps:} & 2M \\
{\bf To steps:} & END \\
\end{tabular}
 
\begin{tabular}{|p{0.8in}|p{1.6in}|p{1.6in}|p{1.6in}|}
\hline
Id & User Action & Condition & System Response  \bl 
1C & Type a wanted content in the "Empty Field" field and press OK button. & The "Print Flow Description" and "Print Use case description" fields is configured to "None" & The preferences configuration is changed. [FR-TARGET-0231] \bl 
2C & Go to "On The Fly Generation" and select each test case from the list. & - & In all test cases, the fields that are empty in the Use Case Document are filled with this content defined by the user. [FR-TARGET-0235, ,FR-TARGET-0237] \bl 
\end{tabular}
\subsubsection*{Scenario UC-130-SC-5}
\begin{tabular}{p{1in}p{4in}}
{\bf Description:} & Changing "Test Case Id" parameter to a fixed value. \\
{\bf From steps:} & 2M \\
{\bf To steps:} & END \\
\end{tabular}
 
\begin{tabular}{|p{0.8in}|p{1.6in}|p{1.6in}|p{1.6in}|}
\hline
Id & User Action & Condition & System Response  \bl 
1D & Type "<tc-featureid>-TESTE-<tc-id>" in the "Test Case Id" field and press OK button. & - & The preferences configuration is changed. [FR-TARGET-0231] \bl 
2D & Go to "On The Fly Generation" and select each test case from the list. & - & All Test Case Id fields are filled according the standard defined by the user. [FR-TARGET-0235, ,FR-TARGET-0237] \bl 
\end{tabular}
\subsection*{Use case UC-169}
\begin{itemize}
\item {\bf Name: }Basic Test Suite Generation 
\item {\bf Description: }Generating Test Cases through Basic Generation.
	  
\end{itemize}
\subsubsection*{Scenario SC-01}
\begin{tabular}{p{1in}p{4in}}
{\bf Description:} & Generating Test Cases through Basic Generation.
	     \\
{\bf From steps:} & START \\
{\bf To steps:} & END \\
\end{tabular}
 
\begin{tabular}{|p{0.8in}|p{1.6in}|p{1.6in}|p{1.6in}|}
\hline
Id & User Action & Condition & System Response  \bl 
1M & Choose the "Tools" option in the menu bar.  & The TaRGeT is already started up. There is an already
		created project with at least one imported use case without
		errors. [FR-TARGET-0100] & A drop down menu is displayed with "Search",
		"Preferences" and "Basic Generation" options. The "Basic
		Generation" option is available.  \bl 
\end{tabular}
\subsection*{Use case UC-110}
\begin{itemize}
\item {\bf Name: }Generating Test Suites through the On The Fly Generation
\item {\bf Description: }Generating Test Cases through the On The Fly Generation.
\end{itemize}
\subsubsection*{Scenario UC-110-SC-3}
\begin{tabular}{p{1in}p{4in}}
{\bf Description:} & Generating tests without a template. \\
{\bf From steps:} & 1B \\
{\bf To steps:} & END \\
\end{tabular}
 
\begin{tabular}{|p{0.8in}|p{1.6in}|p{1.6in}|p{1.6in}|}
\hline
Id & User Action & Condition & System Response  \bl 
1C & Press "No" button. & - & The focus comes back to the work area. \bl 
\end{tabular}
\subsubsection*{Scenario UC-110-SC-5}
\begin{tabular}{p{1in}p{4in}}
{\bf Description:} & Importing an invalid test suite template. \\
{\bf From steps:} & ADV-06-2-2 \\
{\bf To steps:} & END \\
\end{tabular}
 
\begin{tabular}{|p{0.8in}|p{1.6in}|p{1.6in}|p{1.6in}|}
\hline
Id & User Action & Condition & System Response  \bl 
1D & Choose a .xls template and press "Open". & The .xls template is not a valid Test Central template. & A message is displayed informing that the .xls file is invalid and that the template information will not be imported. \bl 
\end{tabular}
\end{document}