%This Latex file is machine-generated by the Hephaestus

\documentclass[a4paper,11pt]{article}
\newcommand{\bl}{\\ \hline}
\title{Use Case Model Generated from TaRGeT}
\begin{document}
\maketitle
\section*{Use cases}
\subsection*{Use case UC_05}
\begin{itemize}
\item {\bf Name: }Creating a Project from Blank Work Area
\item {\bf Description: }Create a TaRGeT project.
\end{itemize}
\subsubsection*{Scenario UC_05_SC_1}
\begin{tabular}{p{1in}p{4in}}
{\bf Description:} & Create a new project. \\
{\bf From steps:} & START \\
{\bf To steps:} & END \\
\end{tabular}
 
\begin{tabular}{|p{0.8in}|p{1.6in}|p{1.6in}|p{1.6in}|}
\hline
Id & User Action & Condition & System Response  \bl 
1M & Go to "File" option in the menu bar. & The TaRGeT is already started up. No project is opened. & The "New Project" option is enabled in the drop down menu. The Close Project option is disabled. \bl 
2M & Select the "New Project" option. [FR_TARGET_0020] & - & The "New Project" screen is displayed. The default project name is displayed in the "Project Name" field. The default project location is displayed in the "Destination Folder" field. \bl 
3M & Click on "Finish" button. & - & A progress bar is displayed. A new project work area is displayed. The project folder is created in the specified location. [FR_TARGET_0110, FR_TARGET_0010] \bl 
\end{tabular}
\subsubsection*{Scenario UC_05_SC_2}
\begin{tabular}{p{1in}p{4in}}
{\bf Description:} & Type a valid project name. \\
{\bf From steps:} & 2M \\
{\bf To steps:} & 3M \\
\end{tabular}
 
\begin{tabular}{|p{0.8in}|p{1.6in}|p{1.6in}|p{1.6in}|}
\hline
Id & User Action & Condition & System Response  \bl 
1A & Type a valid project name. & - & The "Project Name" field is filled. [FR_TARGET_0025] \bl 
\end{tabular}
\subsubsection*{Scenario UC_05_SC_3}
\begin{tabular}{p{1in}p{4in}}
{\bf Description:} & Cancel the project creation. \\
{\bf From steps:} & 2M,1A,1C \\
{\bf To steps:} & END \\
\end{tabular}
 
\begin{tabular}{|p{0.8in}|p{1.6in}|p{1.6in}|p{1.6in}|}
\hline
Id & User Action & Condition & System Response  \bl 
1B & Click on the "Cancel" button. & - & The tool goes back to the window it was before the project creation. \bl 
\end{tabular}
\subsubsection*{Scenario UC_05_SC_4}
\begin{tabular}{p{1in}p{4in}}
{\bf Description:} & Browse destination folder. \\
{\bf From steps:} & 2M,1A \\
{\bf To steps:} & 3M \\
\end{tabular}
 
\begin{tabular}{|p{0.8in}|p{1.6in}|p{1.6in}|p{1.6in}|}
\hline
Id & User Action & Condition & System Response  \bl 
1C & Browse and select the location in which the project shall be created. & - & The "Destination Folder" field is filled. \bl 
\end{tabular}
\subsubsection*{Scenario UC_05_SC_5}
\begin{tabular}{p{1in}p{4in}}
{\bf Description:} & Create the project using shortcut. \\
{\bf From steps:} & START \\
{\bf To steps:} & END \\
\end{tabular}
 
\begin{tabular}{|p{0.8in}|p{1.6in}|p{1.6in}|p{1.6in}|}
\hline
Id & User Action & Condition & System Response  \bl 
1H & Press CTRL+N. & - & The "New Project" screen is displayed. The default project name is displayed in the "Project Name" field. The default project location is displayed in the "Destination Folder" field. [FR_TARGET_0020] \bl 
2H & Click on "Finish" button. & - & A progress bar is displayed. A new project work area is displayed. The project folder is created in the specified location. [FR_TARGET_0110, FR_TARGET_0010] \bl 
\end{tabular}
\subsubsection*{Scenario UC_05_SC_6}
\begin{tabular}{p{1in}p{4in}}
{\bf Description:} & Type an invalid project name. \\
{\bf From steps:} & 2M \\
{\bf To steps:} & END \\
\end{tabular}
 
\begin{tabular}{|p{0.8in}|p{1.6in}|p{1.6in}|p{1.6in}|}
\hline
Id & User Action & Condition & System Response  \bl 
1D & Type an invalid project name (e.g. with one of the following characters ":\/<>?*). & - & A message is displayed indicating that some characters are not allowed. The "Finish" button is disabled. [FR_TARGET_0025] \bl 
\end{tabular}
\subsubsection*{Scenario UC_05_SC_7}
\begin{tabular}{p{1in}p{4in}}
{\bf Description:} & No project name is typed. \\
{\bf From steps:} & 2M \\
{\bf To steps:} & END \\
\end{tabular}
 
\begin{tabular}{|p{0.8in}|p{1.6in}|p{1.6in}|p{1.6in}|}
\hline
Id & User Action & Condition & System Response  \bl 
1E & Type a blank project name. [FR_TARGET_0025] & - & A message is displayed indicating that a name must be specified. The "Finish" button is disabled. \bl 
\end{tabular}
\subsubsection*{Scenario UC_05_SC_8}
\begin{tabular}{p{1in}p{4in}}
{\bf Description:} & Project directory already exists. \\
{\bf From steps:} & 2M \\
{\bf To steps:} & END \\
\end{tabular}
 
\begin{tabular}{|p{0.8in}|p{1.6in}|p{1.6in}|p{1.6in}|}
\hline
Id & User Action & Condition & System Response  \bl 
1F & Type a valid project name. [FR_TARGET_0025] & - & The "Project Name" field is filled. \bl 
2F & Browse and select the location in which there is a directory with the same name of the project. & - & The "Destination Folder" field is filled. \bl 
3F & Click on "Finish" button. & - & A dialog is displayed indicating that the project folder already exists. The dialog asks if the user you want to proceed and erase all content of the folder. [FR_TARGET_0020] \bl 
4F & Click on "Yes" button. & - & All the folder contents is deleted and the new project is created. \bl 
\end{tabular}
\subsubsection*{Scenario UC_05_SC_9}
\begin{tabular}{p{1in}p{4in}}
{\bf Description:} & Do not delete all folder contents. \\
{\bf From steps:} & 3F \\
{\bf To steps:} & END \\
\end{tabular}
 
\begin{tabular}{|p{0.8in}|p{1.6in}|p{1.6in}|p{1.6in}|}
\hline
Id & User Action & Condition & System Response  \bl 
1G & Click on "No" button. & - & The focus goes back to the "New Project" window. \bl 
\end{tabular}
\subsection*{Use case UC_20}
\begin{itemize}
\item {\bf Name: }Opening an Existing Project from Blank Work Area
\item {\bf Description: }Open an existing project.
\end{itemize}
\subsubsection*{Scenario UC_20_SC_1}
\begin{tabular}{p{1in}p{4in}}
{\bf Description:} & Open an existing project. \\
{\bf From steps:} & START \\
{\bf To steps:} & END \\
\end{tabular}
 
\begin{tabular}{|p{0.8in}|p{1.6in}|p{1.6in}|p{1.6in}|}
\hline
Id & User Action & Condition & System Response  \bl 
1M & Go to "File" option in the menu bar. & TaRGeT is already started up. No project is opened. & The "Open Project" option is enabled. \bl 
2M & Select the "Open Project" option. [FR_TARGET_0040] & - & The "Open Project" window is displayed. No project is selected. The "Finish" button is disabled. \bl 
3M & Browse and select an existing project. & - & The "Project Name" field is now filled. The "Finish" button is enabled. \bl 
4M & Click on "Finish" button. & - & A progress bar is displayed indicating that the selected existing project is being opened. After, the existing project is opened. The project work area is displayed. [FR_TARGET_0110] \bl 
\end{tabular}
\subsubsection*{Scenario UC_20_SC_2}
\begin{tabular}{p{1in}p{4in}}
{\bf Description:} & Cancel the project opening. \\
{\bf From steps:} & 2M,3M \\
{\bf To steps:} & END \\
\end{tabular}
 
\begin{tabular}{|p{0.8in}|p{1.6in}|p{1.6in}|p{1.6in}|}
\hline
Id & User Action & Condition & System Response  \bl 
1A & Click on "Cancel" button. & - & The tool goes to the blank work area. No project is opened. [FR_TARGET_0225] \bl 
\end{tabular}
\subsubsection*{Scenario UC_20_SC_3}
\begin{tabular}{p{1in}p{4in}}
{\bf Description:} & Open using shortcut. \\
{\bf From steps:} & START \\
{\bf To steps:} & 3M \\
\end{tabular}
 
\begin{tabular}{|p{0.8in}|p{1.6in}|p{1.6in}|p{1.6in}|}
\hline
Id & User Action & Condition & System Response  \bl 
1B & Press CTRL+O. & - & The "Open Project" window is displayed. No project is selected. The "Finish" button is disabled. [FR_TARGET_0040] \bl 
\end{tabular}
\subsection*{Use case UC_25}
\begin{itemize}
\item {\bf Name: }Opening an Existing Project from an Already Opened Project
\item {\bf Description: }A project is already opened.
\end{itemize}
\subsubsection*{Scenario UC_25_SC_1}
\begin{tabular}{p{1in}p{4in}}
{\bf Description:} & Open an existing project from an already opened project. \\
{\bf From steps:} & START \\
{\bf To steps:} & UC_20#3M \\
\end{tabular}
 
\begin{tabular}{|p{0.8in}|p{1.6in}|p{1.6in}|p{1.6in}|}
\hline
Id & User Action & Condition & System Response  \bl 
1M & Go to "File" option in the menu bar. & TaRGeT is already started up. A project is already opened. & The "Open Project" option is enabled. \bl 
2M & Select the "Open Project" option. [FR_TARGET_0040] & - & A dialog box is displayed informing that the current project will be closed in order to open another one. \bl 
3M & Click on "Yes" button. & - & The "Open Project" screen is displayed. \bl 
\end{tabular}
\subsubsection*{Scenario UC_25_SC_2}
\begin{tabular}{p{1in}p{4in}}
{\bf Description:} & Cancel the project opening. \\
{\bf From steps:} & 2M \\
{\bf To steps:} & END \\
\end{tabular}
 
\begin{tabular}{|p{0.8in}|p{1.6in}|p{1.6in}|p{1.6in}|}
\hline
Id & User Action & Condition & System Response  \bl 
1A & Click on "No" button. & - & The dialog box is closed and the current project work area is displayed. \bl 
\end{tabular}
\subsection*{Use case UC_26}
\begin{itemize}
\item {\bf Name: }Refreshing a Project
\item {\bf Description: }Refresh a project.
\end{itemize}
\subsubsection*{Scenario UC_26_SC_1}
\begin{tabular}{p{1in}p{4in}}
{\bf Description:} & Refresh a project. \\
{\bf From steps:} & START \\
{\bf To steps:} & UC_90#1M \\
\end{tabular}
 
\begin{tabular}{|p{0.8in}|p{1.6in}|p{1.6in}|p{1.6in}|}
\hline
Id & User Action & Condition & System Response  \bl 
1M & Go to "Project" option in the menu bar. & - & A menu with two options shall appear:  "Refresh Automatically" and "Refresh Project". "Refresh Automatically" option is checked. [FR_TARGET_0015] \bl 
\end{tabular}
\subsubsection*{Scenario UC_26_SC_2}
\begin{tabular}{p{1in}p{4in}}
{\bf Description:} & "Refresh automatically" option disabled. \\
{\bf From steps:} & 1M \\
{\bf To steps:} & END \\
\end{tabular}
 
\begin{tabular}{|p{0.8in}|p{1.6in}|p{1.6in}|p{1.6in}|}
\hline
Id & User Action & Condition & System Response  \bl 
1A & Click in "Refresh Automatically" option. & TaRGeT is already started up. A project is already opened. & "Refresh Automatically" option is not checked. \bl 
2A & Open an already imported Use Case Document with the Microsoft Word (outside the TaRGeT). & There is at least one imported Use Case Document. [FR_TARGET_0100] & The document is opened in the default viewer outside the TaRGeT. \bl 
3A & Modify any field in Use Case Document and save the document. & - & The document is modified. \bl 
4A & Go back to TaRGeT work area. & - & The project is not refreshed. \bl 
5A & Go to "Project" in the menu bar and select "Refresh project" option. & - & A progress bar is displayed while the project is being refreshed. [FR_TARGET_0110, FR_TARGET_0015] \bl 
\end{tabular}
\subsubsection*{Scenario UC_26_SC_3}
\begin{tabular}{p{1in}p{4in}}
{\bf Description:} & "Refresh automatically" option enabled. \\
{\bf From steps:} & 1M \\
{\bf To steps:} & END \\
\end{tabular}
 
\begin{tabular}{|p{0.8in}|p{1.6in}|p{1.6in}|p{1.6in}|}
\hline
Id & User Action & Condition & System Response  \bl 
1B & Go to "Project" in the menu bar and select "Refresh project" option. & - & The project is refreshed. [FR_TARGET_0015] \bl 
\end{tabular}
\subsubsection*{Scenario UC_26_SC_4}
\begin{tabular}{p{1in}p{4in}}
{\bf Description:} & Using the Refresh shortcut. \\
{\bf From steps:} & 3A \\
{\bf To steps:} & END \\
\end{tabular}
 
\begin{tabular}{|p{0.8in}|p{1.6in}|p{1.6in}|p{1.6in}|}
\hline
Id & User Action & Condition & System Response  \bl 
1C & Press CTRL+R. & - & A progress bar is displayed while the project is being refreshed. [FR_TARGET_0110, FR_TARGET_0015] \bl 
\end{tabular}
\subsection*{Use case UC_30}
\begin{itemize}
\item {\bf Name: }Importing Valid Use Case Documents
\item {\bf Description: }Import Use Case Documents.
\end{itemize}
\subsubsection*{Scenario UC_30_SC_1}
\begin{tabular}{p{1in}p{4in}}
{\bf Description:} & Import Use Case Documents. \\
{\bf From steps:} & START \\
{\bf To steps:} & END \\
\end{tabular}
 
\begin{tabular}{|p{0.8in}|p{1.6in}|p{1.6in}|p{1.6in}|}
\hline
Id & User Action & Condition & System Response  \bl 
1M & Go to "Artifacts" option in the menu bar. & TaRGeT is started up. A project is already opened. & The "Import Use Case Documents" option is enabled. \bl 
2M & Select "Import Use Case Documents" option. [FR_TARGET_0100] & - & The "Import Documents" window is displayed. The "Finish" button is disabled. \bl 
3M & Select "Add Document" button and browse a valid Use Case Document. & No document is imported. & 
		      The selected document path is inserted in the "Documents To Import" list. The "Finish" button is enabled. @addDocument
		     \bl 
4M & Click on "Finish" button. & - & A progress bar is displayed indicating that the document(s) is (are) being processed and imported.[FR_TARGET_0110] \bl 
5M & Wait the progress bar. & - & The imported document(s) is (are) placed in a specific folder and displayed in the "Artifacts" View. The imported use cases are displayed in the "Use Cases" View. @endOfImportDocument [FR_TARGET_0010, FR_TARGET_0015] \bl 
\end{tabular}
\subsubsection*{Scenario UC_30_SC_2}
\begin{tabular}{p{1in}p{4in}}
{\bf Description:} & Import another document. \\
{\bf From steps:} & 3M \\
{\bf To steps:} & 4M \\
\end{tabular}
 
\begin{tabular}{|p{0.8in}|p{1.6in}|p{1.6in}|p{1.6in}|}
\hline
Id & User Action & Condition & System Response  \bl 
1A & Select "Add Document" button and browse another valid Use Case Document. The document must not have duplicated feature ids, when compared with the already browsed document. & Another document has to be inserted on the "Documents To Import" list. & The selected document path is inserted in the "Documents To Import" list. \bl 
\end{tabular}
\subsubsection*{Scenario UC_30_SC_3}
\begin{tabular}{p{1in}p{4in}}
{\bf Description:} & Cancel the importing. \\
{\bf From steps:} & 2M,3M \\
{\bf To steps:} & END \\
\end{tabular}
 
\begin{tabular}{|p{0.8in}|p{1.6in}|p{1.6in}|p{1.6in}|}
\hline
Id & User Action & Condition & System Response  \bl 
1B & Click on "Cancel" button. & - & The focus goes back to the project work area. \bl 
\end{tabular}
\subsubsection*{Scenario UC_30_SC_4}
\begin{tabular}{p{1in}p{4in}}
{\bf Description:} & Import through shortcut. \\
{\bf From steps:} & START \\
{\bf To steps:} & 3M \\
\end{tabular}
 
\begin{tabular}{|p{0.8in}|p{1.6in}|p{1.6in}|p{1.6in}|}
\hline
Id & User Action & Condition & System Response  \bl 
1C & Press CTRL+I. & - & The "Import Documents" window is displayed. The "Finish" button is disabled. \bl 
\end{tabular}
\subsection*{Use case UC_35}
\begin{itemize}
\item {\bf Name: }Importing a Use Case Document With Duplicated Feature ID
\item {\bf Description: }Use Case Document has duplicated feature ID.
\end{itemize}
\subsubsection*{Scenario UC_35_SC_1}
\begin{tabular}{p{1in}p{4in}}
{\bf Description:} & Imported Use Case Document has duplicated feature ID. \\
{\bf From steps:} & UC_30#2M \\
{\bf To steps:} & END \\
\end{tabular}
 
\begin{tabular}{|p{0.8in}|p{1.6in}|p{1.6in}|p{1.6in}|}
\hline
Id & User Action & Condition & System Response  \bl 
1M & Select "Add Document" button and browse a Use Case Document with duplicated feature ID. This means that the document contains at least two features with the same ID. & There is no imported document. & The selected document path is inserted in the "Documents To Import" list. The "Finish" button is enabled. \bl 
2M & Click on "Finish" button. & - & A progress bar is displayed indicating that the document is being processed. [FR_TARGET_0110] \bl 
3M & Wait the progress bar. & - & An error dialog is displayed indicating that the document that has duplicated feature id [FR_TARGET_0125] \bl 
4M & Click on "OK" button. & - & The selected document(s) will not be imported. The focus goes back to "Import Documents" window. \bl 
\end{tabular}
\subsection*{Use case UC_40}
\begin{itemize}
\item {\bf Name: }Importing a Use Case Document With Duplicated Use Case ID
\item {\bf Description: }Use Case Document has duplicated use case ID.
\end{itemize}
\subsubsection*{Scenario UC_40_SC_1}
\begin{tabular}{p{1in}p{4in}}
{\bf Description:} & Imported Use Case Document has duplicated use case ID. \\
{\bf From steps:} & UC_30#2M \\
{\bf To steps:} & END \\
\end{tabular}
 
\begin{tabular}{|p{0.8in}|p{1.6in}|p{1.6in}|p{1.6in}|}
\hline
Id & User Action & Condition & System Response  \bl 
1M & Select "Add Document" button and browse a Use Case Document with duplicated use case ID. This means that the document contains a feature that has at least two use cases with the same ID. & No document is imported. & The selected document path is inserted in the "Documents To Import" list. The "Finish" button is enabled. \bl 
2M & Click on "Finish" button. & - & A progress bar is displayed indicating that the document is being processed. [FR_TARGET_0110] \bl 
3M & Wait the progress bar. & - & An error dialog is displayed indicating the document that has duplicated use case ID. [FR_TARGET_0125] \bl 
4M & Click on "OK" button. & - & The selected document(s) will not be imported. The focus goes back to "Import Documents" window. \bl 
\end{tabular}
\subsection*{Use case UC_45}
\begin{itemize}
\item {\bf Name: }Importing a Use Case Document With Invalid Structure
\item {\bf Description: }Use Case Document has invalid structure.
\end{itemize}
\subsubsection*{Scenario UC_45_SC_1}
\begin{tabular}{p{1in}p{4in}}
{\bf Description:} & Imported Use Case Document has an invalid structure. \\
{\bf From steps:} & UC_30#2M \\
{\bf To steps:} & END \\
\end{tabular}
 
\begin{tabular}{|p{0.8in}|p{1.6in}|p{1.6in}|p{1.6in}|}
\hline
Id & User Action & Condition & System Response  \bl 
1M & Select "Add Document" button and browse a Use Case Document with invalid structure. & No document is imported. & The selected document path is inserted in the "Documents To Import" list. The "Finish" button is enabled. \bl 
2M & Click on "Finish" button. & - & A progress bar is displayed indicating that the document is being processed. [FR_TARGET_0110] \bl 
3M & Wait the progress bar. & - & An error dialog is displayed indicating the document that has an error. [FR_TARGET_0125] \bl 
4M & Click on "OK" button. & - & The selected document(s) will not be imported. The focus goes back to "Import Documents" window. \bl 
\end{tabular}
\subsubsection*{Scenario UC_45_SC_2}
\begin{tabular}{p{1in}p{4in}}
{\bf Description:} & Import another document. \\
{\bf From steps:} & 1M \\
{\bf To steps:} & 2M \\
\end{tabular}
 
\begin{tabular}{|p{0.8in}|p{1.6in}|p{1.6in}|p{1.6in}|}
\hline
Id & User Action & Condition & System Response  \bl 
1A & Click on "Add Document" button and browse another valid Use Case Document. & Another document has to be imported. & The selected document path is inserted in the "Documents To Import" list. \bl 
\end{tabular}
\subsubsection*{Scenario UC_45_SC_3}
\begin{tabular}{p{1in}p{4in}}
{\bf Description:} & Cancel the importing. \\
{\bf From steps:} & 4M \\
{\bf To steps:} & END \\
\end{tabular}
 
\begin{tabular}{|p{0.8in}|p{1.6in}|p{1.6in}|p{1.6in}|}
\hline
Id & User Action & Condition & System Response  \bl 
1B & Click on "Cancel" button. & - & The focus goes back to the project work area. \bl 
\end{tabular}
\subsection*{Use case UC_50}
\begin{itemize}
\item {\bf Name: }Importing a Use Case Document with an Empty Mandatory Field
\item {\bf Description: }Use Case Document has an empty mandatory field.
\end{itemize}
\subsubsection*{Scenario UC_50_SC_1}
\begin{tabular}{p{1in}p{4in}}
{\bf Description:} & Selected Use Case Document has an empty mandatory field. \\
{\bf From steps:} & UC_30#2M \\
{\bf To steps:} & END \\
\end{tabular}
 
\begin{tabular}{|p{0.8in}|p{1.6in}|p{1.6in}|p{1.6in}|}
\hline
Id & User Action & Condition & System Response  \bl 
1M & Select "Add Document" button and browse a Use Case Document with an empty mandatory field. & No document is imported. & The selected document path is inserted in the "Documents To Import" list. The "Finish" button is enabled. \bl 
2M & Click on "Finish" button. & - & An error dialog is displayed indicating that the document has an error in the xml structure. [FR_TARGET_0125] \bl 
3M & Click on "OK" button. & - & The selected document(s) will not be imported. The focus goes back to "Import Documents" window. \bl 
\end{tabular}
\subsection*{Use case UC_55}
\begin{itemize}
\item {\bf Name: }Importing a Use Case Document With Invalid Step ID References
\item {\bf Description: }Use Case Document has invalid id references.
\end{itemize}
\subsubsection*{Scenario UC_55_SC_1}
\begin{tabular}{p{1in}p{4in}}
{\bf Description:} & The selected Use Case Document contains invalid id references. \\
{\bf From steps:} & UC_30#2M \\
{\bf To steps:} & END \\
\end{tabular}
 
\begin{tabular}{|p{0.8in}|p{1.6in}|p{1.6in}|p{1.6in}|}
\hline
Id & User Action & Condition & System Response  \bl 
1M & Select "Add Document" button and browse a Use Case Document with invalid ID references. & No document is imported. & The selected document path is inserted in the "Documents To Import" list. The "Finish" button is enabled. \bl 
2M & Click on "Finish" button. & - & A progress bar is displayed indicating that the document is being processed. [FR_TARGET_0110] \bl 
3M & Wait the progress bar. & - & A warning message is displayed indicating that the document contains errors. \bl 
4M & Click on "OK" button. & - & The imported document is placed in the documents folder and is displayed in the "Artifacts" View. The imported use cases are displayed in the "Use Cases" View. A special icon indicates the use cases with invalid references. Also the error view displays details about invalid id references and/or duplicated ids. [FR_TARGET_0010, FR_TARGET_0120] \bl 
\end{tabular}
\subsubsection*{Scenario UC_55_SC_2}
\begin{tabular}{p{1in}p{4in}}
{\bf Description:} & Import another valid document. \\
{\bf From steps:} & 1M \\
{\bf To steps:} & 2M \\
\end{tabular}
 
\begin{tabular}{|p{0.8in}|p{1.6in}|p{1.6in}|p{1.6in}|}
\hline
Id & User Action & Condition & System Response  \bl 
1A & Select "Add Document" button and browse another valid Use Case Document. & Another document has to be imported. & The selected document path is inserted in the "Documents To Import" list. \bl 
\end{tabular}
\subsection*{Use case UC_63}
\begin{itemize}
\item {\bf Name: }Importing a Use Case Document with a Feature Already Imported.
\item {\bf Description: }The documents have at least one feature with same name and ID. Another Use Case Document is already imported.
\end{itemize}
\subsubsection*{Scenario UC_63_SC_1}
\begin{tabular}{p{1in}p{4in}}
{\bf Description:} & A Use Case Document is already imported. The Use Case Documents have at least one common feature with same ID and name. \\
{\bf From steps:} & UC_30#2M \\
{\bf To steps:} & END \\
\end{tabular}
 
\begin{tabular}{|p{0.8in}|p{1.6in}|p{1.6in}|p{1.6in}|}
\hline
Id & User Action & Condition & System Response  \bl 
1M & Select "Add Document" button and browse a valid Use Case Document. The document and the already imported document have at least one common feature with same ID and name and the use cases IDs are different. & There is at least one valid document imported in the project. & The "Import Documents" window is displayed. The "Finish" button is enabled. \bl 
2M & Click on "Finish" button. & - & A progress bar is displayed indicating that the document(s) is (are) being processed and imported. [FR_TARGET_0110] \bl 
3M & Wait the progress bar. & - & The imported document is placed in a specific folder and displayed in the "Artifacts" view. The imported use cases are displayed in the "Use Cases" view. The feature icon in the "Use Cases" view shows that the feature was merged. [FR_TARGET_0010, FR_TARGET_0100] \bl 
\end{tabular}
\subsubsection*{Scenario UC_63_SC_2}
\begin{tabular}{p{1in}p{4in}}
{\bf Description:} & A Use Case Document is already imported. The Use Case Documents have at least one common feature with same ID and different names. \\
{\bf From steps:} & UC_30#2M \\
{\bf To steps:} & END \\
\end{tabular}
 
\begin{tabular}{|p{0.8in}|p{1.6in}|p{1.6in}|p{1.6in}|}
\hline
Id & User Action & Condition & System Response  \bl 
1A & Select "Add Document" button and browse a valid Use Case Document. The document and the already imported document have at least one common feature with same ID but different names and the use cases IDs are different. & There is at least one valid document imported in the project. & The "Import Documents" window is displayed. The "Finish" button is enabled. \bl 
2A & Click on "Finish" button. & - & A progress bar is displayed indicating that the document(s) is (are) being processed and imported. [FR_TARGET_0110] \bl 
3A & Wait the progress bar. & - & The imported document is placed in a specific folder and displayed in the "Artifacts" View. The imported use cases are displayed in the "Use Cases" View. The feature icon in the "Use Cases" view shows that the feature was merged. A new warning is displayed in the "Error" view indicating that the features have same ID but different names. [FR_TARGET_0010, FR_TARGET_0100,  FR_TARGET_0120] \bl 
\end{tabular}
\subsubsection*{Scenario UC_63_SC_3}
\begin{tabular}{p{1in}p{4in}}
{\bf Description:} & The document contains a feature with same ID of another feature that belongs to a previously imported document. \\
{\bf From steps:} & UC_30#2M \\
{\bf To steps:} & END \\
\end{tabular}
 
\begin{tabular}{|p{0.8in}|p{1.6in}|p{1.6in}|p{1.6in}|}
\hline
Id & User Action & Condition & System Response  \bl 
1B & Select "Add Document" button and browse a valid Use Case Document. The document and the already imported document have at least one common feature with same ID. The features have at least one use case with same ID. & There is at least one valid imported document. & The selected document path is inserted in the "Documents To Import" list. \bl 
2B & Click on "Finish" button. & - & A progress bar is displayed indicating that the document is being processed. [FR_TARGET_0110] \bl 
3B & Wait the progress bar. & - & An error dialog is displayed indicating the document that has an error. [FR_TARGET_0125] \bl 
4B & Click on "OK" button. & - & The selected document(s) will not be imported. The focus goes back to "Import Documents" window. \bl 
\end{tabular}
\subsection*{Use case UC_70}
\begin{itemize}
\item {\bf Name: }Viewing Valid Use Case in HTML Format
\item {\bf Description: }Viewing Use Case Documents in HTML format.
\end{itemize}
\subsubsection*{Scenario UC_70_SC_1}
\begin{tabular}{p{1in}p{4in}}
{\bf Description:} & View valid Use Case Documents in HTML format in the main panel, by double-clicking the use case. \\
{\bf From steps:} & START \\
{\bf To steps:} & END \\
\end{tabular}
 
\begin{tabular}{|p{0.8in}|p{1.6in}|p{1.6in}|p{1.6in}|}
\hline
Id & User Action & Condition & System Response  \bl 
1M & Double-click on a use case in the "Use Cases" view. & A project is opened in TaRGeT. The project has at least one imported valid Use Case Document. & The selected Use Case is displayed in HTML format in the main panel. [FR_TARGET_0130] \bl 
\end{tabular}
\subsubsection*{Scenario UC_70_SC_2}
\begin{tabular}{p{1in}p{4in}}
{\bf Description:} & View Use Case Documents in HTML format in the main panel, by right-clicking the use case. \\
{\bf From steps:} & START \\
{\bf To steps:} & END \\
\end{tabular}
 
\begin{tabular}{|p{0.8in}|p{1.6in}|p{1.6in}|p{1.6in}|}
\hline
Id & User Action & Condition & System Response  \bl 
1A & Right-click on a use case in the "Use Cases" view. & A project is opened in TaRGeT. The project has at least one imported valid Use Case Document. & A drop down menu is displayed with two options. \bl 
2A & Select "Open in default view" option. & - & The selected Use Case is displayed in HTML format in the main panel. [FR_TARGET_0130] \bl 
\end{tabular}
\subsubsection*{Scenario UC_70_SC_3}
\begin{tabular}{p{1in}p{4in}}
{\bf Description:} & View Use Case Documents in HTML format in the default browser, by right-clicking the use case. \\
{\bf From steps:} & START \\
{\bf To steps:} & END \\
\end{tabular}
 
\begin{tabular}{|p{0.8in}|p{1.6in}|p{1.6in}|p{1.6in}|}
\hline
Id & User Action & Condition & System Response  \bl 
1B & Right-click on a use case in the "Use Cases" view. & A project is opened in TaRGeT. The project has at least one imported valid Use Case Document. & A drop down menu is displayed with two options. \bl 
2B & Select "Open with default browser" option. & - & The selected Use Case is displayed in HTML format in the default browser. [FR_TARGET_0130] \bl 
\end{tabular}
\subsection*{Use case UC_71}
\begin{itemize}
\item {\bf Name: }Viewing Invalid Use Case in HTML Format
\item {\bf Description: }Viewing Use Case Documents in HTML format when they have some error.
\end{itemize}
\subsubsection*{Scenario UC_71_SC_1}
\begin{tabular}{p{1in}p{4in}}
{\bf Description:} & View Use Case Documents with duplicated step ID in HTML format in the main panel, by double-clicking the use case. \\
{\bf From steps:} & START \\
{\bf To steps:} & END \\
\end{tabular}
 
\begin{tabular}{|p{0.8in}|p{1.6in}|p{1.6in}|p{1.6in}|}
\hline
Id & User Action & Condition & System Response  \bl 
1M & Double-click on a use case with duplicated step ID in the "Use Cases" view. & A project is opened in TaRGeT. The project has at least one imported use case with duplicated step ID. & The selected Use Case is displayed in HTML format in the main panel. The duplicated steps IDs are highlighted.   [FR_TARGET_0130] \bl 
\end{tabular}
\subsubsection*{Scenario UC_71_SC_2}
\begin{tabular}{p{1in}p{4in}}
{\bf Description:} & View Use Case Documents with invalid step ID reference in HTML format in the main panel, by right-clicking the use case. \\
{\bf From steps:} & START \\
{\bf To steps:} & END \\
\end{tabular}
 
\begin{tabular}{|p{0.8in}|p{1.6in}|p{1.6in}|p{1.6in}|}
\hline
Id & User Action & Condition & System Response  \bl 
1A & Right-click on a use case with invalid step ID reference in the "Use Cases" view. & A project is opened in TaRGeT. The project has at least one imported use case with invalid step ID reference. & A drop down menu is displayed with two options. \bl 
2A & Select "Open in default view" option. & - & The selected Use Case is displayed in HTML format in the main panel. The invalid step ID reference is highlighted. [FR_TARGET_0130] \bl 
\end{tabular}
\subsubsection*{Scenario UC_71_SC_3}
\begin{tabular}{p{1in}p{4in}}
{\bf Description:} & View Use Case Documents with invalid step ID reference in HTML format in the default browser, by right-clicking the use case. \\
{\bf From steps:} & START \\
{\bf To steps:} & END \\
\end{tabular}
 
\begin{tabular}{|p{0.8in}|p{1.6in}|p{1.6in}|p{1.6in}|}
\hline
Id & User Action & Condition & System Response  \bl 
1B & Right-click on a use case with invalid step ID reference in the "Use Cases" view. & A project is opened in TaRGeT. The project has at least one imported use case with invalid step ID reference. & A drop down menu is displayed with two options. \bl 
2B & Select "Open with default browser" option. & - & The selected Use Case is displayed in HTML format in the default browser. The invalid step ID reference is highlighted.  [FR_TARGET_0130] \bl 
\end{tabular}
\subsection*{Use case UC_80}
\begin{itemize}
\item {\bf Name: }Renaming a Document
\item {\bf Description: }Renaming Use Case or Test Suite Documents in TaRGeT.
\end{itemize}
\subsubsection*{Scenario UC_80_SC_1}
\begin{tabular}{p{1in}p{4in}}
{\bf Description:} & Rename a test suite document. \\
{\bf From steps:} & START \\
{\bf To steps:} & END \\
\end{tabular}
 
\begin{tabular}{|p{0.8in}|p{1.6in}|p{1.6in}|p{1.6in}|}
\hline
Id & User Action & Condition & System Response  \bl 
1M & Right-click on a generated test suite document in the "Artifacts" view. & A project is opened in TaRGeT. The project has at least one already generated test suite. & A drop down menu is displayed. \bl 
2M & Click on "Rename" option in the drop down menu. [FR_TARGET_0001] & - & A dialog is popped up with a text field. \bl 
3M & Change the document name to a valid name and click on "OK" button. [FR_TARGET_0025] & - & The document is renamed. \bl 
\end{tabular}
\subsubsection*{Scenario UC_80_SC_2}
\begin{tabular}{p{1in}p{4in}}
{\bf Description:} & Rename a Use Case Document. \\
{\bf From steps:} & START \\
{\bf To steps:} & END \\
\end{tabular}
 
\begin{tabular}{|p{0.8in}|p{1.6in}|p{1.6in}|p{1.6in}|}
\hline
Id & User Action & Condition & System Response  \bl 
1A & Right-click on a Use Case Document in the "Artifacts" view. & A project is opened in TaRGeT. The project has at least one imported Use Case Document. & A drop down menu is displayed. \bl 
2A & Click on "Rename" option in the drop down menu. [FR_TARGET_0001] & - & A dialog is popped up with a text field. \bl 
3A & Change the document name to a valid name and click on "OK" button. [FR_TARGET_0025] & - & The document is renamed. \bl 
\end{tabular}
\subsubsection*{Scenario UC_80_SC_3}
\begin{tabular}{p{1in}p{4in}}
{\bf Description:} & Select to rename more than one document. \\
{\bf From steps:} & START \\
{\bf To steps:} & END \\
\end{tabular}
 
\begin{tabular}{|p{0.8in}|p{1.6in}|p{1.6in}|p{1.6in}|}
\hline
Id & User Action & Condition & System Response  \bl 
1B & In the "Artifacts" view, Select more than one document and right-click on the selection. [FR_TARGET_0001] & A project is opened in TaRGeT. The project has at least two documents (test suite or Use Case Documents). & A drop down menu is displayed. The "Rename" option is disabled. \bl 
\end{tabular}
\subsubsection*{Scenario UC_80_SC_4}
\begin{tabular}{p{1in}p{4in}}
{\bf Description:} & Type an invalid document name. \\
{\bf From steps:} & 2M \\
{\bf To steps:} & END \\
\end{tabular}
 
\begin{tabular}{|p{0.8in}|p{1.6in}|p{1.6in}|p{1.6in}|}
\hline
Id & User Action & Condition & System Response  \bl 
1C & Try to change the document name to an invalid name (e.g. a name containing "<" or "*" characters). [FR_TARGET_0025] & - & A message is displayed indicating that the name is invalid. The "OK" button is disabled. \bl 
2C & Click on "Cancel" button. & - & The focus goes back to the main window. \bl 
\end{tabular}
\subsubsection*{Scenario UC_80_SC_5}
\begin{tabular}{p{1in}p{4in}}
{\bf Description:} & Type an invalid document name and retype a valid name. \\
{\bf From steps:} & 2M \\
{\bf To steps:} & END \\
\end{tabular}
 
\begin{tabular}{|p{0.8in}|p{1.6in}|p{1.6in}|p{1.6in}|}
\hline
Id & User Action & Condition & System Response  \bl 
1D & Try to change the document name to an invalid name (e.g. a name containing "<" or "*" characters). [FR_TARGET_0025] & - & A message is displayed indicating that the name is invalid. The "OK" button is disabled. \bl 
2D & Erase the invalid name and type a valid name. & - & The OK button is enabled. \bl 
3D & Click on the OK button & - & The document is renamed. \bl 
\end{tabular}
\subsubsection*{Scenario UC_80_SC_6}
\begin{tabular}{p{1in}p{4in}}
{\bf Description:} & Cancel the rename. \\
{\bf From steps:} & 2D \\
{\bf To steps:} & END \\
\end{tabular}
 
\begin{tabular}{|p{0.8in}|p{1.6in}|p{1.6in}|p{1.6in}|}
\hline
Id & User Action & Condition & System Response  \bl 
1E & Click on "Cancel" button. & - & The focus goes back to the main window. \bl 
\end{tabular}
\subsubsection*{Scenario UC_80_SC_7}
\begin{tabular}{p{1in}p{4in}}
{\bf Description:} & Type a document name that is being used by another use case document. \\
{\bf From steps:} & START \\
{\bf To steps:} & 2C,2D \\
\end{tabular}
 
\begin{tabular}{|p{0.8in}|p{1.6in}|p{1.6in}|p{1.6in}|}
\hline
Id & User Action & Condition & System Response  \bl 
1F & Right-click on a Use Case Document in the "Artifacts" view. & A project is opened in TaRGeT. The project has at least two use cases. & A drop down menu is displayed. The "Rename" option is enabled. \bl 
2F & Click on "Rename" option in the drop down menu. [FR_TARGET_0001] & - & A dialog is popped up with a text field. \bl 
3F & Type a name that is already in use. & - & A message is displayed indicating that the name is already in use. The "OK" button is disabled. \bl 
\end{tabular}
\subsubsection*{Scenario UC_80_SC_8}
\begin{tabular}{p{1in}p{4in}}
{\bf Description:} & Type a test suite file name that is being used by another test suite file. \\
{\bf From steps:} & START \\
{\bf To steps:} & 2F \\
\end{tabular}
 
\begin{tabular}{|p{0.8in}|p{1.6in}|p{1.6in}|p{1.6in}|}
\hline
Id & User Action & Condition & System Response  \bl 
1G & Right-click on a test case document in the "Artifacts" view. & A project is opened in TaRGeT. The project has at least two test suites. & A drop down menu is displayed. The "Rename" option is enabled. \bl 
\end{tabular}
\subsubsection*{Scenario UC_80_SC_9}
\begin{tabular}{p{1in}p{4in}}
{\bf Description:} & Type an empty name. \\
{\bf From steps:} & 2M \\
{\bf To steps:} & 2C \\
\end{tabular}
 
\begin{tabular}{|p{0.8in}|p{1.6in}|p{1.6in}|p{1.6in}|}
\hline
Id & User Action & Condition & System Response  \bl 
1H & Type an empty name. [FR_TARGET_0025] & - & A message is displayed indicating that the name must not be empty. The "OK" button is disabled. \bl 
\end{tabular}
\subsection*{Use case UC_85}
\begin{itemize}
\item {\bf Name: }Deleting a Document
\item {\bf Description: }Deleting Use Cases or Test Suite Documents in TaRGeT.
\end{itemize}
\subsubsection*{Scenario UC_85_SC_1}
\begin{tabular}{p{1in}p{4in}}
{\bf Description:} & Delete a test suite document. \\
{\bf From steps:} & START \\
{\bf To steps:} & END \\
\end{tabular}
 
\begin{tabular}{|p{0.8in}|p{1.6in}|p{1.6in}|p{1.6in}|}
\hline
Id & User Action & Condition & System Response  \bl 
1M & Right-click on a generated test suite document in the "Artifacts" view. & A project is opened in TaRGeT. The project has at least one already generated test suite. & A drop down menu is displayed. \bl 
2M & Click on "Delete" option in the drop down menu. [FR_TARGET_0003] & - & A dialog is displayed asking a confirmation. \bl 
3M & Click on "Yes" button. & - & The document(s) is (are) deleted. The project work area is refreshed. [FR_TARGET_0015] \bl 
\end{tabular}
\subsubsection*{Scenario UC_85_SC_2}
\begin{tabular}{p{1in}p{4in}}
{\bf Description:} & Delete a Use Case Document that is not referred by another document. \\
{\bf From steps:} & START \\
{\bf To steps:} & END \\
\end{tabular}
 
\begin{tabular}{|p{0.8in}|p{1.6in}|p{1.6in}|p{1.6in}|}
\hline
Id & User Action & Condition & System Response  \bl 
1B & Right-click on a Use Case Document in the "Artifacts" view. & A project is opened in TaRGeT. The project has at least one imported Use Case Document. The imported documents do not refer to a use case of another imported document. & A drop down menu is displayed. \bl 
2B & Click on "Delete" option in the drop down menu. [FR_TARGET_0003] & - & A dialog is displayed asking a confirmation. \bl 
3B & Click on "Yes" button. & - & The document(s) is(are) deleted. The project work area is refreshed. [FR_TARGET_0015] \bl 
\end{tabular}
\subsubsection*{Scenario UC_85_SC_3}
\begin{tabular}{p{1in}p{4in}}
{\bf Description:} & Delete a Use Case Document that is referred by another document. \\
{\bf From steps:} & START \\
{\bf To steps:} & END \\
\end{tabular}
 
\begin{tabular}{|p{0.8in}|p{1.6in}|p{1.6in}|p{1.6in}|}
\hline
Id & User Action & Condition & System Response  \bl 
1C & Right-click on a Use Case Document in the "Artifacts" view. The selected document is referred by a use case of another imported document. & A project is opened in TaRGeT. The project has at least two imported Use Case Documents. A use case is referred by another use case from a different document. & A drop down menu is displayed. \bl 
2C & Click on "Delete" option in the drop down menu. [FR_TARGET_0003] & - & A dialog is displayed asking a confirmation and informing that the document is being referred by another document. \bl 
3C & Click on "Yes" button. & - & The document(s) is(are) deleted. The project work area is refreshed. Some invalid reference errors are displayed in "Error" view. [FR_TARGET_0015, FR_TARGET_0120] \bl 
\end{tabular}
\subsubsection*{Scenario UC_85_SC_4}
\begin{tabular}{p{1in}p{4in}}
{\bf Description:} & Delete more than one generated test suite. \\
{\bf From steps:} & START \\
{\bf To steps:} & 2M \\
\end{tabular}
 
\begin{tabular}{|p{0.8in}|p{1.6in}|p{1.6in}|p{1.6in}|}
\hline
Id & User Action & Condition & System Response  \bl 
1D & In the "Artifacts" view, Select more than one test suite document and right-click on the selection. & A project is opened in TaRGeT. The project has at least two generated test suites. & A drop down menu is displayed. \bl 
\end{tabular}
\subsubsection*{Scenario UC_85_SC_5}
\begin{tabular}{p{1in}p{4in}}
{\bf Description:} & Delete one test suite and one Use Case Document that is being referenced. \\
{\bf From steps:} & START \\
{\bf To steps:} & END \\
\end{tabular}
 
\begin{tabular}{|p{0.8in}|p{1.6in}|p{1.6in}|p{1.6in}|}
\hline
Id & User Action & Condition & System Response  \bl 
1E & In the "Artifacts" view, Select one test suite document and a Use Case Document, and right-click on the selection. & A project is opened in TaRGeT. The project has at least two imported Use Case Document and at least one generated test suite. The imported Use Case Document is referenced by another imported document. & A drop down menu is displayed. \bl 
2E & Click on "Delete" option in the drop down menu. [FR_TARGET_0003] & - & A dialog is displayed asking a confirmation and informing that the document is being referred by another document. \bl 
3E & Click on "Yes" button. & - & The document(s) is(are) deleted. The project work area is refreshed. Some invalid reference errors are displayed in "Error" view. [FR_TARGET_0015, FR_TARGET_0120] \bl 
\end{tabular}
\subsubsection*{Scenario UC_85_SC_6}
\begin{tabular}{p{1in}p{4in}}
{\bf Description:} & Delete two Use Case Documents. \\
{\bf From steps:} & START \\
{\bf To steps:} & 2C \\
\end{tabular}
 
\begin{tabular}{|p{0.8in}|p{1.6in}|p{1.6in}|p{1.6in}|}
\hline
Id & User Action & Condition & System Response  \bl 
1F & In the "Artifacts" view, Select one test suite two Use Case Documents, and right-click on the selection. One of the selected documents refers to the other. & A project is opened in TaRGeT. The project has two imported Use Case Document. One imported Use Case Document is referenced by the other. & A drop down menu is displayed. \bl 
\end{tabular}
\subsubsection*{Scenario UC_85_SC_7}
\begin{tabular}{p{1in}p{4in}}
{\bf Description:} & Cancel the document deletion. \\
{\bf From steps:} & 2M,2C \\
{\bf To steps:} & END \\
\end{tabular}
 
\begin{tabular}{|p{0.8in}|p{1.6in}|p{1.6in}|p{1.6in}|}
\hline
Id & User Action & Condition & System Response  \bl 
1A & Click on "No" button. & - & The document(s) is (are) not deleted. \bl 
\end{tabular}
\subsection*{Use case UC_106}
\begin{itemize}
\item {\bf Name: }Searching in Use Case Document
\item {\bf Description: }Searching in Use Case Document.
\end{itemize}
\subsubsection*{Scenario UC_106_SC_1}
\begin{tabular}{p{1in}p{4in}}
{\bf Description:} & Perform a search when some valid Use Case Documents are already imported. \\
{\bf From steps:} & START \\
{\bf To steps:} & END \\
\end{tabular}
 
\begin{tabular}{|p{0.8in}|p{1.6in}|p{1.6in}|p{1.6in}|}
\hline
Id & User Action & Condition & System Response  \bl 
1M & Choose "Tools" option in the menu bar. & TaRGeT is started up. A project is already opened. There is at least one document imported. No error is listed in the "Error" list. & A drop down menu is displayed. "Search" option is available. \bl 
2M & Choose "Search" option in the drop down menu. & - & "Search" window is displayed. [FR_TARGET_0135] \bl 
3M & Type a query in the input area. & - & The "Find" field is filled. \bl 
4M & Click on "Search" button. & - & The search results are displayed in "Search Results" View. Verify if the results are in accordance to the query. \bl 
\end{tabular}
\subsubsection*{Scenario UC_106_SC_2}
\begin{tabular}{p{1in}p{4in}}
{\bf Description:} & Cancel the search. \\
{\bf From steps:} & 3M \\
{\bf To steps:} & END \\
\end{tabular}
 
\begin{tabular}{|p{0.8in}|p{1.6in}|p{1.6in}|p{1.6in}|}
\hline
Id & User Action & Condition & System Response  \bl 
1A & Click on "Cancel" button. & - & The focus goes back to the work area. \bl 
\end{tabular}
\subsubsection*{Scenario UC_106_SC_3}
\begin{tabular}{p{1in}p{4in}}
{\bf Description:} & Using the specific field "Use Case Identifier". \\
{\bf From steps:} & 2M \\
{\bf To steps:} & END \\
\end{tabular}
 
\begin{tabular}{|p{0.8in}|p{1.6in}|p{1.6in}|p{1.6in}|}
\hline
Id & User Action & Condition & System Response  \bl 
1B & "Type a query for serching based on use case Id, i.e. "ucid:<use_case_id>". A use case with id <use_case_id> must be contained in the project. & - & The "Find" field is filled. \bl 
2B & Click on "Search" button. & - & At least one use case is displayed in "Search Results" View.  [FR_TARGET_0135] \bl 
\end{tabular}
\subsubsection*{Scenario UC_106_SC_4}
\begin{tabular}{p{1in}p{4in}}
{\bf Description:} & Using the specific field "From Step". \\
{\bf From steps:} & 2M \\
{\bf To steps:} & 4M \\
\end{tabular}
 
\begin{tabular}{|p{0.8in}|p{1.6in}|p{1.6in}|p{1.6in}|}
\hline
Id & User Action & Condition & System Response  \bl 
1C & Type "from step: START" in the input area. & - & The "Find" field is filled. \bl 
\end{tabular}
\subsubsection*{Scenario UC_106_SC_5}
\begin{tabular}{p{1in}p{4in}}
{\bf Description:} & Using shortcut. \\
{\bf From steps:} & START \\
{\bf To steps:} & 3M \\
\end{tabular}
 
\begin{tabular}{|p{0.8in}|p{1.6in}|p{1.6in}|p{1.6in}|}
\hline
Id & User Action & Condition & System Response  \bl 
1D & Press CTRL+F & - & "Search" window is displayed. [FR_TARGET_0135] \bl 
\end{tabular}
\subsection*{Use case UC_107}
\begin{itemize}
\item {\bf Name: }Viewing Search Results
\item {\bf Description: }Viewing search results.
\end{itemize}
\subsubsection*{Scenario UC_107_SC_1}
\begin{tabular}{p{1in}p{4in}}
{\bf Description:} & View found Use Case Documents in HTML format by double-clicking the use case. \\
{\bf From steps:} & UC_106#4M \\
{\bf To steps:} & END \\
\end{tabular}
 
\begin{tabular}{|p{0.8in}|p{1.6in}|p{1.6in}|p{1.6in}|}
\hline
Id & User Action & Condition & System Response  \bl 
1M & Double-click on a use case in the "Search Results" view. & - & The selected Use Case is displayed in HTML format in the main panel. The search results are highlighted. [FR_TARGET_0130] \bl 
\end{tabular}
\subsection*{Use case UC_135}
\begin{itemize}
\item {\bf Name: }Viewing the About Window
\item {\bf Description: }Displaying the About Window.
\end{itemize}
\subsubsection*{Scenario UC_135_SC_1}
\begin{tabular}{p{1in}p{4in}}
{\bf Description:} & Access the about window. \\
{\bf From steps:} & START \\
{\bf To steps:} & END \\
\end{tabular}
 
\begin{tabular}{|p{0.8in}|p{1.6in}|p{1.6in}|p{1.6in}|}
\hline
Id & User Action & Condition & System Response  \bl 
1M & Choose "Help" option in the menu bar. & The TaRGeT is already started up. & A drop down menu is displayed. [FR_TARGET_0205] \bl 
2M & Choose "About TaRGeT" option in the drop down menu. & - & "About" window is displayed. The content of the windows is detailed in TaRGeT requirements document. [FR_TARGET_0210] \bl 
\end{tabular}
\subsection*{Use case UC_140}
\begin{itemize}
\item {\bf Name: }Viewing the Help Window
\item {\bf Description: }Displaying the Help Window.
\end{itemize}
\subsubsection*{Scenario UC_140_SC_1}
\begin{tabular}{p{1in}p{4in}}
{\bf Description:} & Access the help window. \\
{\bf From steps:} & START \\
{\bf To steps:} & END \\
\end{tabular}
 
\begin{tabular}{|p{0.8in}|p{1.6in}|p{1.6in}|p{1.6in}|}
\hline
Id & User Action & Condition & System Response  \bl 
1M & Choose "Help" option in the menu bar. & The TaRGeT is already started up. & A drop down menu is displayed. \bl 
2M & Choose "Help Contents" option in the drop down menu. & - & "Help" window is displayed. [FR_TARGET_0205] \bl 
\end{tabular}
\subsection*{Use case UC_01}
\begin{itemize}
\item {\bf Name: }Opening TaRGeT
\item {\bf Description: }It describes how the TaRGeT should behave when it is started up.
\end{itemize}
\subsubsection*{Scenario UC_01_SC_3}
\begin{tabular}{p{1in}p{4in}}
{\bf Description:} & Start up the TaRGeT when Java runtime environment is not installed in the machine. \\
{\bf From steps:} & START \\
{\bf To steps:} & END \\
\end{tabular}
 
\begin{tabular}{|p{0.8in}|p{1.6in}|p{1.6in}|p{1.6in}|}
\hline
Id & User Action & Condition & System Response  \bl 
1B & Try to start the TaRGeT. & The Java runtime environment is not installed in the machine (see requirements for further information about Java version). [ER_TARGET_0020] & A message is displayed informing that no JRE is installed. The TaRGeT is not started up. [FR_TARGET_0230] \bl 
\end{tabular}
\subsection*{Use case UC_130}
\begin{itemize}
\item {\bf Name: }Generating Test Suites Changing Test Case Field Parameters
\item {\bf Description: }Configuring test case fields.
\end{itemize}
\subsubsection*{Scenario UC_130_SC_1}
\begin{tabular}{p{1in}p{4in}}
{\bf Description:} & Configuring test cases parameters. \\
{\bf From steps:} & START \\
{\bf To steps:} & END \\
\end{tabular}
 
\begin{tabular}{|p{0.8in}|p{1.6in}|p{1.6in}|p{1.6in}|}
\hline
Id & User Action & Condition & System Response  \bl 
1M & Go to "Tools" in the menu bar. & There is an already created project with at least one imported use case without errors. [FR_TARGET_0100] & A drop down menu is displayed with "Search", "Preferences" and "On The Fly Generation" options. The "Preferences" option is available. \bl 
2M & Select "Preferences" option. & - & The Preferences wizard is displayed with "Test Case ID", "Test Case Initial ID", "Empty Field", "Objective Prefix", "Print Use Case Description", "Print Flow Description" and "Keep Requirements" fields. \bl 
\end{tabular}
\subsubsection*{Scenario UC_130_SC_2}
\begin{tabular}{p{1in}p{4in}}
{\bf Description:} & Cancel the preferences. \\
{\bf From steps:} & 2B,2C,2D,2E,2F,2G,2H,2I,2J,2K \\
{\bf To steps:} & END \\
\end{tabular}
 
\begin{tabular}{|p{0.8in}|p{1.6in}|p{1.6in}|p{1.6in}|}
\hline
Id & User Action & Condition & System Response  \bl 
1A & Press "Cancel" button. & - & The field was not changed, the preferences screen is closed and the focus goes back to the main window. [FR_TARGET_0231] \bl 
\end{tabular}
\subsubsection*{Scenario UC_130_SC_3}
\begin{tabular}{p{1in}p{4in}}
{\bf Description:} & Changing "Objective Prefix" field. \\
{\bf From steps:} & 2M \\
{\bf To steps:} & END \\
\end{tabular}
 
\begin{tabular}{|p{0.8in}|p{1.6in}|p{1.6in}|p{1.6in}|}
\hline
Id & User Action & Condition & System Response  \bl 
1B & Type a wanted prefix in the "Objective prefix" field and press OK button. & The "Print Flow Description" field in the Preferences window  cannot be "None" & The preferences configuration is changed. [FR_TARGET_0231] \bl 
2B & Go to "On The Fly Generation" and select each test case from the list. & - & All test cases contain their objective field with the prefix defined by the user. [FR_TARGET_0235, ,FR_TARGET_0237] \bl 
\end{tabular}
\subsubsection*{Scenario UC_130_SC_4}
\begin{tabular}{p{1in}p{4in}}
{\bf Description:} & Changing "Empty Field" parameter. \\
{\bf From steps:} & 2M \\
{\bf To steps:} & END \\
\end{tabular}
 
\begin{tabular}{|p{0.8in}|p{1.6in}|p{1.6in}|p{1.6in}|}
\hline
Id & User Action & Condition & System Response  \bl 
1C & Type a wanted content in the "Empty Field" field and press OK button. & The "Print Flow Description" and "Print Use case description" fields is configured to "None" & The preferences configuration is changed. [FR_TARGET_0231] \bl 
2C & Go to "On The Fly Generation" and select each test case from the list. & - & In all test cases, the fields that are empty in the Use Case Document are filled with this content defined by the user. [FR_TARGET_0235, ,FR_TARGET_0237] \bl 
\end{tabular}
\subsubsection*{Scenario UC_130_SC_5}
\begin{tabular}{p{1in}p{4in}}
{\bf Description:} & Changing "Test Case Id" parameter to a fixed value. \\
{\bf From steps:} & 2M \\
{\bf To steps:} & END \\
\end{tabular}
 
\begin{tabular}{|p{0.8in}|p{1.6in}|p{1.6in}|p{1.6in}|}
\hline
Id & User Action & Condition & System Response  \bl 
1D & Type "<tc_featureid>_TESTE_<tc_id>" in the "Test Case Id" field and press OK button. & - & The preferences configuration is changed. [FR_TARGET_0231] \bl 
2D & Go to "On The Fly Generation" and select each test case from the list. & - & All Test Case Id fields are filled according the standard defined by the user. [FR_TARGET_0235, ,FR_TARGET_0237] \bl 
\end{tabular}
\subsubsection*{Scenario UC_130_SC_6}
\begin{tabular}{p{1in}p{4in}}
{\bf Description:} & Changing "Test Case initial Id" parameter \\
{\bf From steps:} & 2M \\
{\bf To steps:} & END \\
\end{tabular}
 
\begin{tabular}{|p{0.8in}|p{1.6in}|p{1.6in}|p{1.6in}|}
\hline
Id & User Action & Condition & System Response  \bl 
1E & Type an initial test case id (number) in the "Test Case Initial Id" field and press OK button. & - & The preferences configuration is changed. [FR_TARGET_0231] \bl 
2E & Go to "On The Fly Generation" and select each test case from the list. & - & All Test Case Ids are filled according the test case initial id (crescent order) defined by the user. [FR_TARGET_0235, ,FR_TARGET_0237] \bl 
\end{tabular}
\subsubsection*{Scenario UC_130_SC_7}
\begin{tabular}{p{1in}p{4in}}
{\bf Description:} & Changing "Print Use Case Description" parameter to "All". \\
{\bf From steps:} & 2M \\
{\bf To steps:} & END \\
\end{tabular}
 
\begin{tabular}{|p{0.8in}|p{1.6in}|p{1.6in}|p{1.6in}|}
\hline
Id & User Action & Condition & System Response  \bl 
1F & Choose the "ALL" option in the "Print Use Case Description" field and press OK button. & - & The preferences configuration is changed. [FR_TARGET_0232] \bl 
2F & Go to "On The Fly Generation" and select each test case from the list. & - & All descriptions of all Use cases related to the test case are concatenated in the Case Description field for each test case. [FR_TARGET_0235, ,FR_TARGET_0237] \bl 
\end{tabular}
\subsubsection*{Scenario UC_130_SC_8}
\begin{tabular}{p{1in}p{4in}}
{\bf Description:} & Changing "Print Use Case Description" parameter to "Last". \\
{\bf From steps:} & 2M \\
{\bf To steps:} & END \\
\end{tabular}
 
\begin{tabular}{|p{0.8in}|p{1.6in}|p{1.6in}|p{1.6in}|}
\hline
Id & User Action & Condition & System Response  \bl 
1G & Choose the  "Last" option in the "Print Use Case Description" field and press OK button. & - & The preferences configuration is changed. [FR_TARGET_0232] \bl 
2G & Go to "On The Fly Generation" and select each test case from the list. & - & Just the description of last Use case related to each test case is shown in the Case Description field. [FR_TARGET_0235, ,FR_TARGET_0237] \bl 
\end{tabular}
\subsubsection*{Scenario UC_130_SC_9}
\begin{tabular}{p{1in}p{4in}}
{\bf Description:} & Changing "Print Use Case Description" parameter to "None". \\
{\bf From steps:} & 2M \\
{\bf To steps:} & END \\
\end{tabular}
 
\begin{tabular}{|p{0.8in}|p{1.6in}|p{1.6in}|p{1.6in}|}
\hline
Id & User Action & Condition & System Response  \bl 
1H & Choose the  "None" option in the "Print Use Case Description" field and press OK button. & - & The preferences configuration is changed. [FR_TARGET_0232] \bl 
2H & Go to "On The Fly Generation" and select each test case from the list. & - & The Case Description field does not contain any description and it is filled with "Empty Field" suggested in Preferences window. [FR_TARGET_0235, ,FR_TARGET_0237, FR_TARGET_231] \bl 
\end{tabular}
\subsubsection*{Scenario UC_130_SC_10}
\begin{tabular}{p{1in}p{4in}}
{\bf Description:} & Changing "Print Flow Description" parameter to "All". \\
{\bf From steps:} & 2M \\
{\bf To steps:} & END \\
\end{tabular}
 
\begin{tabular}{|p{0.8in}|p{1.6in}|p{1.6in}|p{1.6in}|}
\hline
Id & User Action & Condition & System Response  \bl 
1I & Choose the "ALL" option in the "Print Flow Description" field and press OK button. & - & The preferences configuration is changed. [FR_TARGET_0232] \bl 
2I & Go to "On The Fly Generation" and select each test case from the list. & - & All descriptions of all Flows related to each test case are concatenated after "Objective Prefix" in the respective field. [FR_TARGET_0235, ,FR_TARGET_0237] \bl 
\end{tabular}
\subsubsection*{Scenario UC_130_SC_11}
\begin{tabular}{p{1in}p{4in}}
{\bf Description:} & Changing "Print Flow Description" parameter to "Last". \\
{\bf From steps:} & 2M \\
{\bf To steps:} & END \\
\end{tabular}
 
\begin{tabular}{|p{0.8in}|p{1.6in}|p{1.6in}|p{1.6in}|}
\hline
Id & User Action & Condition & System Response  \bl 
1J & Choose the "Last" option in the "Print Flow Description" field and press OK button. & - & The preferences configuration is changed. [FR_TARGET_0232] \bl 
2J & Go to "On The Fly Generation" and select each test case from the list. & - & Just description of last Flow related to each test case is shown after "Objective Prefix" in the respective field. [FR_TARGET_0235, ,FR_TARGET_0237] \bl 
\end{tabular}
\subsubsection*{Scenario UC_130_SC_12}
\begin{tabular}{p{1in}p{4in}}
{\bf Description:} & Changing "Print Flow Description" parameter to "None". \\
{\bf From steps:} & 2M \\
{\bf To steps:} & END \\
\end{tabular}
 
\begin{tabular}{|p{0.8in}|p{1.6in}|p{1.6in}|p{1.6in}|}
\hline
Id & User Action & Condition & System Response  \bl 
1K & Choose the "None" option in the "Print Flow Description" field and press OK button. & - & The preferences configuration is changed. [FR_TARGET_0232] \bl 
2K & Go to "On The Fly Generation" and select each test case from the list. & - & The Flow Description field does not contain any description and it is filled with "Empty Field" suggested in Preferences window. [FR_TARGET_0235, ,FR_TARGET_0237, FR_TARGET_231] \bl 
\end{tabular}
\subsection*{Use case UC_90}
\begin{itemize}
\item {\bf Name: }Updating the Use Case Documents outside the TaRGeT UI
\item {\bf Description: }Updating Use Case Documents outside TaRGeT UI.
\end{itemize}
\subsubsection*{Scenario UC_90_SC_1}
\begin{tabular}{p{1in}p{4in}}
{\bf Description:} & Edit a Use Case Document with duplicated step ID outside TaRGeT UI. \\
{\bf From steps:} & START \\
{\bf To steps:} & END \\
\end{tabular}
 
\begin{tabular}{|p{0.8in}|p{1.6in}|p{1.6in}|p{1.6in}|}
\hline
Id & User Action & Condition & System Response  \bl 
1M & Open an already imported Use Case Document with the {input} external editor. The document has a duplicated step ID. [FR_TARGET_0005] & TaRGeT is started up and a project is opened. There is at least one imported Use Case Document with duplicated step ID in the project. & The document is opened in the default viewer outside the TaRGeT. \bl 
2M & Edit the duplicated ID, save the document and close it. & - & The document is fixed. \bl 
3M & Go back to the TaRGeT window. & - & A progress bar is displayed while the project is being refreshed. [FR_TARGET_0110, FR_TARGET_0015] \bl 
4M & Wait the progress bar. & - & The fixed duplicated ID error is removed from the "Error" view. \bl 
\end{tabular}
\subsection*{Use case UC_95}
\begin{itemize}
\item {\bf Name: }Rejecting a crashed Use Case Documents
\item {\bf Description: }Rejecting crashed Use Case Documents.
\end{itemize}
\subsubsection*{Scenario UC_95_SC_1}
\begin{tabular}{p{1in}p{4in}}
{\bf Description:} & Damage an imported Use Case Document. \\
{\bf From steps:} & START \\
{\bf To steps:} & END \\
\end{tabular}
 
\begin{tabular}{|p{0.8in}|p{1.6in}|p{1.6in}|p{1.6in}|}
\hline
Id & User Action & Condition & System Response  \bl 
1M & Double-click on a Use Case Document in the "Artifacts" view. & A project is opened in TaRGeT. There is one imported and valid Use Case Document. & The Microsoft Word is invoked and the document is opened. [FR_TARGET_0005] \bl 
2M & Change the document, damaging its structure, and go back to the TaRGeT window. & - & A progress bar is displayed while the project is being refreshed. [FR_TARGET_0110, FR_TARGET_0015] \bl 
3M & Wait the progress bar. & - & A new error is displayed in the "Error" view indicating that the document is damaged. No information related to this document is displayed in the "Use Case" view. An error icon is assigned to the document in the "Artifacts" view. [FR_TARGET_0120, FR_TARGET_0125] \bl 
\end{tabular}
\subsection*{Use case UC_100}
\begin{itemize}
\item {\bf Name: }Rejecting a Use Case Document With Duplicated Feature ID
\item {\bf Description: }Rejecting Use Case Documents that assign to a feature an ID that is being used by a feature of another document.
\end{itemize}
\subsubsection*{Scenario UC_100_SC_1}
\begin{tabular}{p{1in}p{4in}}
{\bf Description:} & Assign to a feature an ID that is being used by a feature of another document. \\
{\bf From steps:} & START \\
{\bf To steps:} & END \\
\end{tabular}
 
\begin{tabular}{|p{0.8in}|p{1.6in}|p{1.6in}|p{1.6in}|}
\hline
Id & User Action & Condition & System Response  \bl 
1M & Double-click on a Use Case Document in the "Artifacts" view. & A project is opened in TaRGeT. There is two imported and valid Use Case Document. & The Microsoft Word is invoked and the document is opened. [FR_TARGET_0005] \bl 
2M & Change the document, by assigning to a feature an ID that is being used by a feature of another document, and go back to the TaRGeT window. & - & A progress bar is displayed while the project is being refreshed. [FR_TARGET_0110, FR_TARGET_0015] \bl 
3M & Wait the progress bar. & - & A new error is displayed in the "Error" view indicating that one of the documents contains a duplicated feature ID. The document is rejected and no information related to the document is displayed in the "Use Case" view. An error icon is assigned to the document in the "Artifacts" view. [FR_TARGET_0120, FR_TARGET_0125] \bl 
\end{tabular}
\subsection*{Use case UC_168}
\begin{itemize}
\item {\bf Name: }Viewing a Generated Test Suite in HTML Format
\item {\bf Description: } Viewing Test Suite Documents in HTML format with
	    default browser in the Operational System.
\end{itemize}
\subsubsection*{Scenario SC_1}
\begin{tabular}{p{1in}p{4in}}
{\bf Description:} & View Test Suite Document by double-clicking.
	     \\
{\bf From steps:} & START \\
{\bf To steps:} & END \\
\end{tabular}
 
\begin{tabular}{|p{0.8in}|p{1.6in}|p{1.6in}|p{1.6in}|}
\hline
Id & User Action & Condition & System Response  \bl 
1M & Double-click on a test suite document in the "Artifacts"
		view. & A project is opened in TaRGeT. The project has at least
		one already generated test suite in HTML format. & The Operational System's default browser to open HTML
		files is invoked and the document is opened. The Test Cases and
		Traceability Matrixes appear in the document.[FR_TARGET_0005,
		FR_TARGET 0243, FR_TARGET 0175, FR_TARGET 0240,] \bl 
\end{tabular}
\subsection*{Use case UC_169}
\begin{itemize}
\item {\bf Name: }Basic Test Suite Generation 
\item {\bf Description: }Generating Test Cases through Basic Generation.
	  
\end{itemize}
\subsubsection*{Scenario SC_01}
\begin{tabular}{p{1in}p{4in}}
{\bf Description:} & Generating Test Cases through Basic Generation.
	     \\
{\bf From steps:} & START \\
{\bf To steps:} & END \\
\end{tabular}
 
\begin{tabular}{|p{0.8in}|p{1.6in}|p{1.6in}|p{1.6in}|}
\hline
Id & User Action & Condition & System Response  \bl 
1M & Choose the "Tools" option in the menu bar.  & The TaRGeT is already started up. There is an already
		created project with at least one imported use case without
		errors. [FR_TARGET_0100] & A drop down menu is displayed with "Search",
		"Preferences" and "Basic Generation" options. The "Basic
		Generation" option is available.  \bl 
\end{tabular}
\end{document}